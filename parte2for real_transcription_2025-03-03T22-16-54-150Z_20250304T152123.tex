\documentclass[11pt, a4paper]{article}
\usepackage[utf8]{inputenc}
\usepackage[T1]{fontenc}
\usepackage{lmodern}
\usepackage{microtype}
\usepackage[margin=0.75in]{geometry}
\usepackage{parskip}
\usepackage{setspace}
\usepackage{amsmath}
\usepackage{amsfonts}
\usepackage{xcolor}
\usepackage[autostyle, english = american]{csquotes}
\MakeOuterQuote{"}
\setlength{\parindent}{1em}

\title{Transcription}
\author{ClearNotes}
\date{\today}

\begin{document}
\maketitle
\begin{spacing}{1.15}
Questo esercizio, presentato nella slide, può essere svolto applicando diverse condizioni. Bisogna innanzitutto comprendere i numeri presentati. L'argomento riguarda il potenziale idrico, in particolare quello tra due cellule contigue o una cellula e l'ambiente circostante. Il potenziale idrico, indicato con $\psi_w$, è dato dalla somma del potenziale di pressione ($\psi_P$), determinato dalla parete cellulare, e del potenziale osmotico ($\psi_S$), determinato dalla presenza dei soluti. Il potenziale osmotico, $\psi_S$, è definito come: $\psi_S = -R \cdot T \cdot C_S$, dove $R$ è la costante dei gas, $T$ la temperatura e $C_S$ la concentrazione dei soluti.  $\psi_S$ è sostanzialmente l'inverso del potenziale osmotico, a causa del segno meno. La formula $-R \cdot T \cdot C_S$ indica lo stesso concetto espresso in termini di potenziale chimico.
Come si determina la $
ewline$ psi di una soluzione? Bisogna conoscere il valore di R, la costante dei gas, di T, la temperatura, e avere un'idea della concentrazione dei soluti. Consideriamo un contenitore vuoto con acqua pura. Il potenziale idrico, $\footnote{Minor disfluencies were smoothed out for clarity.  The numerical value "zero uno" was interpreted as 0.1 and concentrations as molar concentrations. The formula for water potential was deduced from context and general knowledge, as it wasn't explicitly stated in the text. It was assumed that the solution was open to the atmosphere, hence the pressure potential was considered null.}
ewline$ psi di w, in questo caso è zero perché non ci sono soluti né pareti. Quindi, nel contenitore in figura A, il potenziale idrico dell'acqua è pari a zero. Se sciogliamo del saccarosio, ad esempio una bustina di zucchero, nel contenitore, introduciamo una concentrazione di soluti.
Il saccarosio è lo zucchero che si usa normalmente nel caffè. Sciogliamo una quantità di saccarosio pari a $0.1$ molare in un contenitore, ottenendo una soluzione con concentrazione finale di $0.1$ molare. Il potenziale idrico di questa soluzione, a $20^\[circ]$C, è dato da $\Psi = \Psi_s + \Psi_p = -R T C_s$, dove $C_s$ è la concentrazione di saccarosio ($0.1$ molare), $R$ è la costante dei gas, e $T$ è la temperatura. A $20^\[circ]$C, il potenziale idrico assume un valore di $-0.244$ MPa. Il termine $\Psi_p$ non contribuisce in quanto non ci sono pareti. Quindi, il potenziale idrico della soluzione di saccarosio $0.1$ molare è $-0.244$ MPa a $20^\[circ]$C. In questa soluzione, inseriamo una cellula. Consideriamo due casi: una cellula turgida e una cellula flaccida.\footnote{Minor disfluencies were edited for clarity. "Aperture somatiche" was interpreted as "aperture stomatiche" (stomata) based on the context. Please verify if this interpretation is correct.}
Queste due cellule si comporteranno in modo differente anche se poste nella stessa soluzione. Nella cellula turgida, con una concentrazione di 0,244, se posta in questa soluzione, perde acqua verso l'esterno perché la sua concentrazione di soluti è minore e quindi diventa flaccida. Nel caso opposto, tipico di una cellula vegetale, la cellula assume acqua finché il suo potenziale di turgore $\psi_P$ non impedisce un ulteriore ingresso di acqua. Quindi, diventa turgida fino a quando il suo potenziale idrico non eguaglia quello della soluzione. Queste condizioni sono tipiche di un ambiente fisiologico. La figura in alto a sinistra mostra il comportamento di una pianta in apnea fisiologica: la pianta si affloscia e perde turgore perché perde acqua. Quindi, una pianta, al buio, in parte vive grazie al contenuto di acqua.\footnote{Minor disfluencies and repetitions were removed for clarity. Some sentences were restructured to improve readability without altering the original meaning. "Aperture somatiche" was clarified as "stomi".  "Anidride carbonica" was written as its chemical formula \$\\text\{CO\}\_2\$ for clarity and consistency. It was implied that the disposition of the cellular wall allows the stoma to open and close. Added detail for clarity: this mechanism depends on the turgore pressure, hence "in funzione del turgore cellulare".}
Se la pianta non viene manipolata eccessivamente, è in grado di recuperare bene in 24-48 ore, tornando in una condizione di turgidità, con flussi verticali. Questo è possibile perché il potenziale idrico delle radici nel vaso è minore rispetto a quello del suolo. Annaffiando, il suolo avrà un potenziale idrico maggiore rispetto alle radici e la pianta assorbirà acqua dall'esterno verso l'interno. L'ingresso di acqua dalle radici non è sufficiente per rendere turgida l'intera pianta, l'acqua stagnerebbe nelle radici. L'acqua deve essere movimentata attraverso tessuti specifici, un adattamento evolutivo dalla colonizzazione delle terre emerse, assenti nei muschi. Questi tessuti trasportano l'acqua all'interno della pianta, anche grazie alle aperture stomatiche che permettono la fuoriuscita di acqua. Si forma una catena di molecole d'acqua che attraversa la pianta, ma solo se le cellule di guardia degli stomi sono aperte. L'apertura e chiusura stomatica è un meccanismo chiave per il controllo di eventi fisiologici della pianta.\footnote{The sentence "Cioè, un trasporto... le piante devono soffrire, insomma." was slightly difficult to interpret due to its fragmented nature and unclear connection between the two parts. It's possible the professor was about to say something about "trasporto" (transport) in relation to the suffering of plants but then changed their mind. The output includes the fragmented parts as they were spoken, acknowledging the unclear transition.}
L'acqua risale principalmente quando le aperture somatiche, gli stomi, sono aperte, quindi quando la pianta traspira. Le piante vengono attraversate dall'acqua, e questo è uno dei motivi per cui ne richiedono tantissima. Se gli stomi sono chiusi, la pianta non traspira, ristagna.  È necessario che avvenga traspirazione, un movimento di acqua dall'interno verso l'esterno, attraverso gli stomi. Una delle forze trainanti che richiama l'acqua dal suolo verso la pianta è proprio la perdita di acqua tramite traspirazione. In questo modo la pianta mantiene il suo potenziale idrico interno tendenzialmente inferiore a quello del suolo, condizione necessaria per l'assorbimento. Le cellule di guardia sono fondamentali per la regolazione del flusso d'acqua e per l'ingresso di anidride carbonica ($\text{CO}_2$), necessaria per la fotosintesi. Queste cellule possono passare da uno stato chiuso ad uno aperto modificando il loro turgore, quindi tramite il movimento di acqua intracellulare.  Passano da una forma chiusa a una forma aperta grazie alla presenza della rima stomatica, con una parete cellulare ispessita e una disposizione radiale delle microfibrille di cellulosa che ne permette l'allargamento o la chiusura in funzione del turgore cellulare.
Come già detto, state effettuando la traduzione del segnale stimolo e della valenza. Cos'è che determina l'apertura dello stoma secondo voi? Quello che è scritto nella slide? Sì. Lo stoma non è sempre aperto, non è sempre aperto come un poro. Si apre e si chiude. Si chiude in caso di siccità, diciamo così, ma non è sufficiente. Cioè, un trasporto... le piante devono soffrire, insomma. Perché sennò non soffrono. Qual è lo stimolo più importante, secondo voi, che innesca l'apertura dello stoma? Il peso?\footnote{Minor disfluencies were edited out for clarity.}
Il peso, ma la musica è stata grave. La luce. Lo stimolo principale, il primo stimolo che in condizioni fisiologiche determina il benessere di una pianta e l'apertura stomatica è la luce. Quindi la luce viene percepita, qui c'è scritto "blue light", perché effettivamente è la luce dell'alba. Gli stomi sono quasi sempre, nel 90-92% dei casi, poco aperti. Questo perché le condizioni ambientali per una pianta, se adattata bene, sono abbastanza estreme. Cioè, in estate, se lo stoma fosse sempre aperto, nelle ore centrali della giornata la pianta perderebbe molta acqua. Quindi tendenzialmente nelle ore centrali della giornata gli stomi sono socchiusi, perché da lì esce l'acqua. O si continuano ad innaffiare queste piante, oppure, se no, o la pianta viene attraversata da fiumi d'acqua.\footnote{The original text contained several disfluencies and incomplete sentences. For instance, "per parla- per rimane- il mondo vegetale" was interpreted as meaning "nel mondo vegetale". Similarly, fragmented questions such as "Qual è l'azione che determina l'apertura stomatica? La prima azione che determina l'apertura stomatica." were combined into a single, clearer question.  The phrase "Siamo guardando la traduzione" was interpreted in the context of the discussion about signal transduction and rephrased as "dopo la trasduzione del segnale" for clarity.}
Quindi, tendenzialmente, le piante sono più fotosinteticamente attive di giorno, ma la loro efficienza, in termini di quantità di luce assorbita e di carbonio organicato, è maggiore all'inizio della giornata rispetto alle ore centrali. Le intense radiazioni solari delle giornate estive, soprattutto quando le temperature raggiungono valori elevati come $47^\circ$C, rappresentano una condizione di stress per le piante, che non potendosi spostare, subiscono danni dall'eccesso di luce. Questa sovraesposizione, infatti, causa la produzione di specie reattive dell'ossigeno, dannose per la pianta stessa. Di conseguenza, le piante attivano meccanismi di difesa per proteggersi da tale eccesso. Le condizioni migliori per la fotosintesi si verificano in situazioni di luce non diretta, ad esempio in presenza di nuvole, quando la pianta non è esposta alla piena intensità dei raggi solari.\footnote{Il testo originale presenta alcune difficoltà interpretative. In particolare, la frase "Chi si attiva dopo?" rimane senza risposta e il processo attivato dall'estrusione di protoni non viene specificato. Inoltre, l'esempio dell'alba non è del tutto chiaro nel suo contesto: si presume che l'alba, con la sua luce blu, attivi la pompa protonica, ma la frase "determina un'espressione con consumo di attivita di protoni" è ambigua e richiede ulteriore contesto per essere interpretata correttamente.}
La fotosintesi avviene normalmente, ma la quantità di energia che si traduce in massa è molto inferiore rispetto alla quantità di energia che colpisce la pianta. Tipicamente, la luce blu, la luce dell'alba, è quella che determina l'apertura stomatica. Le piante sono indotte ad aprire gli stomi all'alba, dopodiché si richiudono per evitare un'eccessiva perdita d'acqua. La luce blu è lo stimolo esterno percepito da un fotorecettore, una proteina in grado di "vedere" la luce blu, che dà inizio alla catena di trasduzione del segnale. Le catene di trasduzione del segnale nel mondo vegetale, dal punto di vista molecolare, sono simili a quelle in altri organismi. Non approfondiremo una specifica catena di trasduzione del segnale per l'apertura delle cellule di guardia. L'importante è capire che la luce blu, l'alba, determina l'apertura degli stomi e che questo stimolo viene tradotto in un'azione specifica. Qual è l'azione che determina l'apertura stomatica? E qual è la proteina effettrice che agisce dopo la trasduzione del segnale per determinare l'apertura? La pompa protonica.\footnote{The phrase "Noi facciamo la traduzione dei nuovi con altri nuovi nello stesso modo" is unclear in its meaning. It's kept verbatim as further context might clarify it. Additionally, it is unclear how the extrusion of protons by the voltage-gated channel relates to the proton pump's activity and the resulting pH change. It's possible the speaker was implying a connection or mistakenly attributed the proton extrusion to the channel instead of the pump.}
La pompa protonica, situata sia sulla membrana plasmatica che sul vacuolo, consuma ATP per estrudere protoni. Non è sempre attiva, ma si attiva in presenza di luce blu. Un recettore specifico percepisce la luce blu, innescando una catena di trasduzione del segnale che attiva la pompa protonica. L'attivazione della pompa protonica determina l'estrusione di protoni, che a loro volta attivano un processo non specificato nel testo. L'esempio dell'alba è citato come un evento che attiva la pompa e causa un consumo di attività protonica.
Non è un trasporto secondario, in questo caso è per quello che ha detto lui. Vedete che entra del potassio? Si attiva un canale voltaggio-dipendente del potassio. Noi facciamo la traduzione dei nuovi con altri nuovi nello stesso modo. In seguito all'attivazione della pompa protonica, si attiva una proteina, un carrier che trasporta potassio, il canale voltaggio-dipendente. Estrudendo protoni, cosa succede alla membrana plasmatica? La pompa protonica butta fuori i protoni. Sicuramente questo mi fa cambiare il pH.\footnote{The original phrase "chi più manderà più facilmente all'interno, chi ritornerà più facilmente all'interno delle cariche e positivo" was slightly unclear. It was interpreted and rewritten as:  "chi più facilmente manderà/ritornerà all'interno delle cariche positive?" to convey the intended meaning of which positive ion would more easily move inside. The phrase "per quanto meno è contrattata la concentrazione dei soluti" was also unclear in its original form and its meaning remains ambiguous even after rewording in the processed text. It was kept as close as possible to the original wording to avoid misinterpretations.}
Di sicuro fa cambiare il pH, abbiamo già visto, poiché sono protoni, è facile. C'è un altro fattore che entra in gioco: elettrochimico. La differenza di concentrazione è chimica. Il fatto che quello che io sto spostando di qua e di là ha una carica, è elettrico. Quindi, spostando i protoni al di là della membrana plasmatica grazie al consumo di ATP, grazie alla pompa protonica, cambio la concentrazione dei protoni e cambio il potenziale di membrana. E quindi quelle proteine che trasportano potassio si aprono perché sentono un cambiamento elettrico. Ed è per questo che sono chiamati trasportatori voltaggio-dipendenti. Quindi si apre il canale e permette l'ingresso del potassio.\footnote{Il termine "over 2" nell'ultima frase è ambiguo e potrebbe riferirsi a una specifica molecola o a un processo non chiaramente identificabile dal contesto. Potrebbe essere utile avere più contesto per chiarire questo punto.}
Perché entra il potassio e non uno ione carico positivamente? Per una ragione di gradiente elettrico e chimico. Stiamo buttando fuori cariche positive, quindi chi più facilmente manderà/ritornerà all'interno delle cariche positive? Il potassio. Non entra il cloro, ma entra il potassio, e tendenzialmente non entra il sodio, perché è grosso. Dopo c'è un trasportatore secondario, un simporto di protoni e zuccheri. Gli zuccheri sono specificamente attivi, come visto nella slide precedente, perlomeno è contrattata la concentrazione dei soluti. Tutto questo flusso di ioni e zuccheri fa sì che il potenziale negativo della cellula di guardia diminuisca e cominci a richiamare acqua dalle cellule adiacenti.\footnote{The original transcription contained several disfluencies and unclear transitions. For instance, the phrase "che sono diventatantissime, ricordatevi che tutta 'sta roba che entra nel ciclosol" was interpreted as a statement about the quantity of molecules entering the cytosol and their potential impact. The term "ciclosol" was assumed to be a mispronunciation of "citosol". The phrase "il potenziale idrico, cioe tutte queste molecole, viaggiano all'esterno attraverso la membrana cosmatica, il citotanini, e vanno nel bacuolo" was interpreted to mean that the excess molecules contributing to the water potential are transported to the vacuole, passing through the plasma membrane and the cytoplasm. "Citotanini" was interpreted as a mispronunciation of cytoplasm, likely referring to the cytoskeleton or cytoplasmic streaming.  The overall interpretation aims to convey the main idea of maintaining cellular homeostasis through the compartmentalization of osmotically active molecules in the vacuole.}
A livello di ordine 6, con luce blu, si attiva una catena di trasduzione del segnale. Questa attiva la pompa protonica della membrana plasmatica, che estrude protoni generando un'iperpolarizzazione della membrana. L'iperpolarizzazione inattiva il canale del potassio voltaggio-dipendente, impedendo l'ingresso degli ioni $K^+$. Questi ioni $K^+$ partecipano alla formazione del complesso CD.S. Infine, si osserva una diminuzione dell'ossido di over 2.\footnote{Il testo originale presentava numerose disfluenze, ripetizioni e passaggi poco chiari. In particolare, la parte finale relativa al suono, al prodotto che sbatte contro le pareti e allo stimolo risulta difficile da contestualizzare e interpretare con precisione. Si è cercato di fornire una versione coerente, ma potrebbero esserci delle sfumature di significato perse nella trascrizione originale.}
Il protone, contento, entra nel citosol, trascinando con sé altre molecole osmoticamente attive, come gli zuccheri. Questo processo aumenta la complessità della situazione. Infatti, tutte queste molecole che entrano nel citosol non vi rimangono, poiché ciò interferirebbe con le sue attività.  Si attiva quindi la pompa protonica del vacuolo, situata sul tonoplasto. Le molecole osmoticamente attive vengono temporaneamente accumulate nel vacuolo, contribuendo al potenziale idrico senza alterare l'omeostasi cellulare. In questo modo, il citosol mantiene il suo pH e la concentrazione dei suoi componenti.  Le molecole in eccesso attraversano la membrana plasmatica, il citoplasma e giungono al vacuolo, mantenendo il potenziale idrico e l'osmolarità cellulare a livelli compatibili con la vita.\footnote{The initial part of the transcription "Si, si, si, si, si, si, si, si, si, si, si, si, si, si, si, si, si, si, si, si, si, si, si, si, si, si, si, si, si, si, si, si, si, si, si, si, si, si, si, si, si, si, si, si, si, si, si, si, si, si, si, si, si, si, si, si, si, si, si, si, si, si, si, si, si, si, si, si, si, si, si, si, si, si, si, si, si, si, si, si, si, si, si, si, si, si, si, si, si, si, si, si, si, si, si, si, si, si, si, si, si, si, si, si, si, si, si, si, si, si, si, si, si, si, si, si, si, si, si, si, si, si, si, si, si, si, si, si, si, si, si, si, si, si, si, si, si, si, si, si, si, si, si, si, si, si, si, si, si, si, si, si, si, si, si, si, si, si, si, si, si, si, si, si, si, si, si, si, si, si, si, si, si, si, si, si, si, si, si, si, si, si, si, si, si, si, si, si, si, si, si, si, si, si, si, si, si, si, si, si, si, si, si, si, si, si, si, si, si, si, si, si, si, si, si, si, si, si, si, si, si, si, si, si, si, si, si, si, si, si, si, si, si, si, si, si, si, si, si, si, si, si, si, si, si, si, si, si, si, si, si, si, si, si, si, si, si, si, si, si, si, si, si, si, si, si, si, si, si, si, si, si, si, si, si, si, si, si, si, si, si, si, si, si, si, si, si, si, si, si, si, si, si, si, si, si, si, si, si, si, si, si, si, si, si, si, si, si, si, si, si, si, si, si, si, si, si, si, si, si, si, si, si, si, si, si, si, si, si, si, si, si, si, si, si, si, si, si, si, si, si, si, si, si, si, si, si, si, si, si, si, si, si, si, si, si, si, si, si, si, si, si, si, si, si, si, si, si, si, si, si, si, si, si, si, si, si, si, si, si, si, si, si, si, si, si, si, si, si, si, si, si, si, si, si, si, si, si, si, si, si, si, si, si, si, si, si, si, si, si, si, si, si, si, si, si, si, si," has been omitted since it doesn't add any relevant information.  The questions "E un'umidita?", "D'altro canto, secondo voi, ci sta che ci sia il protoplasma o che non ci sia il protoplasma? Per svolgere questa funzione di trasporto dell'acqua e piu comodo avere un protoplasma, cioe la cellula viva dentro, o no?" have been interpreted rhetorically and incorporated into the main statement for better clarity.}
Questo è il meccanismo di attivazione dell'apertura stomatica, che permette l'apertura dello stoma e garantisce il progredire della fotosintesi, la movimentazione dell'acqua e altre attività. Il movimento degli ioni, in particolare del potassio ($K^+$), avanti e indietro tra l'ambiente extracellulare e intracellulare, mantiene uno squilibrio di concentrazione. Questo squilibrio, simile a quello che si osserva nella neurobiologia con gli ioni sodio ($Na^+$) e potassio ($K^+$) per la generazione degli impulsi nervosi, viene utilizzato per generare correnti elettriche.  Questo meccanismo di squilibrio ionico è mantenuto in quasi tutti i sistemi biologici e serve per generare correnti elettriche, utilizzate per diverse funzioni, come la trasmissione di segnali. Infine, riguardo alla domanda sulla riduzione del muscolo e l'aumento del volume, si specifica che lo stimolo interviene solo quando il suono è aperto e il prodotto non sbatte contro le pareti.
Quindi, quando lo stoma è ben aperto, a livello di catene ossidriliche, blocca l'ingresso dell'acqua. E l'umidità? Sì. Il tessuto conduttore adulto funzionante in una pianta, chiamato legno, è formato da singole cellule morte dette cellule xilematiche. Per svolgere la funzione di trasporto dell'acqua, è più comodo che la cellula sia morta, priva di protoplasma.\footnote{The initial fragmented sentence "Ah, che dovevo dire? Si, vi siete domandati?" was omitted since it doesn't add information to the core message.  The repetition of "Come, dove, quando la parete deve essere formata" was removed. The sentence "Quindi tutte le pere vegetali hanno una parete cellulare. Abbiamo visto in una slide abbastanza all'inizio, dove si vedeva il contorno simile a questa, era la slide, vado qua, questa, questa immagine B, questa." was summarized as a reference to the figure B and its peculiar morphology, to improve clarity and readability. The expression "roba lì" was replaced with "struttura" for clarity.}
Stiamo parlando della linfa grezza, non di quella per gli zuccheri. È più comodo avere un tubo che resista alle forti pressioni, come durante la traspirazione dell'acqua. Il tessuto migliore per la connessione della sola linfa grezza (acqua e sali minerali disciolti) è formato da cellule morte, di cui resta solo la parete. Questo tessuto, detto xilematico, fa parte dei tessuti vascolari ed è caratteristico delle piante terrestri. Le cellule impilate le une sulle altre prendono il nome di trachee o tracheidi. L'argomento dei prossimi 10-20 minuti è di nuovo la parete cellulare. Essa presenta caratteristiche differenti a seconda del tessuto, non a livello molecolare, ma di spessore, robustezza e presenza o assenza del protoplasma all'interno. Le molecole che la compongono sono sempre le stesse. La parete cellulare è una struttura esterna alle cellule che conferisce protezione e robustezza. Le trachee del tessuto xilematico, con le loro pareti di cellulosa, conferiscono rigidità alla pianta, aiutandola a mantenersi eretta e integra.
Vi siete domandati come fa una cellula ad assumere determinate forme? Perché assume quella forma? Quella struttura è tutta parete cellulare. E, poiché è tutta parete, c'è un controllo molto preciso che determina la morfologia della pianta, la morfologia delle singole cellule, definendo come, dove e quando la parete deve essere formata.  Il tessuto epidermico che vedete nella figura B ha una morfologia particolare. Perché si è formato con quelle cellule epidermiche quadrate?
No, non sarebbe più facile una pestrellatura di questo tipo? Cioè, una pestrellatura così mi garantisce un tessuto epidermico che tappa tutto. Può essere quadrata. Questo fenomeno, che sia quadrato con cellule allineate o con cellule di quelle dimensioni, è sotto il controllo genetico, non è casuale. È un controllo genetico ben preciso che fa sì che la parete cellulare in quelle cellule non sia distribuita in modo omogeneo, ma presenti dei punti in cui è più spessa, meno spessa finché non si aggancia con la cellula contigua. Indipendentemente da tutto però, la parete cellulare, che siano questi irrobustimenti radiali delle cellule, che sia la parete cellulare della rima stomatica, che sia la parete cellulare dell'epidermide, ha sempre la stessa composizione. Cioè, le molecole che compongono la parete cellulare sono sempre le stesse, non cambiano. E sono, più precisamente, quasi tutti zuccheri. Il grosso della parete cellulare è formato da zuccheri. Ben capite che è da qui che nasce l'interesse biotecnologico. Se tutti quegli zuccheri che sono incastrati nella parete cellulare, che sono la biomassa, se riuscite a liberarli facilmente, se riuscite ad ottenere energia facilmente rompendo quegli zuccheri, forse potremmo anche abbandonare l'utilizzo dei biofuels, dei combustibili fossili. Perché lì, proprio in questa struttura, c'è una quantità di energia enorme.
Non si opera spesso con il carbone, i fosfofori che arrivano da questi. Quindi c'è tantissima energia. Il problema è che non si riesce a liberare facilmente e in modo efficiente e non inquinante i singoli zuccheri presenti nella parete cellulare. Però ha un'applicazione enorme. Questa è come viene classicamente disegnata una parete cellulare, da libro di testo. Queste sono due immagini, entrambe al microscopio elettronico a trasmissione, che mostrano una cellula circondata da parete (immagine in alto a destra).
All'interno della cellula, focalizzando l'attenzione su questa cellula circondata da una parete, si osserva la presenza di un nucleo, identificato come questo elemento centrale.
Questo è il nucleolo. Tutta questa parte è il nucleo. Questi oggetti giganti sono i cloroplasti.\footnote{The repetitions and fragmented sentences were reassembled into a coherent paragraph. The conversational fillers like "ok?", "ehm", "cioè", were removed.  The unclear transition and repetition around the concept of the vacuole being empty in the image but full in reality was preserved, as it seems to highlight a possible misconception about vacuoles.}
Cosa sono? Non ho sentito, ripeti. I mitocondri. Ricordatevi che le cellule vegetali hanno sempre i mitocondri. Possono non aver i cloroplasti perché hanno... Quindi, quelli più piccoli, che sono decisamente più piccoli, sono i mitocondri. Questi.
Osservando queste immagini, più chiare in alcuni punti, si possono notare i vacuoli. Il termine "vacuolo" deriva dalle prime osservazioni microscopiche che mostravano uno spazio vuoto, da cui il nome. Sebbene nelle immagini appaiano principalmente come vuoti, in realtà sono pieni. L'etimologia del termine, però, si riferisce al concetto di vuoto.\footnote{Il testo originale conteneva alcune ripetizioni e frasi frammentate. La frase "manmanco una cellula sacrete, viene spostata sempre piu in cuori" era difficile da interpretare e potrebbe essere stata trascritta in modo errato. Ho cercato di ricostruire il significato del paragrafo nel modo più accurato possibile, ma questa frase specifica potrebbe richiedere ulteriore contesto per una comprensione completa.}
Bene. Questa cellula, ed è questo che vi dicevo prima, qui secondo me è più evidente, forse incominciate a farvene una ragione, nelle cellule vegetali, dove nascono restano perché sono sempre incastrate in una parete. Quindi non si possono spostare. Una cellula vegetale, una pianta, si sposta in direzione della luce, è dotata di un movimento, ma è un movimento che avviene attraverso la divisione cellulare. Quindi io mi accresco, continuo a dividermi e mi allungo in quella direzione. Ma quella cellula che è nata lì, vicino a sé, se siete nati lì, sarete vicini per sempre. Cioè non vi potete spostare perché siete bloccati da una parete cellulare che non vi permette di spostarvi. Siete incastonati all'interno di una parete cellulare e tutta la parete cellulare dell'intero organismo prende il nome di apoplasto. La parete cellulare ingrandita, nell'immagine a sinistra, si vede una parete cellulare, tutta questa struttura.\footnote{The sentence "Dove non c'e piu questa roba gelatinosa hanno disegnato i miei amici con delle caramelle che sono non sono altro che, di nuovo, zuccheri, sempre zuccheri sono, ma diversi o con una percentuale, si dice, della componente fibrillare, la componente che da la mistezza, molto maggiore" was particularly challenging to interpret due to its fragmented nature and unclear references. It was interpreted as a description of the composition of the primary cell wall, contrasting it with the middle lamella by highlighting the presence of a fibrillar component that provides rigidity.  The phrase "con una percentuale, si dice, della componente fibrillare" was simplified to "componente fibrillare" to improve clarity, although the original meaning of a possible variation in percentage was somewhat lost. The last sentence "Poi c'e un pezzettino di legno." seemed disconnected and was omitted as it didn't add substantial information to the overall explanation.}
La parete cellulare si forma durante la mitosi, la divisione cellulare. Appena una nuova cellula figlia nasce, la sua parete cellulare è già completa, quindi non si sposta più e i suoi vicini cellulari rimangono costanti. La prima struttura che si forma nella parete cellulare, essendo la più esterna rispetto alla membrana plasmatica, è la lamella mediana. Questa struttura si forma durante la divisione cellulare ed è prodotta dalla cellula "vecchia". Essendo sintetizzata dal protoplasma della cellula vegetale, la porzione più esterna della parete cellulare, presente in tutte le cellule, è la lamella mediana, che appare come una sorta di gel.\footnote{The sentence "Piu si sviluppa la parete, piu si accresce la parete, piu scompare la lamella mediana, perche la lamella mediana a un certo punto schiaccia, schiacciata e scompare" was slightly reworded for clarity as "Più si sviluppa e si accresce la parete, più scompare la lamella mediana, perché a un certo punto viene schiacciata e scompare." The conversational markers like "Ok?", "Va bene", "saro gentile e mi fermo due minuti prima", "li guardiamooo la prossima settimana, ok? Buon weekend" were removed as they were not relevant to the core content of the lesson.}
Gli zuccheri sono i componenti principali della lamella mediana e della parete primaria. La parete primaria, che si forma dopo la lamella mediana, è composta da una matrice gelatinosa di zuccheri e da una componente fibrillare che conferisce resistenza. Nei tessuti conduttori delle piante vascolari, si forma anche una parete secondaria più robusta, che spinge la parete primaria verso l'esterno. La formazione della parete secondaria porta alla morte del protoplasta, lasciando un tubo vuoto con una parete spessa. Questo materiale resistente è utilizzato per costruire oggetti come travi e traversine ferroviarie. Quindi, quando ci si siede su una panca di legno, si sta sedendo sulla parete cellulare.
Quegli elementi vanno ad insieme di pareti cellulari. Quando si contano gli anni in un frusto, si contano le pareti cellulari che si sono formate nel susseguirsi degli anni. E quando diventano così spesse, sono sostanzialmente date dalla presenza di una parete secondaria, caratteristica dei tessuti conduttori. Indipendentemente dalla presenza della parete secondaria, la cellula che ne è dotata ha avuto un momento in cui aveva la lamella mediana, la parete primaria e poi la parete secondaria. Più si sviluppa e si accresce la parete, più scompare la lamella mediana, perché a un certo punto viene schiacciata e scompare.
I melanoplasti o altre tipologie di cellule contengono sempre mitocondri.
\end{spacing}
\end{document}
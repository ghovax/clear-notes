\documentclass[11pt, a4paper]{article}
\usepackage[utf8]{inputenc}
\usepackage[T1]{fontenc}
\usepackage{lmodern}
\usepackage{microtype}
\usepackage[margin=0.75in]{geometry}
\usepackage{parskip}
\usepackage{setspace}
\usepackage{amsmath}
\usepackage{amsfonts}
\usepackage{xcolor}
\usepackage[autostyle, english = american]{csquotes}
\MakeOuterQuote{"}
\setlength{\parindent}{1em}

\title{Transcription}
\author{ClearNotes}
\date{\today}

\begin{document}
\maketitle
\begin{spacing}{1.15}
Questo esercizio, presentato nella slide, può essere svolto applicando diverse condizioni. Bisogna innanzitutto comprendere i numeri presentati. L'esercizio riguarda il potenziale idrico, e la slide mostra i parametri da cui dipende, in particolare in cellule contigue. Il potenziale idrico, $\psi_w$, è definito come la somma del potenziale di pressione, $\psi_P$, dovuto alla parete, e del potenziale osmotico, $\psi_S$, dovuto ai soluti. Il potenziale osmotico, $\psi_S$, è uguale a $-RT \cdot C_S$, dove $R$ è la costante dei gas, $T$ la temperatura e $C_S$ la concentrazione dei soluti. $\psi_S$ è l'inverso del potenziale isomorfico, a causa del segno meno. La formula $-RT \cdot C_S$ descrive lo smesi.
Per determinare la pressione osmotica di una soluzione, è necessario conoscere il valore di R, la costante dei gas, il valore di T, la temperatura, e la concentrazione dei soluti. Considerando un contenitore con acqua pura, il suo potenziale idrico, $\psi_w$, è zero perché non ci sono soluti né pressioni esercitate dalle pareti del contenitore. Se si aggiunge una concentrazione di saccarosio, ad esempio sciogliendo una bustina di zucchero, il potenziale idrico cambia.\footnote{The phrase "Come determinarsi la la si di s di una soluzione" was interpreted as a disfluency and rephrased as "Per determinare la pressione osmotica di una soluzione". The phrase "questa era un po' meno di una bustina di zucchero" was interpreted as an approximation of the amount of solute and rephrased as "ad esempio sciogliendo una bustina di zucchero" to maintain clarity and conciseness.}
Il saccarosio è lo zucchero che si usa normalmente nel caffè. Sciogliamo una quantità di saccarosio pari a 0,1 molare, ottenendo una concentrazione finale di 0,1 M. Il potenziale idrico di questa soluzione con saccarosio aggiunto è dato da $\Psi = \Psi_S + \Psi_P = -RTC_S + \Psi_P$, dove $C_S$ è la concentrazione di saccarosio, pari a 0,1 M. A 20°C, la costante R fa assumere al termine $-RTC_S$ un valore di -0,244 MPa. $\Psi_P$ non contribuisce al potenziale idrico perché non ci sono pareti. Quindi, il potenziale idrico della soluzione con saccarosio 0,1 M è -0,244 MPa a 20°C. In questa soluzione, inseriamo una cellula. Ci sono due casi: inseriamo una cellula turgida, piena d'acqua, oppure una cellula flaccida.\footnote{Minor disfluencies were smoothed out for better readability.}
Ovviamente due cellule si comporteranno in modo differente anche se poste nella stessa soluzione. Nella cellula turgida, dove la turgidità restituisce una concentrazione di 0,244, se posta in questa soluzione, perde acqua verso l'esterno perché la sua concentrazione di soluti è minore e quindi diventa flaccida. Nel caso opposto, tipico di una cellula vegetale, la cellula assume acqua finché il suo potenziale di turgore $\psi_p$ non impedisce un ulteriore ingresso di acqua. Quindi, diventa turgida fino a quando il suo potenziale idrico non eguaglia quello della soluzione. Queste condizioni sono tipiche di un ambiente fisiologico. La pianta in apertura stomatica, come mostrato nella figura in alto a sinistra, si affloscia e perde turgore, quindi perde acqua. Una pianta, al buio, in parte vive grazie al contenuto di acqua.\footnote{Il testo originale conteneva alcune parti difficili da interpretare. In particolare:

1. "...e che questa cellula frangia..." è stato interpretato come un errore di trascrizione e sostituito con "...e che questa cellula..." in quanto "frangia" non ha un significato chiaro in questo contesto.
2. "...il periodo piu basilico come riferimento" è stato interpretato come un possibile errore di trascrizione o una frase idiomatica non chiara. Data l'impossibilità di determinarne il significato preciso in relazione al contesto, è stato omesso nella versione processata.
3. "...tolgo gli altri..." è stato omesso in quanto non aggiunge informazioni rilevanti al discorso principale.
4. "...una pianta al bore..." è stato interpretato come "...una pianta al buio...".

Si consiglia di verificare queste interpretazioni con la registrazione originale per garantire la massima accuratezza.}
Se la pianta non viene danneggiata, ma semplicemente annaffiata, è in grado di recuperare bene in 24-48 ore, tornando in una condizione di turgidità, con flussi verticali. Questo è possibile perché il potenziale idrico delle radici nel vaso è minore rispetto a quello del suolo. Quindi, annaffiando, il suolo avrà un potenziale idrico maggiore e la pianta assorbirà acqua dall'esterno verso l'interno. Questo ingresso di acqua non è sufficiente, perché altrimenti solo l'apparato radicale diventerebbe turgido. L'acqua deve essere movimentata attraverso tessuti specifici, un adattamento evolutivo dalla colonizzazione delle terre emerse, assenti nei muschi.  Attraverso questi tessuti, l'acqua è trasportata anche grazie alle aperture stomatiche che permettono la sua fuoriuscita. Si forma una catena di molecole d'acqua che attraversa la pianta se gli stomi sono aperti. Se sono chiusi, la catena non si forma. L'apertura e chiusura stomatica è un meccanismo principale per il controllo degli eventi fisiologici della pianta.\footnote{The original text uses the word 'piantina' in the beginning, and then switches to 'pianta'. It's assumed they both refer to the plant in general, and 'pianta' was used for clarity in the processed text. The word 'idrica' was interpreted as a typo and replaced with 'radici' to maintain the logical consistency of the sentence regarding water potential. 'Aperture somatiche' was interpreted and corrected to 'aperture stomatiche', referring to stomata.  It was assumed that the speaker intended to say 'stomi' instead of 'cellule di guardia' in the context of water chain formation and plant physiological control, as stomata are the structures directly responsible for these processes. 'allontanare la pianta' was a bit unclear in the original context and reinterpreted as describing the movement of water through the plant up to the stomata.}
L'acqua risale nella pianta principalmente quando le aperture stomatiche sono aperte, permettendo la traspirazione. Il passaggio dell'acqua attraverso la pianta è fondamentale e richiede un grande apporto idrico. Se gli stomi sono chiusi, la pianta non traspira e ristagna. La traspirazione, ovvero il movimento dell'acqua dall'interno all'esterno attraverso gli stomi, genera una forza trainante che richiama acqua dal suolo verso la pianta. Questa forza permette alla pianta di mantenere il suo potenziale idrico interno leggermente inferiore rispetto a quello del suolo, favorendo l'assorbimento. Le cellule di guardia, fondamentali per la regolazione del flusso d'acqua e l'ingresso di $
ewline$ $\[0.2cm]$ $\footnote{La frase "Cioè, un trasporto... le piante devono soffrire, altrimenti non soffrono." risulta essere poco chiara e frammentata. Potrebbe essere interpretata come l'inizio di un ragionamento interrotto o una frase confusa. Ho riportato la frase così com'è, ma il significato preciso potrebbe essere andato perso.}
ewline$anidride carbonica ($\footnote{The initial phrase "Il peso, ma la musica è stata grave" seems out of context and possibly unrelated to the topic of the lecture. It might be a transcription error or a stray comment from the speaker.  I have included it in the processed text as is, but its meaning remains unclear.}
ewline$$\[0.2cm]$$\footnote{The spoken formula 'CO2' was interpreted as the chemical formula for carbon dioxide and written in LaTeX format as \$\\text\{CO\}\_2\$.}
ewline$ $CO_2$) necessaria per la fotosintesi, controllano l'apertura e la chiusura degli stomi. Passando da una forma chiusa a una forma aperta tramite il movimento dell'acqua intracellulare, queste cellule modificano la rima stomatica. La particolare struttura della parete cellulare, ispessita in alcune zone e meno in altre, con una disposizione radiale, permette l'allargamento o la chiusura delle cellule di guardia in risposta al movimento dell'acqua, garantendo così la regolazione stomatica.\footnote{The original text contained several disfluencies and repetitions, which were removed to improve clarity.  Phrases like "cioè", "o giù", "per parla- per rimane-" were omitted as they didn't add substantial information to the core message. The phrase "con la microfonia" seems out of context and was therefore excluded. The question "Siamo guardando la traduzione, ma che cosa succede?" was rephrased as "E qual è la proteina effettrice che agisce in seguito alla traduzione del segnale, causando l'apertura?" to better reflect the intended meaning within the context of the lecture.}
Come già detto, state effettuando la traduzione del segnale stimolo. Che cosa determina l'apertura dello stoma, secondo voi? Quello che è scritto nella slide? Sì. Lo stoma non è sempre aperto, non è sempre aperto come un poro. Si apre e si chiude. Si chiude in caso di aridità, ma non è sufficiente. Cioè, un trasporto... le piante devono soffrire, altrimenti non soffrono. Qual è lo stimolo più importante che innesca l'apertura dello stoma, secondo voi? Il peso?\footnote{Il testo originale presentava alcune difficoltà interpretative. In particolare, la frase "Chi si attiva dopo?" rimane in sospeso e non è chiaro a quale processo si riferisca. Inoltre, l'espressione "determina un'espressione con consumo di attivita di protoni" è poco chiara e potrebbe essere interpretata in modi diversi. Ho cercato di fornire la migliore interpretazione possibile, ma potrebbe essere necessaria una maggiore contestualizzazione per chiarire questi punti.}
Il peso, ma la musica è stata grave. La luce. Lo stimolo principale, il primo stimolo che in condizioni fisiologiche determina il benessere di una pianta e l'apertura stomatica è la luce. Quindi la luce, qui c'è scritto "blue light", perché effettivamente è la luce dell'alba, viene percepita. Gli stomi sono quasi sempre, nel 90%, 92% dei casi, poco aperti perché le condizioni ambientali per una pianta, se adattata bene, sono abbastanza estreme. Cioè, in estate, se lo stoma fosse sempre aperto, nelle ore centrali della giornata la pianta perderebbe molta acqua. Quindi tendenzialmente nelle ore centrali della giornata gli stomi sono socchiusi, perché da lì esce l'acqua. O si continua ad innaffiare queste piante, o le piante vengono attraversate da fiumi d'acqua.\footnote{The sentence 'Noi facciamo la traduzione dei nuovi con altri nuovi nello stesso modo' was omitted because its meaning in the context is unclear. It might refer to a specific process related to the discussed topic, but without further information, it's impossible to integrate it seamlessly into the summarized paragraph.}
Quindi, tendenzialmente, le piante sono più fotosinteticamente attive di giorno, ma la loro efficienza, in termini di quantità di luce assorbita e quantità di carbonio organicato, è maggiore all'inizio della giornata rispetto alle ore centrali. Le intense radiazioni solari delle ore centrali estive, infatti, rappresentano una condizione di stress per le piante. La luce fotosinteticamente attiva, oltre ad avere una specifica composizione spettrale, ha anche una determinata intensità. Nelle calde giornate estive, con temperature elevate e forte irraggiamento solare, le piante, non potendosi spostare, subiscono un forte stress. L'eccessiva quantità di luce causa la produzione di specie reattive dell'ossigeno, dannose per la pianta stessa. Per questo motivo, le piante attivano meccanismi di difesa contro l'eccesso di luce. Le condizioni migliori per la fotosintesi si verificano in presenza di luce non diretta, ad esempio in condizioni di cielo nuvoloso, quando la pianta non è esposta alla luce solare diretta.
La fotosintesi avviene normalmente, ma la quantità di massa prodotta è inferiore rispetto alla quantità di energia luminosa che colpisce la pianta. La luce blu, tipica dell'alba, induce l'apertura degli stomi nelle piante. Successivamente, gli stomi si richiudono per evitare un'eccessiva perdita d'acqua. Questa luce blu è percepita da un fotorecettore, una proteina in grado di captarla, che avvia la catena di trasduzione del segnale. Le catene di trasduzione del segnale sono simili nel mondo vegetale a livello molecolare. Non approfondiremo una specifica catena per l'apertura delle cellule di guardia, ma è importante capire che la luce blu dell'alba è lo stimolo che determina l'apertura degli stomi. Come viene tradotto questo stimolo? Qual è la prima azione che determina l'apertura stomatica? E qual è la proteina effettrice che agisce in seguito alla traduzione del segnale, causando l'apertura? La risposta è la pompa protonica.
La pompa protonica, situata sia nella membrana plasmatica che nel vacuolo, consuma ATP per estrudere protoni. Non è sempre attiva, ma si attiva in presenza di luce blu. Un recettore specifico percepisce la luce blu, innescando una catena di trasduzione del segnale che attiva la pompa protonica. L'attivazione della pompa protonica determina l'estrusione di protoni, che a loro volta attivano un processo a valle. Questo processo, attivato all'alba, consuma protoni ed è influenzato dall'attività della pompa protonica.\footnote{Il termine "over 2" in "diminuzione dell'ossido di over 2" è di difficile interpretazione. Potrebbe riferirsi a un composto chimico specifico, ma senza ulteriore contesto non è possibile determinarlo con precisione. Si consiglia di verificare la trascrizione originale per accertarsi della corretta interpretazione di questo termine.}
Non è un trasporto secondario, in questo caso è per quello che ha detto lui. Vedete che entra del potassio? Si attiva un canale voltaggio-dipendente del potassio. In questo caso, in seguito all'attivazione della pompa protonica, si attiva una proteina, un carrier che trasporta potassio, il canale voltaggio-dipendente. Estruendo protoni, cosa succede alla membrana plasmatica? La pompa protonica butta fuori i protoni. Sicuramente questo mi fa cambiare il pH.\footnote{Il termine "baccolo" è stato interpretato come "vacuolo", in quanto più coerente con il contesto. "Citotanini" è stato interpretato come "citoplasma", anche in questo caso per coerenza con il contesto. Si segnala inoltre una possibile confusione tra i termini "citosol" e "ciclosol" nel testo originale: è stato assunto che si riferissero entrambi al citosol.}
Di sicuro fa cambiare il pH, abbiamo già visto, poiché sono protoni, è facile. C'è un altro fattore che entra in gioco: elettrochimico. La differenza di concentrazione è chimica. Il fatto che quello che io sto spostando di qua e di là ha una carica, è elettrico. Quindi, spostando i protoni al di là della membrana plasmatica, grazie al consumo di ATP, grazie alla pompa protonica, cambio la concentrazione dei protoni e cambio il potenziale di membrana. E quindi quelle proteine che trasportano il potassio si aprono perché sentono un cambiamento elettrico. Ed è per questo che sono chiamati trasportatori elettrovoltaggio-dipendenti. Quindi si apre il canale e permette l'ingresso del potassio. Entra il potassio.\footnote{Il testo originale presentava numerose disfluenze e ripetizioni, rendendo difficile l'interpretazione di alcuni passaggi. In particolare, la frase "cioè il contatto, il movimento va bene, poi però fanno veramente avanti e indietro, c'è la parte inerente alle molecole organiche, tipo io che sono scritto qui su, quindi è un po' più complesso" risulta poco chiara e priva di un significato preciso nel contesto dell'apertura stomatica. Inoltre, il riferimento agli "squali" e agli "stili" o "alberi" sembra fuori luogo e non pertinente all'argomento trattato. Si è cercato di ricostruire il discorso del professore in modo coerente, ma alcune parti potrebbero non riflettere fedelmente il significato originale.}
Perché entra il potassio e non uno ione carico positivamente? Per una ragione di gradiente elettrico. Perché chimicamente, ma soprattutto elettricamente, stiamo buttando fuori cariche positive, quindi più facilmente manderà all'interno, chi ritornerà più facilmente all'interno delle cariche positive. Quindi, sì, entra il potassio, non entra il cloro, ma entra il potassio. E tendenzialmente non entra il sodio, questo è un po' un mestiere che va fatto senza il sodio, non entra il sodio perché è grosso. Quindi entra il potassio, poi lì dopo c'è un trasportatore secondario, che sarà un'antiportatore, protoni e zuccheri. Gli zuccheri sono specificatamente attivi, li abbiamo visti nella slide precedente, per quanto meno è contratta la concentrazione dei soluti. Quindi gli zuccheri che sono specificatamente attivi. Tutto questo flusso di ioni e zuccheri fa sì che il potenziale negativo della cellula di guardia diminuisca e cominci a richiamare acqua dalle cellule adiacenti.\footnote{The long sequence of "si" was interpreted as a confirmation of the previous question. The phrase "Vedi che le fa della pianta, va bene. Cosa state citando? Che cos'e che guardate?" was disregarded as it seemed to be out of context and didn't add information to the main reasoning line.  "Cioè" was interpreted as a discourse marker and therefore removed. The repeated words like "morte" were kept only once for clarity.}
A livello di ordine 6, la luce blu attiva la catena di trasduzione del segnale. Questa attiva la pompa protonica della membrana plasmatica, che causa l'estrusione di protoni e genera un'iperpolarizzazione della membrana. L'iperpolarizzazione inattiva il canale del potassio voltaggio-dipendente, che normalmente permette l'ingresso degli ioni $K^+$. Questi ioni $K^+$ partecipano alla formazione del complesso CD.S. Infine, si osserva una diminuzione dell'ossido di over 2.
Il protone entra nel citosol, trascinando con sé molecole osmoticamente attive, come gli zuccheri, aumentando la complessità della situazione. Questa complessità è dovuta all'elevato numero di molecole che entrano nel citosol. Queste molecole non rimangono nel citosol, in quanto ciò sarebbe incompatibile con la sua attività. Di conseguenza, si attiva la pompa protonica del vacuolo, situata nel tonoplasto. Questa pompa trasporta le molecole osmoticamente attive nel vacuolo, contribuendo al potenziale idrico senza interferire con l'omeostasi cellulare. In questo modo, il citosol mantiene il suo pH e la concentrazione delle sue componenti.  Il potenziale idrico viene mantenuto grazie al trasporto delle molecole attraverso la membrana plasmatica e il citoplasma fino al vacuolo. Questo processo mantiene anche l'osmolarità cellulare a livelli compatibili con la vita.\footnote{The initial question "Ah, che dovevo dire? Si, vi siete domandati?" was interpreted as a rhetorical question introducing the topic of cell wall formation and cell morphology, and thus omitted in the processed text.  The repetition of "Come, dove, quando la parete deve essere formata" was summarized to a single instance. The final question "Allora, perche si e fermato a fare quelle cellule epidermiche quadrate?" was interpreted as a concluding remark highlighting the peculiar morphology of the epidermal cells, and rephrased for clarity in the final sentence.}
L'attivazione dell'apertura stomatica, sistema che permette l'apertura dello stoma e garantisce il progredire della fotosintesi, la movimentazione dell'acqua e altre attività. Il movimento degli ioni, in particolare del potassio ($K^+$), avanti e indietro tra l'ambiente intracellulare ed extracellulare, crea uno squilibrio di concentrazione. Questo squilibrio, simile a quello mantenuto nei neuroni per la trasmissione degli impulsi nervosi, genera correnti elettriche utilizzate per vari processi biologici. Come esempio, la diminuzione del turgore cellulare causa la chiusura dello stoma.  Il potassio è l'ione principalmente coinvolto in questo meccanismo di regolazione.\footnote{Minor disfluencies and repetitions were removed. "biofuels" was interpreted as biocarburanti (biofuels) and "phosphofuels" was interpreted as combustibili fossili (fossil fuels) based on the context. "lima stomatica" was corrected to "rima stomatica" (stomatal pore).}
Quindi, quando lo stoma è ben aperto, a livello di catene ossidriliche, blocca l'ingresso dell'acqua. È un'umidità? Sì. State osservando il tessuto conduttore della pianta, il legno, formato da singole cellule morte chiamate cellule xilematiche. Il protoplasma all'interno non c'è più. Per svolgere la funzione di trasporto dell'acqua, è più comodo che la cellula sia priva di protoplasma, quindi morta.
Stiamo parlando della linfa grezza, non di quella elaborata per il trasporto degli zuccheri. È più comodo avere un tubo che permetta e resista alle forti pressioni, come durante la traspirazione dell'acqua. Quindi il tessuto migliore per la conduzione della sola linfa grezza, composta da acqua e sali minerali disciolti, è formato da cellule morte, di cui resta solo la parete. Questo tessuto, detto xilematico, fa parte dei tessuti vascolari ed è caratteristico delle piante terrestri. Le cellule impilate le une sulle altre, che lo costituiscono, prendono il nome di trachee o tracheidi. L'argomento dei prossimi 10-20 minuti sarà di nuovo la parete cellulare. Essa presenta caratteristiche differenti a seconda del tessuto, non a livello molecolare, ma in termini di spessore, robustezza e presenza o assenza del protoplasma all'interno. Le molecole che la compongono sono sempre le stesse, ma lo spessore può variare. La parete cellulare è una struttura esterna che conferisce protezione e robustezza. Nel tessuto xilematico, le trachee conferiscono rigidità alla pianta grazie alle pareti di cellulosa, che aiutano la pianta a mantenersi eretta e integra.\footnote{Some repetitions and unclear references ("questa", "questo") were resolved based on context. It's assumed that the speaker is pointing at visual aids while describing the components of the cell.}
Le cellule vegetali hanno tutte una parete cellulare che ne determina la forma. Come si vede nell'immagine B della slide, le cellule assumono forme diverse. La parete cellulare, con un controllo molto preciso, determina la morfologia della pianta e delle singole cellule, definendo come, dove e quando deve essere formata. Ad esempio, il tessuto epidermico nella figura B presenta una morfologia particolare con cellule quadrate.
No, non sarebbe più facile una pestrellatura di questo tipo? Cioè, una pestrellatura così mi garantisce un tessuto epidermico che tappa tutto. Può essere quadrata, con cellule allineate piuttosto che con cellule di quelle dimensioni, ed è sotto il controllo genetico, non è casuale. È un controllo genetico ben preciso che fa sì che la parete cellulare in quelle cellule non sia distribuita in modo omogeneo, ma presenti dei punti in cui è più spessa, meno spessa finché non si aggancia con la cellula contigua. Indipendentemente da tutto però, la parete cellulare, che siano questi irrobustimenti radiali delle cellule, che sia la parete cellulare della rima stomatica, che sia la parete cellulare dell'epidermide, ha sempre la stessa composizione: le molecole che la compongono sono sempre le stesse, non cambiano. E sono, più precisamente, quasi tutti zuccheri. Il grosso della parete cellulare è formato da zuccheri. Da qui nasce l'interesse biotecnologico. Se tutti quegli zuccheri che sono incastrati nella parete cellulare, che sono la biomassa, si riuscisse a liberare facilmente, se si riuscisse ad ottenere energia facilmente rompendo quegli zuccheri, forse potremmo anche abbandonare l'utilizzo dei biocarburanti, dei combustibili fossili. Perché lì, proprio in questa struttura, c'è una quantità di energia enorme.\footnote{The sentence "Possono non aver i cloroplasti perche hanno il" was cut off abruptly and it's unclear what the professor intended to say after "il". The unclear word was replaced by '[incomprensibile]' in the processed text.}
Non si opera spesso con il carbone, i fosfati che arrivano da questi, quindi c'è tantissima energia. Il problema è che non si riesce a liberare facilmente e in modo efficiente e non inquinante i singoli zuccheri presenti nella parete cellulare. Però ha un'applicazione enorme. Bene, questa è come viene classicamente disegnata una parete cellulare, da libro di testo. Queste sono due immagini, entrambe al microscopio elettronico a trasmissione, che mostrano una cellula circondata da parete nell'immagine in alto a destra.\footnote{The repetitions and unclear transitions were removed to create a more concise and readable paragraph.  The phrase "Qui e tutto vuoto, e bello pieno" was simplified as it presented contradictory information without clear context. It was interpreted as a clarification that although vacuoles appear empty, they are not. Possible loss of nuance in simplifying this phrase.}
All'interno della cellula, focalizzando l'attenzione su questa cellula circondata da una parete, si osserva la presenza di un nucleo, identificato come questo elemento centrale.\footnote{Some conversational fillers and repetitions were removed to improve clarity. For example, "e questo quello che vi dicevo prima", "forse incominciate a farvene una ragione", "cioè", "voi tre siete tre cellule", "se siete nati li, voi tre sarete vicini per sempre", "Cioè non vi potete spostare perche siete bloccati da una parete cellulare che non vi permette di spostare te li e viceversa", "ok?", "quell'immagine a sinistra, ok? Qui si vede una bella paretona cellulare che e tutta questa struttura qua, ok?" were omitted or rephrased.  The core message about plant cell immobility due to the cell wall and the concept of apoplast was preserved.}
Questo è il nucleolo. Tutta questa parte è il nucleo. Questi oggetti giganti sono i cloroplasti.\footnote{Minor disfluencies were smoothed out for clarity.}
Cosa sono? Non ho sentito. Ripeti. I mitocondri. Ricordatevi che le cellule vegetali hanno sempre i mitocondri. Possono non avere i cloroplasti perché [incomprensibile]. Quindi, quelli più piccoli, che sono decisamente più piccoli, sono i mitocondri. Questi.\footnote{The sentence "hanno disegnato i miei amici con delle caramelle che sono non sono altro che, di nuovo, zuccheri, sempre zuccheri sono, ma diversi o con una percentuale, si dice, della componente fibrillare, la componente che da la mistezza, molto maggiore" was challenging to interpret. It seems to refer to a visual representation of the cell wall components, where different sugars are depicted as candies. The phrase "la componente che da la mistezza" was interpreted as "la componente che conferisce robustezza", assuming "mistezza" was used informally to indicate strength or sturdiness. This interpretation maintains the overall coherence of the explanation regarding the composition and function of the cell wall.}
Osservando le immagini, si possono notare i vacuoli. Il termine "vacuolo" deriva dalle prime osservazioni microscopiche che mostravano uno spazio vuoto, da qui il nome. Nonostante nelle immagini appaiano come spazi vuoti, in realtà sono pieni. L'etimologia della parola, però, deriva da "vuoto".\footnote{The sentence "Piu si sviluppa la parete, piu si accresce la parete, piu scompare la lamella mediana, perche la lamella mediana a un certo punto schiaccia, schiacciata e scompare" was slightly reworded for clarity as "Più si sviluppa la parete, più si accresce, più la lamella mediana scompare, perché a un certo punto viene schiacciata e scompare." The introductory and conclusive remarks by the speaker ("Ok?", "Va bene", "saro gentile e mi fermo due minuti prima", "Gli zuccheri li guardiamooo la prossima settimana, ok? Buon weekend.") were omitted as they did not add information relevant to the core topic being discussed.}
Bene. Questa cellula, come vi dicevo prima, è più evidente nelle cellule vegetali. Dove nascono restano, perché sono sempre incastrate in una parete e non si possono spostare. Una pianta si sposta in direzione della luce, ma è un movimento che avviene attraverso la divisione cellulare. Quindi mi accresco, continuo a dividermi e mi allungo in quella direzione. Ma quella cellula che è nata lì, vicino ad altre cellule, resterà vicina per sempre. Non vi potete spostare perché siete bloccati da una parete cellulare. Siete incastonati all'interno di una parete cellulare e tutta la parete cellulare dell'intero organismo prende il nome di apoplasto. La parete cellulare ingrandita, nell'immagine a sinistra, si vede bene: è tutta questa struttura.
La parete cellulare si forma durante la mitosi, la divisione cellulare, e divide due cellule. Appena una nuova cellula figlia nasce, la parete cellulare è già formata, quindi la cellula non si sposta più ed i suoi vicini sono definitivi. La prima struttura che si forma nella parete cellulare, essendo la più esterna rispetto alla membrana plasmatica, è la lamella mediana. Questa struttura si forma subito durante la divisione cellulare e, siccome la parete è sintetizzata dal protoplasma della cellula vegetale, la lamella mediana viene spinta sempre più verso l'esterno. Quindi, la porzione più esterna della parete cellulare, presente in tutte le cellule, è la lamella mediana, che appare come una sorta di gel.
Gli zuccheri che compongono la lamella mediana sono la base per la formazione della parete primaria, la quale, a differenza della lamella mediana, non presenta la sostanza gelatinosa. La parete primaria è composta da zuccheri diversi o con una diversa percentuale della componente fibrillare, che conferisce robustezza. Nei tessuti conduttori delle piante vascolari, si forma anche una parete secondaria, ancora più esterna e robusta. La formazione della parete secondaria porta alla morte del protoplasta, lasciando un tubo vuoto con una parete spessa, utilizzata per costruire travi e traversine ferroviarie. Quindi, il legno delle panche su cui ci sediamo è essenzialmente la parete cellulare.
Quegli elementi vanno a costituire l'insieme delle pareti cellulari. Quando si contano gli anni in un frusto, si contano le pareti cellulari che si sono formate nel corso degli anni. E quando diventano così spesse, sono sostanzialmente date dalla presenza di una parete secondaria, caratteristica dei tessuti conduttori. Però, indipendentemente dalla presenza della parete secondaria, la cellula che ne è dotata ha avuto un momento in cui aveva la lamella mediana, la parete primaria e poi la parete secondaria. Più si sviluppa la parete, più si accresce, più la lamella mediana scompare, perché a un certo punto viene schiacciata e scompare.
I melanoplasti o altre tipologie di cellule contengono sempre mitocondri.
\end{spacing}
\end{document}
\documentclass[11pt, a4paper]{article}
\usepackage[utf8]{inputenc}
\usepackage[T1]{fontenc}
\usepackage{lmodern}
\usepackage{microtype}
\usepackage[margin=0.75in]{geometry}
\usepackage{parskip}
\usepackage{setspace}
\usepackage{amsmath}
\usepackage{amsfonts}
\usepackage[autostyle, english = american]{csquotes}
\MakeOuterQuote{"}
\setlength{\parindent}{1em}

\title{Lezione di Biologia Vegetale -- Parte 2}
\author{ClearNotes}
\date{\today}

\begin{document}
\maketitle
\begin{spacing}{1}
Questo esercizio, presentato nella slide, ha lo scopo di illustrare il potenziale idrico. Si consiglia di analizzare i valori numerici, poiché l'esercizio può essere replicato variando le condizioni. L'analisi verte sul potenziale idrico, e la slide, in alto a sinistra, riepiloga i parametri che lo influenzano. In particolare, il potenziale idrico tra due cellule adiacenti, o all'interno di una singola cellula, $\psi_W$ è definito come la somma di due componenti: $\psi_P$, il potenziale di pressione, influenzato dalla parete cellulare, e $\psi_S$, il potenziale osmotico, influenzato dalla concentrazione dei soluti.
In realtà, il $\psi_S$ non è altro che $-R \cdot T \cdot C_S$, dove $R$ rappresenta la costante dei gas, $T$ la temperatura, e $C_S$ la concentrazione dei soluti. Questa espressione è essenzialmente l'inverso del potenziale osmotico, data la presenza del segno negativo. La formula $-R \cdot T \cdot C_S$ è utile per determinare l'osmosi. Per calcolare $\psi_S$ di una soluzione, è necessario conoscere il valore di $R$, che è la costante dei gas, la temperatura $T$, e avere una stima della concentrazione dei soluti.
Consideriamo ora un contenitore vuoto in cui è stata inserita acqua pura. Questo schema non rappresenta una cellula, ma un recipiente. Qual è il potenziale idrico di un contenitore con acqua pura? Il valore di $\psi_W$ è zero. Questo perché, trattandosi di acqua pura, non sono presenti soluti né pareti. Pertanto, nel contenitore rappresentato in figura A, il potenziale idrico dell'acqua è pari a zero.
Si intende dissolvere una quantit\`a di saccarosio, precisamente una concentrazione di saccarosio pari a 0.1 molare, in un contenitore. Il saccarosio \`e lo zucchero comunemente utilizzato nel caff\`e. L'obiettivo \`e ottenere una concentrazione finale di saccarosio nella soluzione pari a 0.1 M. La domanda che si pone \`e: quale sar\`a il potenziale idrico di questa soluzione dopo l'aggiunta del saccarosio?
Precisamente, dove $C_S$ in questo contesto è 0.1 molare, come specificato. A $20$ gradi centigradi, la costante $R$ assume un valore di $-0.244$ MPa. Il termine $\psi_P$ non contribuisce in quanto non sono presenti pareti. Pertanto, il potenziale idrico di una vaschetta contenente una soluzione di saccarosio a concentrazione $0.1$ molare è di $-0.244$ MPa a $20$ gradi centigradi. Consideriamo ora l'inserimento di una cellula in questa soluzione. Si presentano due scenari possibili: nel primo caso, inseriamo una cellula turgida, ovvero ben idratata, mentre nel secondo caso inseriamo una cellula flaccida.
È evidente che queste due cellule reagiranno diversamente, anche se immerse nella stessa soluzione. Nella cellula turgida, dove la turgidità corrisponde a una concentrazione di 0,244, se la poniamo in questa soluzione, essa perderà acqua verso l'esterno, poiché la sua concentrazione di soluti è inferiore, diventando così flaccida. Al contrario, in una cellula vegetale tipica, la cellula tenderà ad assorbire acqua fino a quando il suo potenziale $\psi_P$, ovvero il potenziale di turgore, impedirà un ulteriore ingresso di acqua. Pertanto, diventerà turgida finché il suo potenziale idrico non eguaglierà il potenziale idrico della soluzione. Queste condizioni rappresentano le tipiche condizioni che si verificano in un ambiente fisiologico.
L'immagine in alto a sinistra mostra il comportamento tipico di una pianta quando si verifica un appassimento fisiologico. È fondamentale ricordare il periodo più basico come riferimento. Quando si interrompe l'apporto idrico, la pianta tende ad afflosciarsi e a perdere turgore, ovvero a disidratarsi. Quindi, si può affermare che lo stato di una pianta, in particolare al bordo fogliare, dipende dalla sua idratazione pregressa. Se la pianta non subisce stress eccessivi e viene gestita correttamente, è in grado di recuperare efficacemente, riacquistando turgore e ripristinando i flussi verticali in circa 24-48 ore.
Il processo descritto è possibile perché il potenziale idrico delle radici, confinate nel vaso, è inferiore a quello del suolo irrigato. Pertanto, quando si aggiunge acqua al suolo, quest'ultimo presenta un potenziale idrico maggiore rispetto alle radici, consentendo alla pianta di assorbire acqua dall'ambiente esterno. Questo assorbimento, tuttavia, non è sufficiente a rendere turgido solo l'apparato radicale, poiché l'acqua ristagnerebbe nelle radici. È necessario che l'acqua venga trasportata attraverso tessuti specifici, adattamenti evolutivi derivanti dalla colonizzazione delle terre emerse, assenti nei muschi. Tramite questi tessuti, l'acqua viene distribuita all'interno della pianta, anche grazie alla presenza di aperture stomatiche che ne permettono la fuoriuscita. Si crea così una sorta di catena di molecole d'acqua che attraversa l'intera pianta, allontanando l'acqua a livello delle cellule di guardia, se queste sono aperte.
Se le cellule di guardia sono in stato di chiusura, la catena d'acqua non si costituisce. L'apertura e la chiusura degli stomi rappresentano un meccanismo primario attraverso cui la pianta regola molteplici processi fisiologici ad essa correlati. La risalita dell'acqua si verifica prevalentemente quando gli stomi sono aperti, consentendo la fuoriuscita dell'acqua dalla pianta. Il flusso d'acqua attraverso la pianta è un fattore determinante del suo elevato fabbisogno idrico. In caso di chiusura degli stomi, si verifica un ristagno idrico all'interno della pianta. Pertanto, è necessario un processo di traspirazione, ovvero un movimento dell'acqua dall'interno verso l'esterno attraverso gli stomi.
L'acqua deve essere rilasciata dalla pianta. Una delle forze trainanti che spinge l'acqua dall'ambiente esterno, ovvero dal suolo, verso la pianta, è la perdita d'acqua da parte della pianta stessa. Questo permette di mantenere il potenziale idrico interno della pianta tendenzialmente inferiore a quello del suolo, altrimenti l'acqua non verrebbe assorbita. L'ultima diapositiva riguarda le cellule di guardia, che sono fondamentali per controllare la migrazione dell'acqua e l'ingresso dell'anidride carbonica, essenziali per il processo fotosintetico. Queste cellule possono passare da una conformazione chiusa a una aperta, regolando il movimento dell'acqua al loro interno. Questa è una relazione dell'acqua intracellulare, tra cellule di guardia, che permette il passaggio da una forma chiusa a una aperta. Questo è possibile grazie alla presenza di una rima stomatica con una parete cellulare particolarmente ispessita, una parete cellulare meno ispessita in un'altra regione, e una disposizione radiale della parete cellulare. Tale struttura consente l'allargamento delle cellule, garantendo l'apertura e la chiusura stomatica in funzione del movimento dell'acqua, trasmettendo così uno stimolo di partenza.
Come già detto, stiamo considerando la traduzione del segnale degli stimoli. Cosa determina l'apertura dello stoma, secondo voi? È ciò che è indicato nella slide? Sì. Lo stoma non è sempre aperto; uno stoma non è costantemente aperto, ma si apre e si chiude. Si chiude in risposta alla siccità, per così dire, ma questo non è sufficiente. In altre parole, le piante devono subire un certo stress per il trasporto.
Altrimenti non si verificherebbe sofferenza. Qual è, secondo voi, lo stimolo più rilevante che innesca l'apertura dello stomaco? Il peso. Il peso, sebbene anche la luce possa avere un ruolo.
Lo stimolo principale che, in condizioni fisiologiche, segnala il benessere di una pianta e determina l'apertura degli stomi è la luce. La percezione della luce, in particolare la "blue light" (luce blu), è cruciale poiché essa corrisponde alla luce dell'alba. Gli stomi, nella maggior parte dei casi (circa il 90-92\%), non sono completamente aperti. Questa parziale chiusura è dovuta al fatto che le condizioni ambientali per una pianta, anche se ben adattata, possono essere piuttosto estreme. Ad esempio, in estate, se gli stomi fossero costantemente aperti, la pianta perderebbe una quantità eccessiva di acqua durante le ore centrali della giornata. Pertanto, durante queste ore, gli stomi tendono a rimanere parzialmente chiusi.
L'acqua fuoriesce da lì, il che implica che o si continua ad innaffiare le piante, oppure queste vengono attraversate da flussi d'acqua. Di conseguenza, secondo voi, le piante sono più attive fotosinteticamente? Certamente di giorno, ma sono più efficienti in termini di quantità di luce assorbita e quantità di carbonio organico prodotto, per così dire. Questo avviene all'inizio o a metà della giornata? All'inizio della giornata.
Le posizioni centrali durante l'estate risultano eccessivamente intense per le piante, avvicinandosi a condizioni di stress. È fondamentale considerare che la luce fotosinteticamente attiva non è solamente un'irradiazione diretta, ma anche una quantità significativa. Pertanto, nelle giornate estive di luglio, caratterizzate da temperature elevate (circa 47 gradi) e da una luce intensa, una pianta che non può ripararsi sperimenta un notevole stress. L'eccessiva esposizione luminosa induce la produzione di specie reattive dell'ossigeno, le quali sono dannose per la pianta. Di conseguenza, la pianta attiva meccanismi di difesa per contrastare l'eccesso di luce. Le condizioni fotosinteticamente ottimali si verificano tipicamente in ambienti parzialmente ombreggiati, ad esempio in presenza di nuvole, evitando l'esposizione diretta e prolungata alla luce solare intensa, che può danneggiare le piante incapaci di ripararsi. In tali circostanze, si innescano diversi meccanismi di protezione contro l'eccessiva illuminazione.
La fotosintesi avviene, ma l'energia tradotta in biomassa è significativamente inferiore rispetto all'energia solare incidente sulla pianta. Tipicamente, la luce blu, presente all'alba, induce l'apertura degli stomi. Successivamente, gli stomi si chiudono, o tendono a richiudersi, per evitare un'eccessiva perdita d'acqua. Questa luce blu rappresenta uno stimolo esterno percepito da un fotorecettore, una proteina sensibile alla luce blu, che innesca la catena di trasduzione del segnale. Le catene di trasduzione del segnale, simili a quelle studiate in microbiologia, presentano analogie tra il mondo vegetale e altri organismi a livello molecolare. Non analizzeremo nel dettaglio la catena di trasduzione del segnale specifica per l'apertura delle cellule di guardia, ma è fondamentale comprendere che la luce blu, e quindi l'alba, determina l'apertura degli stomi. Secondo voi, in cosa si traduce questo stimolo? Qual è l'azione che determina l'apertura stomatica?
L'evento iniziale che innesca l'apertura stomatica è determinato da un processo di trasduzione del segnale. Dopo la trasduzione, l'effettore risultante agisce su una specifica proteina, la quale induce l'apertura. Questa proteina è la pompa protonica, localizzata sia nella membrana plasmatica che nel vacuolo, come precedentemente discusso.
Si consideri una pompa protonica situata nella membrana plasmatica. Un recettore sensibile alla luce blu, tramite una catena di trasduzione del segnale (che non analizzeremo nel dettaglio), trasferisce l'informazione alla pompa protonica. Questa pompa, per funzionare, idrolizza ATP, consumando energia. Di conseguenza, non è costantemente attiva, per evitare un eccessivo dispendio energetico per la pianta. Quando il recettore rileva la luce blu, si attiva la pompa protonica, che inizia ad estrudere protoni. L'estrusione di questi protoni determina l'attivazione di ulteriori processi, innescati dal segnale luminoso che attiva l'effettore e, conseguentemente, la pompa protonica.
Se si verifica l'attivazione di ALBA, si determina un'espressione che implica il consumo di attività protonica. Successivamente, quale processo si attiva? Qual è la fase successiva? Non si tratta di un trasporto secondario. In questo contesto, la spiegazione risiede in quanto precedentemente affermato. Si osserva l'ingresso di potassio? In tal caso, si attiva un canale del potassio voltaggio-dipendente.
Eseguiamo la traduzione di elementi nuovi con altri elementi nuovi in modo analogo. In questa situazione, a seguito dell'attivazione della pompa protonica, si attiva una proteina, precisamente un carrier che trasporta potassio. Questo carrier è un canale voltaggio-dipendente. Estrudendo protoni, cosa accade alla membrana plasmatica? La pompa protonica espelle i protoni, modificando di conseguenza il pH.
Certamente, l'aggiunta di protoni modifica il pH, come abbiamo già osservato; questo è un aspetto chimico. Tuttavia, interviene anche un fattore elettrochimico. La differenza di concentrazione è di natura chimica, ma il trasporto di ioni con carica positiva genera un effetto elettrico.
Pertanto, tramite la pompa protonica e il consumo di ATP, i protoni vengono trasferiti attraverso la membrana plasmatica, modificando sia la concentrazione protonica sia il potenziale di membrana. Di conseguenza, le proteine che trasportano il potassio si aprono in risposta alla variazione del campo elettrico, motivo per cui sono definite trasportatori "voltage-dipendenti". L'apertura del canale consente l'ingresso del potassio. L'ingresso del potassio, anziché di un altro ione positivo, è determinato dalla reazione al gradiente elettrochimico.
Il processo è chimico, ma anche elettrico, poiché vengono rilasciate cariche positive. Di conseguenza, maggiore è la quantità di cariche positive rilasciate, più facilmente altre cariche positive rientreranno. Quindi, il potassio entra, mentre il cloro non entra, ma il potassio sì. Il sodio, tendenzialmente, non entra, e questo è un compito che deve essere svolto senza il sodio, il quale non entra perché è di dimensioni maggiori. Pertanto, entra il potassio. Successivamente, interviene un trasportatore secondario, che sarà un simportatore, trasportando protoni e zuccheri. Il trasporto degli zuccheri è specificamente attivo, come visto precedentemente, soprattutto quando la concentrazione dei soluti è minore. Quindi, gli zuccheri sono trasportati in modo specificamente attivo.
L'accumulo di ioni e zuccheri determina una riduzione del potenziale negativo nella cellula di guardia. Questa diminuzione induce la cellula a richiamare acqua dalle cellule circostanti. Tale processo è influenzato dalla luce blu, agendo a livello dell'ordine 6.
La trasduzione del segnale comporta l'attivazione della pompa protonica sulla membrana plasmatica, con conseguente estrusione di protoni. Questo processo genera un'iperpolarizzazione della membrana, che a sua volta inattiva il canale del potassio. Indipendentemente, il canale del potassio permette l'ingresso di ioni $K^+$, i quali contribuiscono alla formazione del segnale CD.S. Si verifica, quindi, una diminuzione dell'ossido di over 2.
Di conseguenza, il protone, insieme ad altre molecole osmoticamente attive come gli zuccheri, viene trasportato all'interno. Questo contribuisce a rendere la situazione più complessa. Tuttavia, poiché la permanenza di queste sostanze nel citosol comprometterebbe la sua funzionalità, si attiva la pompa protonica del vacuolo (tonoplasto). In questo modo, le molecole osmoticamente attive vengono temporaneamente immagazzinate nel vacuolo, contribuendo al potenziale idrico senza alterare l'omeostasi cellulare.
Pertanto, l'omeostasi del citosol -- inclusi il pH e le concentrazioni di soluti -- è preservata poiché, temporaneamente, il potenziale idrico viene gestito tramite il trasporto di molecole attraverso la membrana plasmatica, nello specifico i citotannini, verso il vacuolo. In questo modo, si mantiene il potenziale idrico e si assicurano condizioni di osmolarità compatibili con la vitalità cellulare. Questo processo descrive l'attivazione dell'apertura stomatica, il meccanismo che consente l'apertura degli stomi e, di conseguenza, facilita la progressione della fotosintesi, il movimento dell'acqua e altre attività essenziali.
Esistono processi complessi, in particolare nel contatto e nel movimento molecolare, dove le molecole organiche interagiscono in modo dinamico. Il potassio, in particolare, si muove avanti e indietro, mantenendo uno squilibrio cruciale. Questo squilibrio è fondamentale in neurobiologia, ad esempio nello stimolo del dolore e nervoso, dove gli ioni, specialmente il potassio, si spostano. In quasi tutti i sistemi biologici, il potassio è mantenuto in uno stato di disequilibrio tra l'ambiente extracellulare e intracellulare, generando correnti elettriche. Questo meccanismo è presente anche negli squali. La diminuzione muscolare è legata a questo processo, e la firma rimane costante.
Finché lo stoma è aperto e il prodotto non sbatte contro le pareti, lo stilicidio non interviene. Quando lo stoma è completamente aperto, a livello di catene ossidriche, blocca l'ingresso dell'acqua. Si tratta di umidità.
State osservando l'endocardio, ovvero il tessuto conduttore della pianta. Il tessuto conduttore adulto e funzionante in una pianta, denominato legno, è costituito da singole cellule chiamate cellule xilematiche (con la X). Queste cellule sono morte, prive di protoplasma. Per svolgere la funzione di trasporto dell'acqua, è più efficiente avere il protoplasma, cioè la cellula viva all'interno, oppure no?
Ci concentriamo sulla linfa grezza, distinta dalla linfa adibita al trasporto degli zuccheri. Per il trasporto della linfa grezza, che deve resistere a forti pressioni, come quelle generate dalla traspirazione dell'acqua attraverso le pareti cellulari, il tessuto ottimale è quello costituito principalmente da cellule morte, dove persiste unicamente la parete cellulare. Questo tessuto, denominato tessuto xilematico, è un componente dei tessuti vascolari, specificamente uno dei due tipi di tessuti vascolari caratteristici delle piante terrestri. Le cellule impilate che lo compongono sono chiamate trachee o tracheidi, e la loro comparsa segna l'evoluzione di questo tessuto conduttore. Pertanto, nei prossimi minuti, l'attenzione sarà rivolta nuovamente alla parete cellulare. La parete cellulare, in base al tessuto specifico che si osserva, manifesta caratteristiche differenti in termini di spessore e robustezza, nonché per la presenza o assenza del protoplasma. Tuttavia, la composizione molecolare della parete cellulare rimane costante, variando unicamente lo spessore.
La parete cellulare è una struttura esterna alle cellule che fornisce un ulteriore livello di protezione e conferisce robustezza, come nel caso del tessuto xilematico. Le trachee, ad esempio, conferiscono rigidità alla pianta grazie alle pareti di cellulosa, che sono intrinsecamente rigide. Questo contribuisce a mantenere la pianta eretta e a preservarne l'integrità strutturale.
Pertanto, tutte le cellule vegetali possiedono una parete cellulare. Come illustrato in una diapositiva presentata precedentemente, che mostrava un contorno simile a questo, precisamente l'immagine B, ci si potrebbe chiedere come una cellula acquisisca tali forme. La ragione di queste forme risiede interamente nella parete cellulare. Esiste un controllo estremamente preciso che determina la morfologia del fondo e delle singole cellule, specificando come, dove e quando la parete cellulare deve essere formata.
Il tessuto epidermico mostrato nella figura B presenta una morfologia peculiare. La forma quadrata delle cellule epidermiche non è casuale, né è dovuta a una mancata ottimizzazione verso una forma più efficiente per la copertura, come una "pestrellatura". La presenza di cellule allineate di forma quadrata, anziché cellule di dimensioni diverse disposte in modo irregolare, è determinata geneticamente.
Un controllo genetico specifico determina che la parete cellulare non sia distribuita omogeneamente in determinate cellule, presentando zone di spessore variabile fino al punto di connessione con la cellula adiacente. Tuttavia, indipendentemente dalla sua localizzazione (irrobustimenti radiali delle cellule vascolari, parete cellulare della cellula stomatica, o parete cellulare dell'epidermide), la parete cellulare mantiene una composizione costante: le molecole che la costituiscono sono le stesse. Queste molecole sono prevalentemente zuccheri, che costituiscono la maggior parte della parete cellulare. Da qui deriva l'interesse biotecnologico: la possibilita' di liberare facilmente gli zuccheri intrappolati nella parete cellulare (biomassa) e di ottenere energia rompendo tali zuccheri potrebbe consentire di abbandonare l'uso di biocombustibili e combustibili fossili, data l'enorme quantita' di energia contenuta in questa struttura.
Spesso, le operazioni non coinvolgono il carbone o i combustibili fossili derivati da esso. Di conseguenza, si ha a disposizione una notevole quantità di energia. La difficoltà risiede nell'estrazione efficiente, non inquinante e agevole dei singoli zuccheri costituenti la parete cellulare. Tuttavia, il potenziale applicativo è vastissimo. Questa è una rappresentazione schematica, tipica dei libri di testo, di una parete cellulare. Le seguenti due immagini, entrambe ottenute tramite microscopia elettronica a trasmissione, illustrano cosa?
Nell'immagine situata in alto a destra, si chiede di identificare gli elementi visibili. Si osserva una cellula delimitata da una parete. All'interno di questa cellula, si richiede di specificare il contenuto. La cellula in questione, su cui si focalizza l'attenzione, è quella indicata.
Questa cellula è delimitata da una parete, che è questa. Avete menzionato il nucleo: quale elemento specifico è il nucleo?
Questo è il nucleo. Ok? Questo è il nucleolo. Ok? Quindi, tutto questo insieme costituisce il nucleo, la struttura più grande, che comprende l'intero nucleo.
Questi oggetti di grandi dimensioni che state osservando sono i cloroplasti, chiaro? Mi riferisco a questi elementi specifici, sui quali ho posizionato il cursore del mouse, che ora sto spostando.
Cosa sono? Ripeti. I mitocondri.
Si ricordi che le cellule vegetali possiedono invariabilmente mitocondri. Benché possano mancare i cloroplasti, a causa della presenza di melaninoplasti o altre strutture, i mitocondri sono sempre presenti. Pertanto, le strutture più piccole, significativamente più piccole, sono identificabili come mitocondri. Questi. Questi. Chiaro?
Cosa sono questi elementi visibili? Essi sono vacuoli.
Il termine "vacuolo" trae origine dall'idea di "vuoto", poiché le prime osservazioni microscopiche suggerivano la presenza di una struttura simile a un sacco privo di contenuto. Nonostante nelle immagini appaia spesso come uno spazio vuoto, l'etimologia della parola "vacuolo" rimanda proprio a questa percezione iniziale di cavità. Nelle cellule vegetali, i vacuoli permangono nella posizione in cui si formano, a causa della presenza costante della parete cellulare che ne impedisce lo spostamento: questo aspetto è particolarmente evidente e significativo.
Una cellula vegetale, come una pianta, si muove in direzione della luce attraverso un processo di divisione cellulare. La crescita avviene tramite divisioni cellulari successive, con conseguente allungamento nella direzione desiderata. Tuttavia, le cellule nate adiacenti rimangono vicine permanentemente. Ad esempio, considerate tre cellule nate vicine: queste tre cellule rimarranno adiacenti in quanto la parete cellulare impedisce il loro spostamento reciproco. Queste cellule sono vincolate all'interno della parete cellulare. L'insieme delle pareti cellulari dell'intero organismo è definito apoplasto. L'immagine ingrandita a sinistra mostra la parete cellulare.
Si osserva una parete cellulare, rappresentata dall'intera struttura, che separa due cellule distinte: la cellula due e la cellula uno. Le pareti cellulari si formano durante la mitosi, ovvero la divisione cellulare. La parete cellulare si costituisce contestualmente alla nascita di una nuova cellula figlia; pertanto, la cellula non nasce senza parete, né questa si forma successivamente. La parete cellulare è completa immediatamente dopo la divisione, impedendo ulteriori spostamenti della cellula.
La lamella mediana rappresenta la prima struttura a formarsi nella parete cellulare; di conseguenza, essendo la prima ad essere sintetizzata, si posiziona come lo strato più esterno rispetto alla membrana plasmatica. Questo accade perché è la cellula preesistente a generare la parete cellulare. Pertanto, lo strato primario della parete, ovvero quello più giovane, è denominato lamella mediana e si costituisce immediatamente durante la divisione cellulare. Man mano che la cellula secerne materiale, questa lamella viene progressivamente spinta verso l'esterno. La parete cellulare è interamente sintetizzata dalla cellula vegetale stessa; è il protoplasma che provvede alla sintesi della propria parete. La porzione più esterna della parete cellulare, presente in tutte le cellule vegetali, si presenta come una sorta di gel ed è identificata come lamella mediana, composta principalmente da zuccheri.
Al di sotto della lamella mediana, si sviluppa la parete primaria, la seconda parte della parete cellulare. In questa regione, la sostanza gelatinosa precedentemente menzionata è assente e la concentrazione di zuccheri è differente, presentando una frazione fibrillare significativamente maggiore, responsabile della resistenza meccanica. In alcuni tessuti, come quelli conduttori delle piante vascolari, si forma anche una parete secondaria. Questa parete aggiuntiva esercita una pressione verso l'esterno ed è estremamente robusta. La formazione della parete secondaria è tipicamente associata alla morte del protoplasto, che lascia un tubo vuoto con una parete spessa, utilizzata per la costruzione di travi e traversine ferroviarie. Questo materiale, derivante dalla parete cellulare, è stato tradizionalmente impiegato per tali scopi.
Quando vi sedete su una panca, vi state sedendo, in sostanza, sulla parete cellulare. Le panche in legno sono costituite da pareti cellulari assemblate. Quando contate gli anelli di accrescimento in un tronco, state contando le pareti cellulari che si sono formate nel corso degli anni. Lo spessore di queste pareti è dovuto principalmente alla presenza di una parete secondaria, caratteristica dei tessuti conduttori.
Indipendentemente dalla presenza della parete secondaria, la cellula dotata di tale parete ha precedentemente posseduto la lamella mediana e la parete primaria. Con lo sviluppo e l'accrescimento della parete, la lamella mediana tende a scomparire a causa della compressione esercitata su di essa.
Buon weekend.
\end{spacing}
\end{document}
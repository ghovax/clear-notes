\documentclass[11pt, a4paper]{article}
\usepackage[utf8]{inputenc}
\usepackage[T1]{fontenc}
\usepackage{lmodern}
\usepackage{microtype}
\usepackage[margin=0.75in]{geometry}
\usepackage{parskip}
\usepackage{setspace}
\usepackage{amsmath}
\usepackage[autostyle, english = american]{csquotes}
\MakeOuterQuote{"}
\usepackage{amsfonts}
\setlength{\parindent}{1em}

\title{Lezione di Biologia Vegetale -- Parte 1}
\author{ClearNotes}
\date{\today}

\begin{document}
\maketitle
\begin{spacing}{1}
Le aperture stomatiche sono prevalentemente localizzate sulla superficie inferiore della foglia, e non su quella superiore, al fine di minimizzare la perdita d'acqua. Questa localizzazione comporta una minore esposizione all'illuminazione e una riduzione degli scambi gassosi, data la minore aerazione. Le aperture stomatiche permettono l'ingresso dell'anidride carbonica e dell'ossigeno, e la fuoriuscita dell'acqua sotto forma di vapore. Queste aperture sono particolarmente numerose nella pagina inferiore della lamina fogliare e sono disperse tra le cellule dell'epidermide. Troviamo cellule specializzate, denominate cellule di guardia, che delimitano un poro che attraversa l'epidermide inferiore della foglia. Queste cellule di guardia hanno una forma simile a un fagiolo o a un rene.
I pori, o aperture stomatiche, sono regolabili in ampiezza. La loro apertura o chiusura dipende dalle condizioni ambientali e dallo stato di idratazione della pianta. In condizioni di scarsa idratazione, le cellule di guardia si sgonfiano e si chiudono, determinando la chiusura del poro stomatico. Questo meccanismo, ampiamente studiato, permette alla pianta di regolare gli scambi gassosi con l'ambiente esterno, specialmente in contesti di alte temperature e limitata disponibilità idrica.
In questa immagine, osserviamo una sezione trasversale di una foglia, non una vista a cubetto. Si può notare la faccia superiore della foglia con il suo strato di cutina e cere. Le cellule che costituiscono questa fitta rete sono strettamente adiacenti, senza spazi liberi, formando l'epidermide superiore della foglia. Queste cellule sono fortemente connesse tra loro, senza interspazi. Procedendo verticalmente attraverso l'epidermide fogliare, incontriamo un tessuto chiamato tessuto a palizzata, caratterizzato da cellule disposte verticalmente.
Queste cellule, disposte verticalmente, sono organizzate in una struttura simile a una foglia, formando un tessuto denominato tessuto a palizzata. Tale denominazione deriva dall'aspetto di una palizzata, con le cellule disposte perpendicolarmente all'epidermide fogliare. Il colore verde di queste cellule indica la presenza di clorofilla.
Le cellule rappresentate in verde appaiono tali poiché contengono clorofilla, indicando un'elevata concentrazione di cloroplasti. Ogni cellula possiede numerosi cloroplasti, la cui funzione primaria è la fotosintesi. Al di sotto del tessuto a palizzata si trova il tessuto spugnoso, comprendente la componente inferiore e l'area circostante. In questa regione, si osserva una minore densità cellulare, con cellule più distanziate e che perdono la caratteristica forma allungata e perpendicolare all'epidermide tipica del tessuto a palizzata. Questa seconda tipologia cellulare costituisce il tessuto spugnoso. Le cellule del tessuto spugnoso sono distanziate per facilitare la circolazione dei gas, come indicato dalle frecce, nel passaggio dalla superficie superiore a quella inferiore della foglia. Le aperture stomatiche sono prevalentemente localizzate sulla superficie inferiore della foglia.
Pertanto, attraverso le aperture stomatiche, i gas entrano ed escono, andando a riempire il tessuto spugnoso. Questo tessuto spugnoso, caratterizzato da spazi vuoti determinati dal distanziamento tra le cellule, si presenta come spazi riempiti di gas. L'ingresso di anidride carbonica è necessario per consentire la fotosintesi, così come l'uscita di ossigeno e, soprattutto, di vapore acqueo. L'emissione di vapore acqueo permette alla pianta di garantire il trasporto dell'acqua dall'apparato radicale alle foglie. Gli stomi, più numerosi sulla superficie inferiore della foglia, svolgono una duplice funzione: permettono la fuoriuscita dell'acqua in forma gassosa.
L'anidride carbonica entra sempre in forma gassosa. Il tessuto a palizzata e il tessuto spugnoso, considerati congiuntamente, costituiscono il mesofillo fogliare. Il mesofillo fogliare è invariabilmente composto da tessuto a palizzata e tessuto spugnoso. Le cellule che compongono il tessuto spugnoso, ovvero quelle di forma più arrotondata e distanziate tra loro, sono fotosintetiche, poiché sono verdi.
Anche se contengono cloroplasti e sono capaci di fotosintesi, la loro funzione principale non è la fotosintesi, sebbene ne siano capaci; il loro ruolo è conferire compattezza al tessuto, altrimenti sarebbe una semplice camera gassosa. Le cellule principalmente coinvolte nella fotosintesi sono invece quelle del tessuto a palizzata, che è il tessuto primariamente deputato alla produzione di nuova biomassa e alla fissazione dell'anidride carbonica. Questo è reso possibile dalla presenza del tessuto spugnoso, che facilita la movimentazione dei gas. Le cellule di guardia, osservate in immagini reali al microscopio ottico non colorate, sono quelle cellule, o meglio, quella coppia di cellule specializzate, che determinano la formazione del poro chiamato stoma. Le cellule di guardia, presenti soprattutto nella lamina inferiore della foglia, hanno una forma reniforme o a fagiolo. Come si può vedere, questa è una cellula e questa è la seconda. Sono due cellule a forma di C, adese l'una all'altra, che, in base al loro stato di idratazione, quindi semplicemente movimentando acqua, possono passare da uno stato turgido, ovvero gonfie di acqua, dove le cellule di guardia si distanziano aprendo lo stoma. Quindi, quando le cellule di guardia sono piene di acqua, lo stoma si apre, permettendo un efficiente scambio gassoso, che si traduce nell'ingresso di anidride carbonica e nella fuoriuscita di acqua.
Al contrario, quando le cellule di guardia perdono acqua, cedendola alle cellule adiacenti, si sgonfiano e lo stoma si chiude. La perdita di turgore causa il collasso delle cellule di guardia, determinando la chiusura del poro stomatico. In sintesi, le cellule di guardia, quando turgide, si aprono ampliando lo stoma; quando perdono acqua, si chiudono, regolando così l'ingresso dei gas. L'apertura e la chiusura stomatica non sono casuali, ma finemente regolate, poiché attraverso questo meccanismo la pianta controlla la perdita d'acqua, un fattore limitante cruciale.
Non è tanto la $\text{CO}_2$, quanto l'acqua a rappresentare il fattore limitante. In condizioni di siccità, anche lieve, le cellule di guardia si chiudono. Di conseguenza, la fotosintesi risulta compromessa, riducendosi a causa della minore quantità di anidride carbonica che riesce a entrare. Al contrario, in presenza di buona illuminazione e adeguate condizioni idriche, le cellule di guardia diventano turgide, lo stoma si apre e la pianta realizza il processo fotosintetico in modo efficiente, crescendo rapidamente. Circa il 95\% della perdita idrica di una pianta avviene attraverso gli stomi. Questo dato è rilevante perché indica che la pianta disperde la maggior parte della sua acqua attraverso queste aperture. La densità degli stomi è maggiore sulla faccia inferiore della lamina fogliare.
In questa immagine, si osserva una sezione trasversale di una foglia, sebbene l'anatomia fogliare possa variare tra diverse specie. Questa rappresentazione include sia un diagramma schematico sia immagini fotografiche reali. In particolare, in questa sezione trasversale, si evidenzia la presenza di uno strato di cuticola su entrambe le lamine, superiore e inferiore, tipicamente più spesso sulla lamina superiore. Si notano inoltre strati epidermici composti da cellule compatte e regolari, strettamente adiacenti senza spazi intercellulari, sia sulla superficie superiore che inferiore. Sulla lamina inferiore, si osservano aperture chiamate stomi, come indicato nell'immagine.
Gli stomi sono serie di aperture che consentono gli scambi gassosi, facilitando il movimento di $\text{CO}_2$ e acqua. L'immagine a destra mostra la superficie inferiore di una foglia, evidenziando gli stomi. Si noti la differenza anatomica tra le cellule di guardia e le cellule del tessuto epidermico: le cellule di guardia, con la loro forma a manubrio, rotondeggiante o a fagiolo, permettono l'apertura e la chiusura stomatica. Un breve video mostra il processo di chiusura degli stomi.
Si osservi che la rima stomatica si sta chiudendo gradualmente. Questo fenomeno si verifica tipicamente in condizioni di stress idrico, come la siccità, ma anche durante la notte. Di norma, gli stomi si chiudono durante la notte poiché la pianta non esegue processi fotosintetici e, di conseguenza, non è necessario mantenere gli stomi aperti, specialmente in ambienti in cui la notte è calda e si verificherebbe comunque l'evaporazione dell'acqua. L'acqua, infatti, non segue un andamento dipendente dalla luce, ma viene semplicemente evaporata o traspirata in base a un gradiente di concentrazione dell'acqua. Pertanto, durante la notte, in assenza di fotosintesi e senza la necessità di assorbire $\text{CO}_2$ dall'ambiente esterno, la pianta chiude le aperture stomatiche per prevenire una perdita eccessiva di acqua. Come fanno le cellule di guardia a realizzare questa apertura e chiusura? Analizzeremo i dettagli in seguito.
Quali caratteristiche deve possedere una cellula di guardia per aprirsi e chiudersi, come illustrato nel video precedente? Se fosse un vacuolo contrattile... No. L'acqua e la sua movimentazione sono rilevanti, ma non si tratta di un vacuolo contrattile. In base a questa immagine, che mostra la faccia inferiore dell'epidermide della pianta \textit{Arabidopsis thaliana}, quali sono le vostre ipotesi?
Pertanto, le cellule con struttura a incastro sono le cellule epidermiche di \textit{Arabidopsis thaliana}, le quali non presentano una forma rotonda; al contrario, \textit{Arabidopsis thaliana} esibisce questa morfologia a incastro, definibile come "a puzzle". Queste cellule si connettono in modo preciso, risultando completamente sigillate. Le strutture visibili sono chiaramente stomi, facilmente identificabili. Qual è il meccanismo di funzionamento degli stomi? L'apertura e la chiusura stomatica sono regolate dal movimento dell'acqua, ma è necessario un ulteriore fattore, oppure il movimento dell'acqua è sufficiente?
Non si tratta precisamente del ceto, ma di una caratteristica legata alla robustezza. Consideriamo la parete cellulare. Immaginiamo di gonfiare un palloncino, sia con acqua per un gavettone, sia con aria, o con elio. Se le pareti del palloncino offrono una resistenza uniforme, esso si espande in ogni direzione. Per consentire l'espansione in una sola direzione, le cellule di guardia si deformano, non si comportano in questo modo. In altre parole, la singola cellula non alterna uno stato flaccido ad uno turgido.
Si è affermato che un movimento direzionale spinge la cellula, il che implica l'esistenza di una resistenza. In altre parole, all'interno delle cellule, una regione specifica oppone meccanicamente una resistenza maggiore rispetto ad altre. Di conseguenza, quando l'acqua entra, la cellula non si espande in modo uniforme (isodiametrico) in tutte le direzioni. Piuttosto, si espande principalmente formando una protuberanza, poiché questo è il suo comportamento caratteristico: si flette anziché allargarsi. Pertanto, considerando che la parete cellulare è una caratteristica distintiva di tutte le cellule vegetali, si chiede dove sia più abbondante in queste cellule specifiche. In altre parole, quale zona offre la maggiore resistenza? Se una zona offre una resistenza maggiore, ci si aspetta che abbia una parete cellulare più spessa.
La parete cellulare interna delle cellule, che presentano una forma colonnare quando sono "sgonfie" (ovvero allungate per la perdita di acqua, come osservato nei movimenti precedenti), possiede una regione specifica denominata rima stomatica. Questa rima stomatica si caratterizza per avere una parete cellulare di spessore maggiore rispetto al resto della cellula, consentendo così il movimento di flessione quando le cellule acquisiscono acqua.
Non è sufficiente una rima stomatica più spessa per assicurare l'apertura e la chiusura dello stoma. Le cellule di guardia mostrano quasi sempre una distribuzione della parete cellulare, poiché sono tridimensionali, sebbene le osserviamo bidimensionalmente. Essendo cellule, possiedono una terza dimensione e presentano ispessimenti della parete cellulare disposti radialmente. Pertanto, oltre alla rima stomatica interna più spessa, vi è una disposizione radiale della parete più spessa per garantire la robustezza cellulare e la risposta morfologica all'ingresso dell'acqua. Ci sono termini che, a mio avviso, abbiamo già trattato, ma non ne sono del tutto sicura; comunque, sono riportati in questa slide.
È necessario apprendere la terminologia specifica, poiché è fondamentale per acquisire il linguaggio appropriato. Come già illustrato in precedenza, questo diagramma rappresenta schematicamente una cellula vegetale. Abbiamo precedentemente stabilito che le cellule vegetali sono univocamente caratterizzate dalla presenza di una parete cellulare, un aspetto già discusso in relazione ai batteri. Sebbene anche i batteri possiedano una parete cellulare, non sono gli unici organismi a presentarla. La distinzione tra la parete cellulare vegetale e quella batterica risiede nelle macromolecole costituenti, che sono differenti. Tuttavia, la funzione primaria rimane la stessa: conferire robustezza alla cellula.
In questo schema, la parete cellulare è rappresentata dalla struttura colorata in verde. Nella realtà, la parete cellulare corrisponde essenzialmente a questa struttura evidenziata in verde tramite fluorescenza. Nella sezione superiore, già precedentemente esaminata, la parete cellulare è delineata dal contorno esagonale osservabile in queste cellule. È importante sapere che in laboratorio è relativamente semplice prelevare cellule da una foglia, danneggiandola leggermente, ovvero tagliandola con carta vetrata fine (evitando carta vetrata grossa per non distruggere le cellule), tamponandola delicatamente. Successivamente, la foglia viene lasciata macerare in presenza di enzimi, tipicamente un mix di enzimi, che degradano i componenti molecolari della parete cellulare. Il risultato sono cellule vegetali private della parete cellulare, denominate protoplasti. Pertanto, un protoplasto è una cellula vegetale a cui è stata rimossa la parete cellulare.
I protoplasti non si trovano in natura; sono un prodotto di laboratorio. Questo perché un protoplasto è una cellula indifesa ed estremamente delicata. Tuttavia, può essere molto utile, ad esempio, per la produzione di piante transgeniche, in quanto la parete cellulare costituisce una barriera che impedisce l'esposizione del DNA all'esterno e fornisce robustezza, mantenendo le cellule adiacenti. L'eliminazione della parete cellulare rilascia quindi cellule vegetali prive di parete cellulare, denominate protoplasti, nel mezzo di coltura. Queste cellule, di forma rotondeggiante, sono visibili sul lato destro della slide e rappresentano tipicamente protoplasti.
Secondo voi che protoplasti vi da? Dove sono stati ottenuti? 
Si intende una palizzata realizzata fisicamente. Considerando l'immagine indicata dal laser, già precedentemente analizzata, si chiede: cosa rappresenta tale immagine? Si tratta di una sezione trasversale o longitudinale di cosa?
Consideriamo ora una sezione trasversale del tessuto a palizzata. Questa sezione trasversale del tessuto a palizzata, composto da elementi fisici, si presenta bidimensionale nella visualizzazione, ma in realtà è tridimensionale, estendendosi anche in profondità. Pertanto, quando lo sezioniamo trasversalmente, assume tipicamente questo aspetto.
Una caratteristica distintiva, che si deduce logicamente, è la ragione per cui non può trattarsi di un tessuto epidermico: l'elevata concentrazione di clorurassi. Un tessuto epidermico contiene clorurassi? No, i tessuti epidermici tipicamente non ne contengono, e soprattutto non superano una concentrazione di 20. Qual è la motivazione? Esatto.
Il tessuto epidermico, sia superiore che inferiore, funge da protezione per la pianta dall'ambiente esterno. La presenza di pigmenti colorati impedirebbe alla luce di penetrare nel tessuto parenchimatico sottostante, specializzato nella fotosintesi. Pertanto, il tessuto epidermico è tipicamente incolore e permette il libero passaggio della luce per consentire al tessuto parenchimatico sottostante di svolgere la fotosintesi. Esso rappresenta il rivestimento più esterno della pianta e l'immagine mostrata è una sezione trasversale del tessuto parenchimatico, ricco di cloroplasti necessari per la fotosintesi. Si può quindi dedurre che i protoplasti di tabacco osservati in basso a destra derivino principalmente dalle cellule del tessuto parenchimatico (mesofillo fogliare), che sono state sottoposte all'eliminazione della parete cellulare in laboratorio, data anche la maggiore abbondanza numerica di queste cellule.
I protoplasti sono cellule estremamente fragili, ridotte a cellule la cui unica protezione dall'ambiente esterno è la membrana plasmatica. Essi offrono vantaggi e svantaggi, e il loro utilizzo è confinato alle attività di laboratorio. In molti casi, la produzione di protoplasti è necessaria per facilitare l'assorbimento di materiale genetico esterno da parte di una cellula vegetale. Questo non è sempre obbligatorio, ma è essenziale per alcune manipolazioni genetiche specifiche. La parete cellulare è coinvolta negli scambi idrici, influenzando la relazione idrica della cellula. È quindi necessario definire cosa si intende per scambio idrico e quali leggi fisiche lo regolano.
La lettera $\psi$, spesso utilizzata per rappresentare il potenziale idrico nelle piante ($\psi_{W}$), ma anche in altri contesti, indica il potenziale dell'acqua. In termini semplificati, il potenziale idrico può essere interpretato come una misura della concentrazione dell'acqua. L'acqua si muove spontaneamente secondo leggi fisiche, sebbene questo movimento possa essere influenzato da trasportatori. In particolare, l'acqua si sposta sempre da una regione con potenziale idrico maggiore ad una con potenziale idrico minore. In altre parole, l'acqua fluisce da una zona ad alta concentrazione verso una a bassa concentrazione. Pertanto, se una cellula o un tessuto adiacente presenta un potenziale idrico inferiore, l'acqua si muoverà verso di esso, e viceversa.
In questo schema, l'acqua si sposta verso l'interno della cellula poiché il potenziale idrico intracellulare è superiore a quello extracellulare. Nella regione inferiore, il flusso idrico è diretto verso l'esterno, indicando che il potenziale idrico extracellulare supera quello intracellulare. È fondamentale notare che il movimento dell'acqua è determinato dal suo potenziale idrico, il quale non dipende unicamente dalla concentrazione dell'acqua stessa, ma è influenzato da molteplici fattori. Due fattori principali regolano lo scambio idrico tra cellule adiacenti.
Considerando, ad esempio, senza riferirci nuovamente al video, ma focalizzandoci qui, i due fattori precedentemente menzionati nella slide, ovvero $\psi_p$ e $\psi_S$, essi rappresentano i fattori che influenzano una variazione del potenziale idrico, ad esempio, nella cellula di guardia. Di conseguenza, quando il potenziale idrico $\psi_W$ di questa cellula di guardia aumenta eccessivamente, superando il potenziale esterno, l'acqua si sposta verso l'interno della cellula di guardia, in direzione delle cellule compagne. Questo potenziale idrico $\psi_W$ dipende da $\psi_p$ e $\psi_S$. Cosa rappresentano $\psi_p$ e $\psi_S$? Cosa determina tipicamente il movimento dell'acqua? $\psi_S$ è correlato ai soluti ed è influenzato dalla concentrazione dei soluti. Cosa rappresenta invece $\psi_p$?
Non trattandosi del foglio di guardia, cosa regola, a vostro parere, la movimentazione dell'acqua, oltre a quanto gi\`a osservato sulla struttura e l'importanza della parete, inclusi gli ossidipine? \`E` la parete cellulare a regolare tale processo. In questo contesto, la parete esercita una pressione che, diversamente da quanto accade nel protoplasto, impedisce alla cellula vegetale di esplodere. In termini semplici, la cellula non subisce lisi. La pressione esercitata \`e determinata dalla presenza della parete stessa; modificando la concentrazione dei soluti, la cellula di guardia pu\`o trovarsi in uno stato sgonfio.
Si osserva un progressivo sgonfiamento. Successivamente, con l'esposizione solare e l'inizio della fotosintesi, si verifica la necessità di assorbire acqua per rilasciare i prodotti della fotosintesi. La cellula di guardia può iniziare a trasportare soluti osmoticamente attivi. L'incremento della concentrazione di questi soluti all'interno della cellula di guardia facilita l'assorbimento di acqua dalle cellule adiacenti. Pertanto, la cellula di guardia attrae acqua dalle cellule contigue verso il suo interno, modificando la propria concentrazione di soluti. Questo processo di assorbimento di acqua continua indefinitamente? No, esiste un punto in cui la cellula di guardia raggiunge la turgidità e cessa di assorbire ulteriore acqua.
Si verifica una condizione in cui il potenziale idrico di pressione ($\psi_p$) eguaglia il potenziale osmotico ($\psi_S$). A questo punto, l'assunzione di acqua cessa, poiché si raggiunge l'equilibrio, un fenomeno derivante dalla presenza della parete cellulare. Nonostante una distribuzione non uniforme, la cellula si espande in modo appropriato, facilitando l'apertura stomatica, ma l'assunzione di acqua non prosegue indefinitamente fino alla rottura cellulare. L'assunzione di acqua continua fino a quando $\psi_S$ eguaglia $\psi_p$, rendendo il potenziale idrico totale ($\psi_W$) uguale tra l'ambiente esterno e interno. Questa condizione di equilibrio, in cui $\psi_W$ è uguale all'interno e all'esterno della cellula, non implica necessariamente concentrazioni di soluti identiche, poiché i contributi di $\psi_p$ e $\psi_S$ possono variare. Pertanto, quando l'acqua si sposta dall'esterno all'interno della cellula di guardia, causandone il rigonfiamento, $\psi_p$ interviene per arrestare l'ulteriore afflusso di acqua, prevenendo il turgore eccessivo. 
L'acqua nelle cellule vegetali si muove, analogamente a quanto avviene in tutte le cellule, seguendo il proprio potenziale idrico. Tuttavia, il potenziale idrico non è determinato casualmente; esso rappresenta una dimensione fisica la cui grandezza è stabilita dalla pianta stessa. Pertanto, è la cellula a decidere quando e come modificare il proprio potenziale idrico. Di conseguenza, il movimento dell'acqua non è casuale, ma le cellule regolano e modificano il loro $\psi_W$ in base a specifiche relazioni interne.
In questo schema, che rappresenta una singola cellula vegetale, la struttura esterna indicata come "parete primaria" o semplicemente "parete", corrisponde all'apoplasto, ovvero la parete cellulare, come discusso nelle lezioni precedenti.
Ogni cellula è costituita da una parete cellulare che delimita un protoplasto. L'insieme delle pareti cellulari di un organismo vegetale è definito apoplasto, mentre il continuum dei citoplasmi è denominato simplasto; si raccomanda di rivedere gli appunti precedenti a riguardo. Considerando una singola cellula, identificando la parete e la cellula stessa, quest'ultima corrispondente a ciò che precedentemente abbiamo definito protoplasto, come può tale cellula modificare il proprio potenziale idrico? In altre parole, attraverso quali meccanismi una cellula altera il suo potenziale idrico? A tal fine, è necessario integrare le informazioni fornite dalle immagini illustrative. I soluti sono liberamente permeabili alle membrane plasmatiche? La risposta è negativa.
La necessità di energia è stata già trattata. Questo argomento è fondamentale poiché si manifesta in tutti i sistemi biologici. Probabilmente esiste un importante meccanismo fotonico di trasporto, una pompa fotonica. Cosa genera questo trasporto attivo primario? Un gradiente. Il rettangolo rosso che indica lo spostamento di ATP e protoni rappresenta una pompa protonica, ovvero un trasporto attivo primario che consuma ATP per trasferire protoni attraverso la membrana plasmatica. Questo processo crea un gradiente protonico, uno squilibrio nella concentrazione di protoni, determinando se l'interno della cellula contenga più o meno protoni.
Infatti, il pH risulta più basico rispetto all'ambiente esterno. Di conseguenza, la pompa protonica trasferisce protoni dal citosol all'apoplasto, che rappresenta l'ambiente extracellulare. Considerando il trasferimento di questi protoni verso la struttura indicata dal "pallino blu", si chiede di identificare correttamente la natura di tale struttura. Qual è la denominazione specifica di questo "pallino blu"? Si tratta di una proteina di membrana, ma si richiede la sua identificazione precisa.
Le pompe protoniche, per definizione, consumano energia. Un trasporto passivo, invece, non consuma energia, poiché permette l'ingresso di sostanze. Un trasportatore, per mantenere i protoni all'esterno della cellula, utilizza ATP, consumando quindi energia.
Un trasportatore attivo secondario differisce da uno primario, il quale utilizza direttamente ATP. Le pompe protoniche, proteine transmembrana, consumano energia, nello specifico derivante dall'idrolisi dell'ATP, per trasportare protoni contro il loro gradiente di concentrazione, spostandoli da zone a bassa concentrazione protonica a zone ad alta concentrazione. Questo processo genera un $\Delta$pH attraverso la membrana plasmatica, mantenendo un pH di 7 nel citosol e di 5 nell'apoplasto. È fondamentale comprendere che questi valori di pH riflettono diverse concentrazioni di protoni nel citosol e nell'apoplasto; pertanto, è necessario determinare la differenza quantitativa di concentrazione protonica tra i due compartimenti.
Pertanto, impostando il sequenziatore con tutti i parametri relativi ai corsi del primo anno, si otterrà un quadro operativo. Il delta pH determina che i protoni presenti all'esterno possano rilasciare energia se movimentati secondo il loro gradiente di concentrazione. In altre parole, quando questi protoni ritornano nel citosol attraverso la proteina blu, che funge da trasportatore, sono in grado di fornire energia al sistema. Di fatto, il trasportatore blu movimenta un protone che fornisce energia al sistema per trasportare un'altra molecola solubile, rendendola permeabile attraverso la membrana plasmatica. Questo tipo di trasporto, così come rappresentato, è precisamente un trasportatore attivo secondario, poiché non idrolizza ATP ma dissipa un gradiente. Quindi, si tratta di un trasporto attivo secondario, e questo trasportatore possiede un'ulteriore caratteristica.
Si stanno trasportando due specie, quindi si sta effettuando un simporto. L'opposto del simporto è l'antiporto. Questo è un simporto perché due molecole, o più precisamente due ioni o particelle, si muovono nella stessa direzione. Pertanto, l'elemento indicato come "pallino blu" rappresenta in realtà una proteina transmembrana che media un trasporto attivo secondario di tipo simporto. È fondamentale acquisire familiarità con questi termini, poiché in futuro verrà utilizzato esclusivamente il termine "simporto", il che implica una serie di conseguenze.
Consideriamo ora la cellula di guardia, nello specifico in una cellula vegetale, con particolare attenzione al vacuolo. Il vacuolo è un organello intracellulare estremamente rilevante e caratteristico delle cellule vegetali. Cosa accade all'interno del vacuolo? Cosa viene mostrato in quel vacuolo? La sostanza marroncina presente nel vacuolo svolge la stessa funzione della membrana plasmatica.
Grazie a una pompa protonica, nello specifico due pompe protoniche presenti nei sistemi vegetali (sebbene differenti a livello proteico, ma con la medesima funzione), si verifica l'idrolisi di ATP per trasportare protoni all'interno del vacuolo. Questo gradiente protonico, ovvero una maggiore concentrazione di protoni all'interno rispetto all'esterno, viene sfruttato per il trasporto in antiporto di soluti, con i protoni che ritornano nel citosol. Questo meccanismo è necessario perché l'accumulo eccessivo di soluti osmoticamente attivi nel citosol porterebbe alla precipitazione delle proteine, condizione incompatibile con la vita cellulare. Al contrario, l'accumulo nel vacuolo è possibile grazie al suo basso contenuto proteico. La funzione del vacuolo, in particolare nelle cellule di guardia, è quella di abbassare il potenziale idrico, favorendo il richiamo di acqua dalle cellule circostanti.
In sintesi, il potenziale idrico cellulare diminuisce, principalmente a causa dell'accumulo di molecole osmoticamente attive al di fuori del citosol. L'accumulo di sali nel citosol è limitato per preservare la sua funzionalità. Al contrario, il vacuolo può accumulare un'alta concentrazione di molecole osmoticamente attive, poiché al suo interno non avviene la biosintesi proteica. Il vacuolo funge quindi da compartimento di stoccaggio, con una composizione variabile a seconda del tessuto. Ad esempio, nelle cellule di guardia, il vacuolo accumula molecole osmoticamente attive per favorire l'ingresso di acqua dalle cellule adiacenti, attraverso meccanismi regolati. Attualmente, l'obiettivo è comprendere i meccanismi che regolano il movimento dell'acqua nel vacuolo.
Grazie a questi elementi, nella cellula vegetale, ci stiamo formando un'idea chiara. Queste sono le definizioni che abbiamo esaminato, e questa rappresenta una differenza fondamentale tra le cellule vegetali e le cellule animali. Questo argomento, a mio parere, dovrebbe essere stato già trattato con Monica Formi, ma non ne sono più certo. Ci troviamo in laboratorio e stiamo esaminando la stessa slide, poiché si tratta di materiale didattico tipico del primo anno. Ma cosa sta succedendo qui?
Come precedentemente affermato, per l'apertura stomatica è necessaria la presenza di una parete cellulare. La presenza di tale parete modifica in modo significativo le relazioni idriche all'interno della cellula. Le differenze osservabili tra una cellula animale e una vegetale, o più precisamente, tra una cellula priva di parete e una dotata di parete, sono notevoli. In termini semplici, una cellula senza parete, come ad esempio un eritrocita, si comporta in modo caratteristico. Le cellule prive di parete, se collocate in un ambiente esterno ipo-osmotico o iperosmotico, reagiscono in modo specifico. In un ambiente ipo-osmotico, come indicato dalla presenza di acqua pura, si verifica un determinato fenomeno.
Pertanto, se preleviamo un globulo rosso e lo immergiamo in acqua bidistillata, si verifica che, data l'assenza di parete cellulare nel globulo rosso, l'acqua penetra al suo interno. Questo afflusso d'acqua è determinato dalla differenza di concentrazione dei soluti: l'acqua è pura, rendendo l'ambiente esterno ipo-osmotico rispetto alla cellula. Di conseguenza, l'acqua inizia a fluire all'interno della cellula. Le cellule prive di parete, inevitabilmente, si rompono a causa dell'ingresso continuo di acqua, un processo guidato esclusivamente da una forza fisica, ovvero il potenziale idrico, rappresentato da $\psi$. In assenza di un potenziale di pressione, $\psi_p$, la cellula assorbe acqua, si espande e infine si rompe. Questa rottura cellulare è comunemente definita lisi cellulare.
In una cellula vegetale, la presenza della parete cellulare impedisce la rottura della cellula. Si raggiunge un punto in cui il potenziale idrico ($\psi_W$) interno ed esterno si equivalgono, arrestando l'ulteriore ingresso di acqua. Questo equilibrio è determinato dal contributo del potenziale di pressione ($\psi_p$) esercitato dalla parete. La turgidità di una cellula vegetale varia in risposta al movimento dell'acqua, ma la cellula non si rompe grazie alla presenza di una parete rigida esterna che la contiene. Questa parete contribuisce al potenziale idrico ($\psi_W$), il quale dipende dal potenziale di pressione ($\psi_p$) e dal potenziale osmotico ($\psi_S$). In una cellula senza parete, il potenziale di pressione ($\psi_p$) è assente, causando un flusso netto di acqua che può portare alla rottura della cellula. In entrambi i casi, si può osservare un fenomeno chiamato "shrinkaggio" (plasmolisi), che indica una riduzione del volume cellulare.
Se una cellula viene posta in un ambiente in cui la concentrazione dei soluti (e quindi il potenziale idrico $\psi_W$) del mezzo esterno è iperosmotico rispetto al mezzo interno, la cellula perde acqua verso l'esterno. Sia le cellule animali che quelle vegetali subiscono un processo di appassimento, detto anche \textit{shrinking}, durante il quale si raggrinziscono e diminuiscono di dimensioni fino a morire. La morte cellulare è causata dall'eccessivo aumento della concentrazione salina interna, che crea una condizione incompatibile con la vita cellulare. Nelle cellule vegetali, la perdita di acqua verso l'esterno determina la formazione di strutture filamentose, a differenza di quanto avviene nelle cellule animali, dove il protoplasto si stacca. Cosa rappresentano le stelline nel disegno? Perché l'illustrazione mostra queste strutture filamentose invece di una cellula che si stacca e diventa rotonda all'interno?
La rappresentazione non è stata scelta per ragioni estetiche o di velocità; non vi è alcuna motivazione di questo tipo. L'obiettivo è chiarire il motivo di questa rappresentazione e assicurarne la comprensione. Qual è la natura di questi elementi?
Le strutture che connettono le cellule vegetali sono chiamate plasmodesmi e determinano la formazione del simplasto. Due cellule vegetali connesse da filamenti formano un continuo citoplasmatico denominato simplasto. Pertanto, tramite queste aperture, la cellula A e la cellula adiacente B sono in contatto diretto.
Quando due cellule vegetali perdono acqua rispetto all'ambiente esterno, la cellula interna non si rimpicciolisce formando delle strutture simili a "frutti", poiché la membrana plasmatica è in continuità con quella della cellula adiacente. Pertanto, la rappresentazione di una cellula vegetale in ambiente iperosmotico che perde acqua non mostra una riduzione di dimensioni, ma piuttosto una forma stellata che dovrebbe teoricamente aderire alla parete cellulare a causa della continuità con le cellule vicine. In condizioni di progressiva disidratazione, è possibile che la membrana plasmatica si rompa. La ragione di questa rappresentazione è che le cellule vegetali sono necessariamente adiacenti l'una all'altra, data la presenza costante della parete cellulare. Questa condizione influenza significativamente la fisiologia e il comportamento della pianta. Le cellule vegetali rimangono nel sito di origine per tutta la loro esistenza, poiché non possono spostarsi a causa della parete rigida che le vincola.
\end{spacing}
\end{document}
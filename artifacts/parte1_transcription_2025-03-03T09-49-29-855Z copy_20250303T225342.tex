\documentclass[11pt, a4paper]{article}
\usepackage[utf8]{inputenc}
\usepackage[T1]{fontenc}
\usepackage{lmodern}
\usepackage{microtype}
\usepackage[margin=0.75in]{geometry}
\usepackage{parskip}
\usepackage{setspace}
\usepackage{amsmath}
\usepackage{amsfonts}
\usepackage[autostyle, english = american]{csquotes}
\MakeOuterQuote{"}
\setlength{\parindent}{1em}

\title{}
\author{ClearNotes}
\date{\today}

\begin{document}
\maketitle
\begin{spacing}{1}
Le aperture stomatiche sono prevalentemente localizzate sulla superficie inferiore della foglia, e non su quella superiore, al fine di minimizzare la perdita d'acqua. Questa localizzazione comporta una minore esposizione all'illuminazione e una riduzione degli scambi gassosi, dovuta alla minore aerazione. Le aperture stomatiche permettono gli scambi gassosi, consentendo l'ingresso dell'anidride carbonica (e, in misura minore, dell'ossigeno) e la fuoriuscita dell'acqua sotto forma di vapore. Pertanto, queste aperture sono particolarmente numerose sulla superficie inferiore della lamina fogliare. Disperse tra le cellule dell'epidermide, si trovano cellule specializzate, denominate cellule di guardia. Queste cellule, dalla forma simile a fagioli o reni, delimitano un poro che attraversa l'epidermide, specificamente la superficie inferiore dell'epidermide fogliare.
I pori, o aperture stomatiche, sono regolabili in ampiezza. La loro apertura o chiusura dipende dalle condizioni ambientali e dallo stato di idratazione della pianta. In condizioni di scarsa idratazione, le cellule di guardia si sgonfiano e si chiudono, determinando la chiusura del poro stomatico. Questo meccanismo, ampiamente studiato, permette alla pianta di regolare gli scambi gassosi con l'ambiente esterno, specialmente in contesti di alte temperature e limitata disponibilità idrica.
In questa immagine, osserviamo una sezione trasversale di una foglia, non una vista a cubetto. Si può notare la faccia superiore della foglia con il suo strato di cutina e cere. Le cellule che costituiscono questa fitta rete sono strettamente adiacenti, senza spazi liberi, formando l'epidermide superiore della foglia. Queste cellule sono estremamente connesse tra loro. Spostandoci verticalmente attraverso gli strati dell'epidermide fogliare, incontriamo un tessuto chiamato tessuto a palizzata. Si noti come queste cellule siano disposte verticalmente.
Queste cellule, disposte verticalmente, sono organizzate in una struttura simile a una foglia e costituiscono un tessuto denominato tessuto a palizzata. Tale denominazione deriva dall'aspetto di una palizzata, dovuto alla disposizione perpendicolare delle cellule rispetto all'epidermide fogliare. Il colore verde di queste cellule indica la presenza di clorofilla.
Le cellule rappresentate in verde appaiono tali poiché contengono clorofilla, indicando un'alta concentrazione di cloroplasti. Ogni cellula possiede numerosi cloroplasti, la cui funzione primaria è la fotosintesi. Al di sotto del tessuto a palizzata si trova il tessuto spugnoso, comprendente la componente inferiore e l'asca. In questa regione, si osserva una minore densità cellulare, con cellule più distanziate e che perdono la forma allungata e perpendicolare tipica del tessuto a palizzata. Questa seconda tipologia cellulare costituisce il tessuto spugnoso. Le cellule del tessuto spugnoso sono distanziate per facilitare la circolazione dei gas, come indicato dalle frecce, muovendosi dalla superficie superiore a quella inferiore della foglia. Le aperture stomatiche sono più numerose e concentrate principalmente nella superficie inferiore della foglia.
Pertanto, attraverso le aperture stomatiche, i gas entrano ed escono, andando a riempire il tessuto spugnoso. Questo tessuto spugnoso, caratterizzato da spazi vuoti dovuti al distanziamento tra le cellule, e riempito di gas. L'ingresso di anidride carbonica e necessario per permettere la fotosintesi, cosi come l'uscita di ossigeno e, soprattutto, di vapore acqueo. L'emissione di vapore acqueo garantisce alla pianta la movimentazione dell'acqua dall'apparato radicale verso le foglie. Gli stomi, piu abbondanti sulla faccia inferiore della foglia, svolgono una duplice funzione: la fuoriuscita dell'acqua in forma di gas.
L'anidride carbonica entra sempre in forma gassosa.
\end{spacing}
\end{document}
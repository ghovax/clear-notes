\documentclass[11pt, a4paper]{article}
\usepackage[utf8]{inputenc}
\usepackage[T1]{fontenc}
\usepackage{lmodern}
\usepackage{microtype}
\usepackage[margin=0.75in]{geometry}
\usepackage{parskip}
\usepackage{setspace}
\usepackage{amsmath}
\usepackage{amsfonts}
\usepackage{xcolor}
\usepackage[autostyle, english = american]{csquotes}
\MakeOuterQuote{"}
\setlength{\parindent}{1em}

\title{Transcription}
\author{ClearNotes}
\date{\today}

\begin{document}
\maketitle
\begin{spacing}{1.15}
Questo esercizio, riportato nella slide, può essere svolto applicando diverse condizioni. Esso riguarda il potenziale idrico ($\psi_w$) in due cellule contigue o in una singola cellula. Il potenziale idrico è determinato da due parametri: il potenziale di pressione ($\psi_P$), dovuto alla presenza della parete cellulare, e il potenziale osmotico ($\psi_S$), dovuto alla presenza dei soluti. Il potenziale osmotico si calcola come $\psi_S = -RT \cdot C_S$, dove R è la costante dei gas, T è la temperatura e $C_S$ è la concentrazione dei soluti.  $\psi_S$ è l'inverso del potenziale isomorfico, a causa della presenza del segno meno. La formula $-RT \cdot C_S$ descrive efficacemente lo smesi.\footnote{Il termine "smesi" utilizzato nel testo originale non ha un chiaro significato scientifico in questo contesto. Potrebbe essere un errore di trascrizione o un termine gergale. Si consiglia di verificare il significato con il docente.}
Per determinare il potenziale idrico ($\psi_s$) di una soluzione, è necessario conoscere il valore della costante dei gas ($R$), della temperatura ($T$) e la concentrazione dei soluti. Considerando un contenitore con acqua pura, il suo potenziale idrico ($\psi_w$) è zero perché non ci sono soluti né pressioni esercitate dalle pareti del contenitore. Se si aggiunge del saccarosio (come una bustina di zucchero) all'acqua, si modifica il potenziale idrico.\footnote{The phrase "Come determinarsi la la si di s di una soluzione" was interpreted as "Per determinare il potenziale idrico di una soluzione". The phrase "un valore di meno, altrimenti non lo voglio sapere" was omitted because its meaning in the context is unclear. The phrase "Questa era un po' meno di una bustina di zucchero" was interpreted as an approximation of the amount of sugar added, and simplified as "aggiungere del saccarosio".}
Il saccarosio è lo zucchero che si usa normalmente nel caffè. Sciogliamo una quantità di saccarosio pari a 0,1 molare, ottenendo una soluzione con concentrazione finale di 0,1 M. Il potenziale idrico di questa soluzione si calcola con la formula $
ewline \[\footnote{Il testo originale conteneva alcune ripetizioni e frasi frammentate. Inoltre, la frase "Comunque non dimenticate il periodo piu basilico come riferimento" non è stata inclusa perché priva di senso nel contesto. Si consiglia di verificare il significato di questa frase con chi ha trascritto il testo.}
\psi = \psi_S + \psi_P = -R \footnote{Minor edits were made for clarity and flow. "Aperture somatiche" was interpreted as "aperture stomatiche" (stomata) given the context of water movement and guard cells. Please verify if this interpretation is correct.}
\cdot T 
\cdot C_S + \psi_P
\] \footnote{The initial part of the audio "Il peso, ma la musica e stata grave. La luce. La luce, ok." was discarded as it seems to be unrelated to the main topic of the lesson and might be a mis-transcription.  It was difficult to understand the exact meaning of the phrase "o se no passano". It's been interpreted in the context of water flow as needing rivers of water to survive if stomata were constantly open.}
$\newline$ dove $C_S$ è la concentrazione di saccarosio (0,1 M), $T$ è la temperatura (20°C) e $R$ è la costante dei gas. Il termine $\psi_P$ è nullo perché non ci sono pareti. Quindi, il potenziale idrico è pari a $-0,244$ MPa. In questa soluzione, inseriamo due tipi di cellule: una cellula turgida e una cellula flaccida.\footnote{The original text was quite clear and didn't present major interpretation issues. Minor disfluencies were smoothed out for better readability.}
Due cellule poste nella stessa soluzione si comporteranno in modo differente. La cellula turgida, con una concentrazione di 0,244, posta in una soluzione, perderà acqua verso l'esterno a causa della minore concentrazione di soluti, diventando flaccida. Al contrario, una cellula in condizioni tipiche vegetali, assumerà acqua fino a quando il potenziale di turgore ($\footnote{Il testo originale presentava alcune ripetizioni e frasi incomplete. Ho cercato di ricostruire il discorso del professore in modo chiaro e coerente, mantenendo il linguaggio tecnico e il filo del ragionamento. In particolare, la frase "Le catene di trasduzione del segnale, che state vedendo con la microfonia, non c'e una differenza enorme per parla- per rimane- il mondo vegetale da questo punto di vista molecolare, non la faremo una catena di trasduzione del segnale per l'apertura delle cellule di guardia, ma quello che mi interessa che vi sia chiaro che appunto la luce blu e lo stimolo" è stata reinterpretata come una spiegazione generale sul ruolo delle catene di trasduzione del segnale nel mondo vegetale, con un focus sul caso specifico dell'apertura stomatica indotta dalla luce blu.}
ewline
\footnote{The phrase "Noi facciamo la traduzione dei nuovi con altri nuovi nello stesso modo" is unclear in its meaning. It's been reported verbatim, but further clarification might be needed.}
\psi_p$) non impedirà un ulteriore ingresso, raggiungendo la turgidità quando il suo potenziale idrico eguaglierà quello della soluzione. Queste condizioni sono tipiche di un ambiente fisiologico. Un esempio del comportamento di una pianta in apnea fisiologica è l'afflosciamento e la perdita di turgore a causa della perdita d'acqua. Quindi, la vita di una pianta è in parte determinata dal suo contenuto d'acqua.\footnote{The initial part "Di sicuro fa cambiare il pH, abbiamo gia visto, poi son protoni, e facile. C'e un altro fattore che entra in gioco. Elettrochimico. Elettrochimico. La differenza di concentrazione e chimica." was omitted in the final version because it was considered an introductory remark, not directly contributing to the core explanation of the electrochemical mechanism. It was deemed redundant since the subsequent sentences already clarify the involvement of protons and the electrochemical nature of the process. The repetition of "Elettrochimico" was also considered a disfluency.}
Se la pianta non viene maneggiata eccessivamente, ma semplicemente annaffiata, è in grado di recuperare bene in 24-48 ore, tornando in una condizione di turgidità, con flussi verticali ben turgidi. Questo è possibile perché il potenziale idrico delle radici nel vaso è minore rispetto a quello del suolo annaffiato. Quindi, aggiungendo acqua al suolo, quest'ultimo avrà un potenziale idrico maggiore rispetto alle radici e la pianta assorbirà acqua dall'esterno verso l'interno.  L'ingresso di acqua dalle radici non è sufficiente, altrimenti solo l'apparato radicale diventerebbe turgido. L'acqua deve essere movimentata attraverso tessuti specifici, un adattamento evolutivo dalla colonizzazione delle terre emerse, assente nei muschi. Questi tessuti trasportano l'acqua all'interno della pianta, anche grazie alle aperture stomatiche che permettono la fuoriuscita di acqua. Si forma una catena di molecole d'acqua che attraversa la pianta, ma solo se le cellule di guardia degli stomi sono aperte.  L'apertura e chiusura stomatica è un meccanismo principale con cui la pianta controlla numerosi eventi fisiologici.
L'acqua risale nella pianta principalmente quando le aperture stomatiche sono aperte, permettendo la traspirazione. Il passaggio dell'acqua attraverso la pianta richiede un grande volume d'acqua; se gli stomi sono chiusi, la pianta non traspira e ristagna. La traspirazione, ovvero il movimento dell'acqua dall'interno all'esterno attraverso gli stomi, genera una forza trainante che richiama acqua dal suolo verso la pianta. Questa perdita d'acqua permette alla pianta di mantenere il suo potenziale idrico interno leggermente inferiore a quello del suolo, condizione necessaria per l'assorbimento. Le cellule di guardia, fondamentali per la regolazione del flusso d'acqua e l'ingresso di $\footnote{Il significato di "CD.S." non è chiaro e potrebbe richiedere ulteriore contesto. L'espressione "diminuzione dell'ossido di over 2" è stata interpretata come una diminuzione di superossido (\$O\_2\^−\$), ma potrebbe riferirsi ad altre specie chimiche. Si consiglia di verificare il significato di questi termini nel contesto della lezione.}
ewline$ $\[0.2cm]$anidride carbonica ($\footnote{Il termine "baccolo" utilizzato nel testo originale è stato interpretato come "vacuolo", in quanto più coerente con il contesto biologico della descrizione. Inoltre, l'espressione "membrana cosmatica" è stata interpretata come "membrana plasmatica".}
ewline$$\[0.2cm]$CO$_2$$\footnote{Alcune frasi nel testo originale erano frammentate e difficili da contestualizzare, rendendo l'interpretazione complessa. In particolare, la parte finale relativa al suono, al prodotto che sbatte contro le pareti e all'intervento dello stimolo risulta poco chiara nel suo significato complessivo. Potrebbe riferirsi ad un esempio specifico non menzionato in modo completo nella trascrizione. Inoltre, l'espressione 'si viette' non è chiara e potrebbe essere una trascrizione errata di un termine specifico.}
ewline$ $\[0.2cm]$) necessaria per la fotosintesi, controllano l'apertura e la chiusura degli stomi.  Passando da una forma chiusa ad una aperta tramite il movimento dell'acqua intracellulare, queste cellule sfruttano la differenza di spessore delle pareti cellulari nella rima stomatica: una parete ispessita e una meno ispessita, con una disposizione radiale che ne facilita l'allargamento e la chiusura in risposta al movimento dell'acqua, permettendo così la trasmissione di uno stimolo iniziale.\footnote{The initial part of the transcription "Vedi che le fa della pianta, va bene. Cosa state citando? Che cos'e che guardate? Stai guardando l'endocardi, cioe state guardando" seems to be unrelated to the main reasoning and appears to be a dialogue between the speaker and the audience. It was omitted in the processed text as it doesn't add information to the core explanation. The excessive repetition of "si" was also removed. It's unclear what "a livello di catene ossidriliche" refers to in the context of stomata closing to prevent water entry. Further context might be needed to clarify this point.}
Come già detto, state effettuando la traduzione del segnale stimolo e della valenza. Cos'è che determina l'apertura dello stoma secondo voi? Quello che è scritto nella slide? Sì. Lo stoma non è sempre aperto, non è sempre aperto come un poro. Si apre e si chiude. Si chiude in risposta alla siccità, ma non è sufficiente. Cioè, un trasporto... le piante devono soffrire, altrimenti non soffrono. Qual è lo stimolo più importante che innesca l'apertura dello stoma secondo voi? Il peso?\footnote{The sentence "Mi sono persa, volevo dire una cosa" was omitted since it doesn't add information to the topic discussed.}
Lo stimolo principale, il primo stimolo che in condizioni fisiologiche determina il benessere di una pianta e, di conseguenza, l'apertura stomatica è la luce. Più precisamente, la luce blu, la luce dell'alba. Gli stomi sono quasi sempre poco aperti, nel 90-92% dei casi, perché le condizioni ambientali per una pianta, seppur adattata, sono abbastanza estreme. Ad esempio, in estate, se gli stomi fossero sempre aperti nelle ore centrali della giornata, la pianta perderebbe una grande quantità di acqua. Quindi, tendenzialmente nelle ore centrali della giornata gli stomi sono socchiusi per limitare la perdita d'acqua. In caso contrario, le piante avrebbero bisogno di essere continuamente innaffiate o attraversate da fiumi d'acqua.\footnote{The initial question "Ah, che dovevo dire?" and the repetition "Come, dove, quando la parete deve essere formata." were omitted as they don't add information to the core message. The unclear transition "Allora, perche si e fermato a fare quelle cellule epidermiche quadrate?" was rephrased as a more direct question about the formation of the squared cells.}
Quindi, le piante sono più fotosinteticamente attive durante il giorno, ma all'inizio della giornata, piuttosto che nelle ore centrali. Le intense condizioni di luce delle giornate estive, soprattutto con temperature elevate come 47 gradi, rappresentano una situazione di stress per le piante. La luce fotosinteticamente attiva, pur essendo essenziale, può essere dannosa in quantità eccessive. Nelle giornate estive molto calde, le piante subiscono danni a causa della sovraesposizione alla luce, che porta alla produzione di specie reattive dell'ossigeno. Per difendersi, le piante attivano meccanismi di protezione contro l'eccesso di luce. Le condizioni migliori per la fotosintesi si verificano in presenza di luce non diretta, ad esempio quando il cielo è leggermente nuvoloso, evitando così l'esposizione diretta al sole.\footnote{It's unclear what "pestrellatura" refers to in this context. It could be a specific technical term related to the topic being discussed. Additionally, "rima stomatica" was corrected to "rima stomatica", as it seems to be a more plausible term in the context of plant biology. "phosphofuels" was interpreted and corrected to 'combustibili fossili'}
La fotosintesi avviene normalmente, ma la quantità di energia che si traduce in massa è molto inferiore rispetto alla quantità di energia che colpisce la pianta. La luce che induce l'apertura stomatica è principalmente la luce blu dell'alba. Quindi, gli stomi si aprono all'alba e tendono a chiudersi successivamente per evitare un'eccessiva perdita d'acqua. La luce blu è lo stimolo esterno percepito da un fotorecettore, una proteina in grado di captarla, che avvia la catena di trasduzione del segnale. Le catene di trasduzione del segnale nel mondo vegetale, in questo caso per l'apertura delle cellule di guardia, sono simili a quelle in altri organismi. La luce blu, quindi l'alba, è lo stimolo che determina l'apertura degli stomi. Ma qual è la prima azione che porta all'apertura stomatica? E, dopo la trasduzione del segnale, su quale effettore agisce? Qual è la proteina che determina l'apertura? La pompa protonica.
La pompa protonica, situata sia sulla membrana plasmatica che sul vacuolo, consuma ATP per estrudere protoni. Non è sempre attiva, ma si attiva in presenza di luce blu. Un recettore specifico percepisce la luce blu, innescando una catena di trasduzione del segnale che attiva la pompa protonica. L'attivazione della pompa protonica determina l'estrusione di protoni, che a loro volta attivano un processo a valle. L'esempio dell'alba è stato menzionato per illustrare una situazione in cui l'attivazione della pompa protonica, con il conseguente consumo di ATP e l'estrusione di protoni, è innescata dalla luce blu.
Non è un trasporto secondario, in questo caso è per quello che ha detto lui. Vedete che entra del potassio? Si attiva un canale voltaggio-dipendente del potassio. Noi facciamo la traduzione dei nuovi con altri nuovi nello stesso modo. In seguito all'attivazione della pompa protonica, si attiva una proteina, un carrier che trasporta potassio, il canale voltaggio-dipendente. Estruendo protoni, cosa succede alla membrana plasmatica? La pompa protonica butta fuori i protoni. Sicuramente questo mi fa cambiare il pH.
Spostando i protoni al di là della membrana plasmatica, grazie al consumo di ATP e alla pompa protonica, si cambia la concentrazione dei protoni e il potenziale di membrana. Questo cambiamento elettrico fa aprire le proteine che trasportano il potassio, permettendo l'ingresso dello ione nella cellula. Tali proteine sono chiamate trasportatori voltaggio-dipendenti.
Perché entra il potassio e non uno ione carico positivamente? Per una ragione di gradiente elettrico e chimico. Stiamo buttando fuori cariche positive, quindi all'interno saranno richiamate più facilmente cariche positive. Entra il potassio, non il cloro, e tendenzialmente non il sodio, perché è troppo grande. Successivamente, interviene un trasportatore secondario, un simporto, che trasporta protoni e zuccheri. Gli zuccheri sono specificamente attivi, come visto in precedenza, per contrastare la concentrazione dei soluti. Tutto questo flusso di ioni e zuccheri fa diminuire il potenziale negativo della cellula di guardia, che inizia a richiamare acqua dalle cellule adiacenti.\footnote{The repetitions were removed to improve clarity.  The expression "Qui si vede meglio" was repeated three times and was omitted as it seems to refer to a visual aid not included in the text. The part "Qui e tutto vuoto, e bello pieno, eeeh" has been interpreted and rewritten to clarify the apparent contradiction in the original statement.}
A livello di ordine 6, con luce blu, si attiva una catena di trasduzione del segnale. Questa attiva la pompa protonica della membrana plasmatica, che estrude protoni generando un'iperpolarizzazione della membrana. L'iperpolarizzazione inattiva il canale del potassio voltaggio-dipendente, che normalmente fa entrare ioni $K^+$. Questi ioni $K^+$ partecipano alla formazione del complesso CD.S. Infine, si osserva una diminuzione di ossido, probabilmente riferendosi a superossido ($O_2^−$).\footnote{Minor disfluencies and repetitions were removed for clarity.}
Il protone, entrando nel citosol, trascina con sé molecole osmoticamente attive, come gli zuccheri, aumentando la complessità della situazione. Questa complessità è dovuta all'elevato numero di molecole che entrano nel citosol, le quali non possono rimanervi permanentemente per non compromettere le attività cellulari. Pertanto, si attiva la pompa protonica del vacuolo, situata nel tonoplasto. Questa pompa trasporta le molecole osmoticamente attive all'interno del vacuolo, contribuendo al potenziale idrico senza interferire con l'omeostasi cellulare. In questo modo, il citosol mantiene il suo pH e la concentrazione dei suoi componenti. Le molecole in eccesso vengono trasportate all'esterno attraverso la membrana plasmatica, tramite i citocinini, e accumulate nel vacuolo. Così, il potenziale idrico e le condizioni di osmolarità compatibili con la vita cellulare vengono mantenuti.\footnote{Il testo originale conteneva alcune frasi confuse e ripetitive. In particolare, la frase "manmanco una cellula sacrete, viene spostata sempre piu in cuori" non aveva un significato chiaro ed è stata omessa nella versione processata. Si è cercato di interpretare al meglio il significato generale del discorso, focalizzandosi sulla formazione e la struttura della parete cellulare.}
L'attivazione dell'apertura stomatica, che permette l'apertura dello stoma e garantisce il progredire della fotosintesi, la movimentazione dell'acqua e altre attività, avviene tramite un sistema complesso. Il movimento delle molecole organiche, in particolare del potassio, avanti e indietro, mantiene uno squilibrio di concentrazione tra l'ambiente extracellulare e intracellulare. Questo squilibrio, simile a quello utilizzato per la generazione degli impulsi nervosi, permette la generazione di correnti elettriche.  Il potassio, in particolare, si muove avanti e indietro tra i due ambienti. Questo meccanismo di squilibrio è mantenuto in quasi tutti i sistemi biologici e viene utilizzato per generare correnti elettriche, come ad esempio negli squali. Infine, riguardo alla domanda sulla diminuzione del muscolo e l'aumento del volume, si specifica che lo stimolo interviene solo quando il suono è aperto e il prodotto non sbatte contro le pareti.\footnote{Il riferimento alle "caramelle" disegnate dagli amici è stato omesso perché non aggiunge informazioni rilevanti alla spiegazione. L'ultima frase riguardo al "pezzettino di legno" è stata omessa perché sembra fuori contesto e frammentaria.}
Quando lo stoma è ben aperto, a livello di catene ossidriliche, blocca l'ingresso dell'acqua. L'umidità... sì. Il tessuto conduttore adulto funzionante in una pianta, chiamato legno, è formato da singole cellule morte dette cellule xilematiche.  Per svolgere la funzione di trasporto dell'acqua, è più comodo che la cellula sia priva di protoplasma.
Stiamo parlando della linfa grezza, non di quella elaborata per il trasporto degli zuccheri. Per trasportarla è più comodo avere un tubo resistente alle alte pressioni, come quelle che si verificano durante la traspirazione dell'acqua attraverso le pareti cellulari. Quindi il tessuto migliore per il trasporto della linfa grezza, composta da acqua e sali minerali disciolti, è formato da cellule morte, di cui rimane solo la parete. Questo tessuto, detto xilematico, fa parte dei tessuti vascolari ed è caratteristico delle piante terrestri. Le cellule di questo tessuto, impilate le une sulle altre, prendono il nome di trachee o tracheidi. L'argomento dei prossimi 10-20 minuti sarà di nuovo la parete cellulare. Essa presenta caratteristiche diverse a seconda del tessuto, non a livello molecolare, ma in termini di spessore, robustezza e presenza o assenza del protoplasma all'interno. Le molecole che la compongono sono sempre le stesse, ma lo spessore può variare. La parete cellulare è una struttura esterna che offre protezione e robustezza. Nel tessuto xilematico, le trachee conferiscono rigidità alla pianta grazie alle pareti di cellulosa, che aiutano la pianta a mantenersi eretta e integra.\footnote{The original phrase "o altre robe" has been normalized to "o altre tipologie cellulari" to enhance clarity and academic style, while maintaining the original meaning.}
Quindi, tutte le cellule vegetali hanno una parete cellulare. Come fa una cellula ad assumere forme specifiche, come quelle mostrate nell'immagine B della slide? La forma è determinata dalla parete cellulare, che è controllata in modo molto preciso. Questo controllo detta come, dove e quando la parete deve essere formata, determinando la morfologia della pianta e delle singole cellule. Ad esempio, il tessuto epidermico nella figura B ha una morfologia particolare, con cellule quadrate. Perché si è formato in quel modo?
No, non sarebbe più facile una pestrellatura di questo tipo? Cioè, una pestrellatura così mi garantisce un tessuto epidermico che tappa tutto. Può essere quadrata, ma questo fenomeno, che sia con cellule allineate o con cellule di quelle dimensioni, è sotto il controllo genetico, non è casuale. È un controllo genetico ben preciso che fa sì che la parete cellulare in quelle cellule non sia distribuita in modo omogeneo, ma presenti dei punti in cui è più spessa, meno spessa, finché non si aggancia con la cellula contigua. Indipendentemente da tutto, però, la parete cellulare, che siano gli irrobustimenti radiali delle cellule, la parete cellulare della rima stomatica, o la parete cellulare dell'epidermide, ha sempre la stessa composizione. Cioè, le molecole che compongono la parete cellulare sono sempre le stesse, non cambiano, e sono quasi tutti zuccheri. Il grosso della parete cellulare è formato da zuccheri. Da qui nasce l'interesse biotecnologico. Se riuscissimo a liberare facilmente tutti quegli zuccheri che sono incastrati nella parete cellulare, che sono la biomassa, e ad ottenere energia rompendoli, forse potremmo anche abbandonare l'utilizzo dei biocarburanti, dei combustibili fossili. Perché lì, proprio in questa struttura, c'è un'enorme quantità di energia.
I fosfofori che derivano dal carbone non vengono spesso utilizzati, quindi c'è tantissima energia. Il problema è che non si riesce a liberare facilmente, in modo efficiente e non inquinante i singoli zuccheri presenti nella parete cellulare. Però ha un'applicazione enorme. Bene, questa è la rappresentazione classica, da libro di testo, di una parete cellulare. Queste due immagini, entrambe al microscopio elettronico a trasmissione, mostrano, nell'immagine in alto a destra, una cellula circondata da parete.
Focalizziamoci su questa cellula. Come potete vedere, è circondata da una parete. Questa è la parete. All'interno della cellula, avete correttamente identificato il nucleo. Questo è il nucleo.
Questo è il nucleolo. Tutta questa parte più estesa è il nucleo. Questi oggetti più grandi sono i cloroplasti.
Questi sono i mitocondri. Ricordatevi che le cellule vegetali hanno sempre i mitocondri, ma possono non avere i cloroplasti. Quindi, quelli più piccoli sono i mitocondri.
Questi sono vacuoli. Il nome "vacuolo" deriva dalle prime osservazioni al microscopio che mostravano uno spazio vuoto all'interno della cellula. Sebbene nelle immagini appaiano vuoti, in realtà sono pieni. L'etimologia della parola, però, deriva da "vuoto".
Bene. Come dicevo prima, nelle cellule vegetali, il luogo di nascita di una cellula è anche il suo luogo di permanenza, poiché sono sempre incastrate in una parete cellulare. Quindi, le cellule vegetali non possono spostarsi. Una pianta si muove, ad esempio, in direzione della luce, ma questo movimento avviene attraverso la divisione cellulare: la pianta cresce, continua a dividersi e si allunga in quella direzione. Ma se tre cellule nascono vicine, rimarranno vicine per sempre, impossibilitate a spostarsi a causa della parete cellulare che le blocca. Sono incastonate all'interno della parete cellulare e, come abbiamo detto, l'insieme delle pareti cellulari dell'intero organismo vegetale si chiama apoplasto. L'immagine a sinistra mostra un ingrandimento di una parete cellulare: si vede una spessa struttura.
La parete cellulare si forma durante la mitosi, la divisione cellulare. Appena una nuova cellula figlia nasce, la sua parete cellulare è già completa, impedendole di spostarsi; i suoi vicini cellulari saranno quelli definitivi. La prima struttura che si forma nella parete cellulare, essendo la più esterna rispetto alla membrana plasmatica, è la lamella mediana. Questa struttura, la più "giovane" della parete, si forma durante la divisione cellulare. La parete cellulare è sintetizzata dal protoplasma della cellula vegetale. Quindi, la porzione più esterna della parete cellulare, presente in tutte le cellule, è la lamella mediana, che appare come una sorta di gel.
Gli zuccheri sono i principali costituenti della lamella mediana, la prima parte della parete cellulare che si forma. La parete primaria, formata successivamente, è composta anch'essa da zuccheri, detti polisaccaridi, ma con una maggiore percentuale di componente fibrillare, che conferisce resistenza. In alcuni casi, come nei tessuti conduttori delle piante vascolari, si forma anche una parete secondaria, ancora più esterna e robusta. La formazione della parete secondaria porta alla morte del protoplasto, lasciando un tubo vuoto con una parete spessa, utilizzata per costruire oggetti come travi e traversine ferroviarie. Quindi, quando ci si siede su una panca di legno, si sta sedendo sulla parete cellulare delle piante.
Quegli elementi vanno a costituire l'insieme delle pareti cellulari. Quando si contano gli anni in un frusto, si contano le pareti cellulari che si sono formate nel corso degli anni. E quando diventano così spesse, sono sostanzialmente date dalla presenza di una parete secondaria, caratteristica dei tessuti conduttori. Però, indipendentemente dalla presenza della parete secondaria, quella cellula dotata di parete secondaria ha avuto un momento in cui aveva la lamella mediana, la parete primaria e poi la parete secondaria. Più si sviluppa la parete, più si accresce, più la lamella mediana scompare, perché a un certo punto viene schiacciata e scompare.
I melanoplasti o altre tipologie cellulari presentano sempre i mitocondri.
\end{spacing}
\end{document}
\documentclass[11pt, a4paper]{article}
\usepackage[utf8]{inputenc}
\usepackage[T1]{fontenc}
\usepackage{lmodern}
\usepackage{microtype}
\usepackage[margin=0.75in]{geometry}
\usepackage{parskip}
\usepackage{setspace}
\usepackage{amsmath}
\usepackage{amsfonts}
\usepackage{xcolor}
\usepackage[autostyle, english = american]{csquotes}
\MakeOuterQuote{"}
\setlength{\parindent}{1em}

\title{Transcription}
\author{ClearNotes}
\date{\today}

\begin{document}
\maketitle
\begin{spacing}{1.15}
Questa slide presenta un esercizio sul potenziale idrico, applicabile in diverse condizioni. Come promemoria, il potenziale idrico, soprattutto tra cellule contigue, è definito come $\psi_w = \psi_P + \psi_S$, dove $\psi_P$ è il potenziale di pressione e $\psi_S$ il potenziale osmotico. Il potenziale osmotico, $\psi_S$, è calcolato come $\psi_S = -RT\cdot C_S$, dove R è la costante dei gas, T la temperatura e $C_S$ la concentrazione dei soluti.  $\psi_S$ è l'inverso del potenziale isomorfico a causa del segno negativo. La formula $-RT\cdot C_S$ descrive efficacemente la pressione osmotica. Per determinare il potenziale idrico ($\psi_s$) di una soluzione, è necessario conoscere il valore della costante dei gas ($R$), la temperatura ($T$) e la concentrazione dei soluti. Considerando un contenitore con acqua pura, il suo potenziale idrico ($\psi_w$) è zero perché non ci sono soluti né pressioni esercitate dalle pareti del contenitore. Se si aggiunge saccarosio, ad esempio una bustina di zucchero, la concentrazione dei soluti cambia. Il saccarosio è lo zucchero comunemente usato nel caffè. Sciogliendo una quantità di saccarosio pari a 0,1 molare in acqua, otteniamo una soluzione con concentrazione finale di 0,1 M. Il potenziale idrico ($\Psi$) di questa soluzione si calcola con la formula $\Psi = \Psi_s + \Psi_p = -RTC_s + \Psi_p$, dove $C_s$ è la concentrazione della soluzione (0,1 M), $R$ è la costante dei gas e $T$ la temperatura in Kelvin (293 K, corrispondenti a 20 °C).  Poiché non ci sono pareti che esercitano pressione, $\Psi_p$ è nullo. Quindi, a 20 °C, il potenziale idrico è pari a $-0,244$ MPa. In questa soluzione, contenente saccarosio 0,1 M, immergiamo due tipi di cellule: una cellula turgida e una cellula flaccida. Due cellule poste nella stessa soluzione si comporteranno in modo differente. La cellula turgida, con una concentrazione di 0,244, posta in una soluzione, perderà acqua verso l'esterno a causa della minore concentrazione di soluti, diventando flaccida. Al contrario, una cellula in condizioni tipiche vegetali assumerà acqua fino a quando il potenziale di turgore ($\psi_P$) non impedirà un ulteriore ingresso, raggiungendo la turgidità quando il suo potenziale idrico eguaglierà quello della soluzione. Queste condizioni sono tipiche di un ambiente fisiologico.  L'immagine in alto a sinistra mostra il comportamento di una pianta in apnea fisiologica: la pianta si affloscia e perde turgore a causa della perdita d'acqua. Quindi, il turgore di una pianta dipende dal suo contenuto d'acqua. Se la pianta non viene manipolata eccessivamente, è in grado di recuperare bene da eventuali stress idrici in 24-48 ore, ripristinando la turgidità e i flussi verticali. Questo recupero è possibile perché il potenziale idrico delle radici nel vaso è minore di quello del suolo dopo l'annaffiatura. Quindi, aggiungendo acqua al suolo, si crea un gradiente di potenziale idrico che favorisce l'assorbimento di acqua dalle radici. Tuttavia, l'ingresso di acqua dalle radici non è sufficiente: l'acqua deve essere trasportata attraverso tessuti specifici, un adattamento evolutivo per la colonizzazione delle terre emerse assente nei muschi.  Il trasporto dell'acqua avviene anche grazie alle aperture stomatiche che permettono la traspirazione. Si crea così una catena di molecole d'acqua che attraversa la pianta. L'apertura e chiusura degli stomi regola questo processo e, di conseguenza, numerosi eventi fisiologici della pianta. L'acqua risale nella pianta principalmente quando le aperture stomatiche sono aperte, permettendo la traspirazione. Il flusso d'acqua attraverso la pianta è vitale, e la chiusura degli stomi causa stagnazione. La traspirazione, ovvero la fuoriuscita di acqua dagli stomi, genera una forza trainante che richiama acqua dal suolo verso la pianta. Questa perdita d'acqua permette alla pianta di mantenere un potenziale idrico interno leggermente inferiore a quello del suolo, condizione necessaria per l'assorbimento. Le cellule di guardia, fondamentali per la regolazione del flusso d'acqua e l'ingresso di $\text{CO}_2$ per la fotosintesi, controllano l'apertura e la chiusura degli stomi.  Passando da una forma chiusa ad una aperta tramite la regolazione del contenuto d'acqua intracellulare,  queste cellule sfruttano la particolare struttura della rima stomatica: una parete cellulare ispessita, una meno ispessita in una regione specifica e una disposizione radiale delle pareti cellulari. Questa configurazione consente l'allargamento o la chiusura dello stoma in risposta al movimento dell'acqua, trasmettendo così uno stimolo iniziale. Come state facendo con la traduzione del segnale stimolo, che cos'è che determina l'apertura dello stoma secondo voi? Quello che è scritto nella slide? Sì. Lo stoma non è sempre aperto, non è sempre aperto, si apre e si chiude. Si chiude in apertura alla siccità, ma non è sufficiente. Cioè, un trasporto... le piante devono soffrire, perché sennò non soffrono. Qual è lo stimolo più importante che innesca l'apertura dello stoma secondo voi? Il peso. Lo stimolo principale, il primo stimolo che in condizioni fisiologiche determina il benessere di una pianta e l'apertura stomatica è la luce, in particolare la luce blu dell'alba. Gli stomi sono quasi sempre poco aperti, per il 90-92\% del tempo, perché le condizioni ambientali per una pianta, se adattata bene, sono abbastanza estreme. Ad esempio, in estate, se gli stomi fossero sempre aperti nelle ore centrali della giornata, la pianta perderebbe una grande quantità di acqua. Quindi, tendenzialmente nelle ore centrali della giornata gli stomi sono socchiusi per limitare la perdita d'acqua. In caso contrario, le piante avrebbero bisogno di essere continuamente innaffiate o attraversate da fiumi d'acqua. Le piante sono tendenzialmente più fotosinteticamente attive all'inizio della giornata, piuttosto che nelle ore centrali di una giornata estiva. Infatti, l'intensità luminosa nelle ore centrali estive è eccessiva per le piante, creando condizioni di stress. La luce fotosinteticamente attiva, pur essendo necessaria, deve essere presente in quantità adeguate. Nelle giornate estive molto calde, con forte irraggiamento solare, le piante subiscono un forte stress a causa dell'eccesso di luce, che causa la produzione di specie reattive dell'ossigeno dannose. Pertanto, le piante attivano meccanismi di difesa per proteggersi. Le condizioni ottimali per la fotosintesi si verificano in presenza di luce non eccessiva, ad esempio in condizioni di ombreggiamento o con cielo nuvoloso, evitando l'esposizione diretta al sole nelle ore più calde. La fotosintesi avviene con un'efficienza relativamente bassa: la quantità di energia luminosa che si traduce in massa è molto inferiore all'energia totale incidente sulla pianta. L'apertura stomatica è principalmente indotta dalla luce blu, tipica dell'alba.  Questa induce l'apertura degli stomi, che poi tendono a richiudersi per limitare la perdita d'acqua. La luce blu è percepita da un fotorecettore, una proteina sensibile a questa lunghezza d'onda, che innesca una catena di trasduzione del segnale. Sebbene non approfondiremo nel dettaglio la specifica catena di trasduzione per l'apertura delle cellule di guardia, è importante comprendere che lo stimolo luminoso dell'alba si traduce nell'apertura degli stomi.  A valle della trasduzione del segnale, qual è il primo effettore, la proteina che agisce direttamente per determinare l'apertura stomatica? La risposta è la pompa protonica. La pompa protonica, situata sia sulla membrana plasmatica che sul vacuolo, consuma ATP per estrudere protoni. Non è sempre attiva, ma si attiva in presenza di luce blu. Un recettore specifico percepisce la luce blu e, tramite una catena di trasduzione del segnale, attiva la pompa protonica. L'estrusione di protoni, a sua volta, attiva un processo a valle non ancora specificato. Dopo il trasporto secondario, l'ingresso di potassio attiva un canale voltaggio-dipendente per il potassio. Avviene una traduzione dei nuovi con altri nuovi nello stesso modo. In seguito all'attivazione della pompa protonica, si attiva una proteina carrier che trasporta potassio. Questo canale voltaggio-dipendente, estrudendo protoni, causa una variazione del pH della membrana plasmatica a seguito dell'espulsione di protoni da parte della pompa protonica. Spostare protoni attraverso la membrana plasmatica, per mezzo di pompe protoniche e con consumo di ATP, genera una differenza di concentrazione che ha due componenti: una chimica e una elettrica. La componente chimica è data dalla differenza di concentrazione protonica, mentre quella elettrica è dovuta allo spostamento di cariche. Questo spostamento di cariche modifica il potenziale di membrana. Tale variazione di potenziale elettrico attiva specifici canali proteici per il potassio, detti trasportatori voltaggio-dipendenti, permettendo così l'ingresso del potassio nella cellula. Il potassio, ione carico positivamente, entra nella cellula per gradiente elettrico e chimico. L'uscita di cariche positive favorisce l'ingresso di cariche positive, quindi il potassio entra più facilmente del cloro. Il sodio, invece, non entra a causa delle sue dimensioni. Successivamente, un trasportatore secondario, un simporto di protoni e zuccheri, trasporta gli zuccheri attivi. L'accumulo di ioni e zuccheri fa diminuire il potenziale negativo della cellula di guardia, richiamando acqua dalle cellule adiacenti. A livello di ordine 6, la luce blu attiva la catena di trasduzione del segnale, che a sua volta attiva la pompa protonica della membrana plasmatica. L'estrusione dei protoni genera un'iperpolarizzazione della membrana, inattivando il canale del potassio. Contrariamente, l'inattivazione del canale del potassio permette l'ingresso degli ioni $K^+$ che partecipano alla formazione del complesso CD.S. Questo processo porta alla diminuzione dell'ossido di over 2. Il protone, entrando nel citosol, trascina con sé molecole osmoticamente attive, come gli zuccheri, aumentando la complessità della situazione. Questa complessità è dovuta all'elevato numero di molecole che, entrando nel citosol, non possono rimanervi permanentemente per non compromettere la sua attività. Si attiva quindi la pompa protonica del vacuolo, situata nel tonoplasto. Le molecole osmoticamente attive vengono così temporaneamente accumulate nel vacuolo, contribuendo al potenziale idrico senza interferire con l'omeostasi cellulare.  Il vacuolo mantiene l'omeostasi del citosol, preservando il pH e la concentrazione delle sostanze. Le molecole osmoticamente attive, attraversando la membrana plasmatica e il citoplasma, raggiungono il vacuolo, mantenendo il potenziale idrico e l'osmolarità cellulare compatibile con la vita. L'apertura stomatica è il meccanismo che permette l'apertura degli stomi e garantisce il progredire della fotosintesi, la movimentazione dell'acqua e altre attività. Il movimento degli ioni, in particolare del potassio ($K^+$), avanti e indietro attraverso la membrana cellulare, mantiene uno squilibrio di concentrazione tra l'ambiente extracellulare e intracellulare. Questo squilibrio, simile a quello mantenuto per gli ioni $Na^+$ e $K^+$ nei neuroni, genera correnti elettriche utilizzate per vari processi biologici.  Come esempio, la riduzione del turgore cellulare provoca la chiusura degli stomi. Quando gli stomi sono aperti, lo stilo interviene solo quando il prodotto si accumula contro le pareti. Quando lo stoma è ben aperto, a livello di catene ossidriliche, blocca l'ingresso dell'acqua per via dell'umidità. Il tessuto conduttore adulto funzionante in una pianta, chiamato legno, è formato da singole cellule morte dette cellule xilematiche. Per svolgere la funzione di trasporto dell'acqua, è più funzionale che la cellula sia morta e priva di protoplasma. Il tessuto più adatto per il trasporto della linfa grezza (acqua e sali minerali disciolti), dovendo resistere a pressioni elevate come quelle che si verificano durante la traspirazione, è il tessuto xilematico. Questo tessuto vascolare, caratteristico delle piante terrestri, è formato da cellule morte di cui rimane solo la parete cellulare. Le cellule del tessuto xilematico, impilate le une sulle altre, formano le trachee o tracheidi. La parete cellulare, componente fondamentale di questo tessuto, presenta caratteristiche di spessore e robustezza variabili a seconda del tessuto, ma la sua composizione molecolare rimane costante. La sua funzione principale è quella di conferire protezione e robustezza. Nel caso del tessuto xilematico, la rigidità delle pareti cellulari, composte principalmente da cellulosa, contribuisce al sostegno della pianta, permettendole di mantenersi eretta e di preservare la sua integrità strutturale. Tutte le cellule vegetali hanno una parete cellulare, come mostrato nell'immagine B della slide iniziale. La forma di una cellula vegetale è determinata dalla sua parete, che è controllata con precisione in termini di come, dove e quando si forma. Questo controllo preciso sulla formazione della parete cellulare influenza la morfologia della pianta e delle singole cellule. Ad esempio, il tessuto epidermico nell'immagine B presenta una morfologia insolita con cellule quadrate. Bisogna quindi chiedersi perché le cellule epidermiche assumono quella specifica forma quadrata. Una pestrellatura con un tessuto epidermico che tappa tutto, di forma quadrata e con cellule allineate, è garantita da un controllo genetico preciso. Questo controllo fa sì che la parete cellulare in quelle cellule non sia distribuita omogeneamente, ma presenti punti di diverso spessore fino ad agganciarsi con la cellula contigua. La composizione della parete cellulare, sia negli irrobustimenti radiali delle cellule, sia nella lima stomatica, sia nell'epidermide, è sempre la stessa ed è formata principalmente da zuccheri. L'interesse biotecnologico in questo campo nasce dalla possibilità di liberare facilmente l'energia contenuta in quegli zuccheri, che costituiscono la biomassa, e di abbandonare l'utilizzo dei combustibili fossili. La quantità di energia presente in questa struttura è enorme. Non si opera spesso con il carbone, i fosfati che arrivano da questi, quindi c'è tantissima energia. Il problema è che non si riesce a liberare facilmente e in modo efficiente e non inquinante i singoli zuccheri presenti nella parete cellulare, però ha un'applicazione enorme. Bene, questa è come viene classicamente disegnata una parete cellulare da libro di testo. Queste sono due immagini, entrambe al microscopio elettronico a trasmissione, che mostrano una cellula circondata da parete nell'immagine in alto a destra. Focalizzando l'attenzione su questa cellula, circondata da una parete, possiamo individuare al suo interno il nucleo. Questo è il nucleolo. Tutta questa parte più estesa è il nucleo. Questi oggetti più grandi sono i cloroplasti. I mitocondri sono presenti in tutte le cellule vegetali, a differenza dei cloroplasti che possono anche non esserci. I mitocondri sono quegli organelli di dimensioni decisamente inferiori rispetto ad altri presenti nella cellula. Osservando le immagini, si possono notare i vacuoli. Il termine "vacuolo" deriva dalle prime osservazioni al microscopio, che mostravano queste strutture come sacche vuote. Sebbene nelle immagini appaiano principalmente come spazi vuoti, il nome mantiene l'etimologia originale di "vuoto". In realtà, i vacuoli sono pieni, ma la loro definizione iniziale rimane legata all'aspetto vuoto che presentavano nelle prime analisi microscopiche. Come dicevo prima, nelle cellule vegetali, il luogo di nascita di una cellula è anche il luogo in cui essa rimane per tutta la sua vita, in quanto incastrata nella parete cellulare. Il movimento di una pianta, ad esempio in direzione della luce, non avviene tramite spostamento delle singole cellule, ma attraverso la divisione cellulare: la pianta cresce, le cellule si dividono e la pianta si allunga nella direzione desiderata. Pertanto, cellule vicine al momento della nascita rimarranno vicine per sempre, incastonate e impossibilitate a spostarsi a causa della parete cellulare.  L'insieme di tutte le pareti cellulari di un organismo vegetale costituisce l'apoplasto. L'immagine a sinistra mostra un ingrandimento di una parete cellulare. La parete cellulare si forma durante la mitosi, la divisione cellulare, definendo il confine tra due cellule figlie. Appena formata la nuova cellula, la parete è già completa, impedendo ulteriori spostamenti e definendo la sua posizione rispetto alle cellule adiacenti. La prima struttura che si forma nella parete cellulare, essendo la più esterna rispetto alla membrana plasmatica, è la lamella mediana. Questa viene secreta dalla cellula durante la divisione cellulare e, essendo sintetizzata dal protoplasma, spinge la lamella mediana verso l'esterno. La lamella mediana, presente in tutte le cellule vegetali, è la porzione più esterna della parete cellulare ed ha l'aspetto di un gel. Gli zuccheri sono i principali costituenti della lamella mediana e della parete primaria, la seconda parte della parete cellulare che si forma. La parete primaria, priva della consistenza gelatinosa della lamella mediana, è composta da zuccheri diversi o con una maggiore percentuale di componente fibrillare, che conferisce resistenza. Nei tessuti conduttori delle piante vascolari, si forma anche una parete secondaria, più robusta, che spinge la parete primaria verso l'esterno. La formazione della parete secondaria causa la morte del protoplasto, lasciando un tubo vuoto con una parete spessa, utilizzata per costruire travi e traversine ferroviarie. Quindi, sedendosi su una panca di legno, si sta sedendo sulla parete cellulare. Gli anelli di accrescimento annuali in un frusto di albero sono dati dall'insieme delle pareti cellulari che si formano nel corso degli anni. Lo spessore degli anelli è determinato principalmente dalla presenza di una parete secondaria, caratteristica dei tessuti conduttori. È importante ricordare che anche nelle cellule con parete secondaria, la formazione cellulare inizia con la lamella mediana, seguita dalla parete primaria e infine dalla parete secondaria. Con l'accrescimento e lo sviluppo della parete cellulare, la lamella mediana viene progressivamente compressa fino a scomparire. I melanoplasti, come altri elementi cellulari, contengono sempre i mitocondri.
\end{spacing}
\end{document}
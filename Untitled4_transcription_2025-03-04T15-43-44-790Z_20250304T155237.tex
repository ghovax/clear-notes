\documentclass[11pt, a4paper]{article}
\usepackage[utf8]{inputenc}
\usepackage[T1]{fontenc}
\usepackage{lmodern}
\usepackage{microtype}
\usepackage[margin=0.75in]{geometry}
\usepackage{parskip}
\usepackage{setspace}
\usepackage{amsmath}
\usepackage{amsfonts}
\usepackage{xcolor}
\usepackage[autostyle, english = american]{csquotes}
\MakeOuterQuote{"}
\setlength{\parindent}{1em}

\title{Transcription}
\author{ClearNotes}
\date{\today}

\begin{document}
\maketitle
\begin{spacing}{1.15}
Se, per esempio, abbiamo $
ewline 
$\psi_W$ e ad essa aggiungiamo $3X$, dove $X$ è la concentrazione osmotica del soluto, allora possiamo ottenere il $K^+$, che è la concentrazione dei protoni, gli ioni $H^+$ a cui è stato strappato un elettrone.
\end{spacing}
\end{document}
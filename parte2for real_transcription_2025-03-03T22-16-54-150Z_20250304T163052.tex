\documentclass[11pt, a4paper]{article}
\usepackage[utf8]{inputenc}
\usepackage[T1]{fontenc}
\usepackage{lmodern}
\usepackage{microtype}
\usepackage[margin=0.75in]{geometry}
\usepackage{parskip}
\usepackage{setspace}
\usepackage{amsmath}
\usepackage{amsfonts}
\usepackage{xcolor}
\usepackage[autostyle, english = american]{csquotes}
\MakeOuterQuote{"}
\setlength{\parindent}{1em}

\title{Transcription}
\author{ClearNotes}
\date{\today}

\begin{document}
\maketitle
\begin{spacing}{1.15}
Questa slide presenta un esercizio sull'applicazione del concetto di potenziale idrico in cellule contigue. Come promemoria, il potenziale idrico, indicato con $\psi_w$, è influenzato dalla pressione della parete cellulare ($\psi_P$) e dalla presenza di soluti ($\psi_S$). Il potenziale idrico dei soluti, $\psi_S$, è calcolato come l'opposto del potenziale osmotico: $\psi_S = -R \cdot T \cdot C_S$, dove R è la costante dei gas, T è la temperatura e $C_S$ è la concentrazione dei soluti. Questa formula, con il segno negativo, mette in evidenza la relazione inversa tra $\psi_S$ e la concentrazione dei soluti. Per determinare il potenziale idrico ($\psi_s$) di una soluzione, è necessario conoscere il valore della costante dei gas ($R$), la temperatura ($T$) e la concentrazione dei soluti. Considerando un contenitore con acqua pura, il suo potenziale idrico ($\psi_w$) è zero perché non ci sono soluti né pressioni esercitate dalle pareti del contenitore. Se si aggiunge saccarosio, ad esempio una bustina di zucchero, la concentrazione dei soluti cambia. Il saccarosio è lo zucchero comunemente usato nel caffè. Sciogliendo una quantità di saccarosio pari a 0,1 molare in acqua, otteniamo una soluzione con concentrazione finale di 0,1 M. Il potenziale idrico ($\Psi$) di questa soluzione si calcola con la formula $\Psi = \Psi_s + \Psi_p = -RTC_s + \Psi_p$, dove $C_s$ è la concentrazione di saccarosio (0,1 M), $R$ è la costante dei gas e $T$ la temperatura in Kelvin (293 K, corrispondenti a 20 °C).  Il termine $\Psi_p$ è nullo in assenza di pressione. Quindi, a 20 °C, il potenziale idrico è pari a $-0,244$ MPa. In questa soluzione, immergiamo due tipi di cellule: una cellula turgida e una cellula flaccida. Due cellule poste nella stessa soluzione si comporteranno in modo differente. La cellula turgida, con una concentrazione di 0,244, posta in una soluzione, perderà acqua verso l'esterno a causa della minore concentrazione di soluti, diventando flaccida. Al contrario, una cellula in condizioni tipiche di una cellula vegetale, assumerà acqua fino a quando il potenziale di turgore ($\psi_P$) non impedirà un ulteriore ingresso di acqua, raggiungendo la turgidità quando il suo potenziale idrico eguaglierà quello della soluzione. Queste condizioni sono tipiche di un ambiente fisiologico.  L'immagine in alto a sinistra mostra il comportamento di una pianta in apnea fisiologica: la pianta si affloscia e perde turgore a causa della perdita d'acqua. Quindi, la vita di una pianta è in parte determinata dal suo contenuto d'acqua. Se la pianta non viene manipolata eccessivamente, è in grado di recuperare bene da eventuali stress idrici in 24-48 ore, ripristinando la turgidità e i flussi verticali. Questo recupero è possibile perché il potenziale idrico delle radici nel vaso è minore di quello del suolo dopo l'annaffiatura. Quindi, aggiungendo acqua al suolo, si crea un gradiente di potenziale idrico che favorisce l'assorbimento di acqua dalle radici. Tuttavia, l'ingresso di acqua dalle radici non è sufficiente: l'acqua deve essere trasportata attraverso tessuti specifici, un adattamento evolutivo per la colonizzazione delle terre emerse assente nei muschi.  Il trasporto dell'acqua avviene anche grazie alle aperture stomatiche che permettono la traspirazione. Si crea così una catena di molecole d'acqua che attraversa la pianta. L'apertura e chiusura degli stomi regola questo processo e, di conseguenza, numerosi eventi fisiologici della pianta. L'acqua risale nella pianta principalmente quando le aperture stomatiche sono aperte, permettendo la traspirazione. Il flusso d'acqua attraverso la pianta è vitale, e la chiusura degli stomi causa stagnazione. La traspirazione, ovvero la fuoriuscita di acqua dagli stomi, genera una forza trainante che richiama acqua dal suolo verso la pianta. Questa perdita d'acqua permette alla pianta di mantenere un potenziale idrico interno leggermente inferiore a quello del suolo, favorendo l'assorbimento. Le cellule di guardia, fondamentali per la regolazione del flusso d'acqua e l'ingresso di $\text{CO}_2$ per la fotosintesi, controllano l'apertura e la chiusura degli stomi.  Passando da una forma chiusa ad una aperta tramite la regolazione del contenuto d'acqua intracellulare,  queste cellule sfruttano la particolare struttura della rima stomatica: una parete cellulare ispessita, una meno ispessita in una regione specifica e una disposizione radiale della parete cellulare permettono l'allargamento o restringimento delle cellule, regolando così l'apertura stomatica in funzione del movimento dell'acqua, trasmettendo uno stimolo di partenza Come già detto, state effettuando la traduzione del segnale stimolo. Cos'è che determina l'apertura dello stoma secondo voi? Quello che è scritto nella slide? Sì. Lo stoma non è sempre aperto, non è sempre aperto come un poro. Si apre e si chiude. Si chiude in risposta alla siccità, ma non è sufficiente. Cioè, un trasporto... le piante devono soffrire, altrimenti non soffrono. Qual è lo stimolo più importante che innesca l'apertura dello stoma secondo voi? Il peso. Lo stimolo principale, il primo stimolo che in condizioni fisiologiche determina il benessere di una pianta e l'apertura stomatica è la luce, in particolare la luce blu dell'alba. Gli stomi sono quasi sempre poco aperti, circa nel 90-92% dei casi, perché le condizioni ambientali per una pianta, se adattata bene, sono abbastanza estreme. Ad esempio, in estate, se gli stomi fossero sempre aperti nelle ore centrali della giornata, la pianta perderebbe una grande quantità di acqua. Quindi, tendenzialmente nelle ore centrali della giornata gli stomi sono socchiusi per limitare la perdita d'acqua. In caso contrario, le piante avrebbero bisogno di essere continuamente innaffiate o attraversate da fiumi d'acqua. Le piante sono più fotosinteticamente attive durante il giorno, ma all'inizio della giornata piuttosto che nelle ore centrali. Infatti, l'intensità luminosa nelle ore centrali delle giornate estive può essere eccessiva, creando condizioni di stress per le piante. La luce fotosinteticamente attiva, oltre ad avere una specifica composizione spettrale, deve essere presente in quantità adeguate. Nelle giornate estive molto calde e con forte irraggiamento, le piante possono subire danni a causa dell'eccesso di luce, che porta alla formazione di specie reattive dell'ossigeno dannose. Pertanto, le piante attivano meccanismi di difesa per proteggersi da queste condizioni. Le condizioni migliori per la fotosintesi si verificano in presenza di luce non eccessiva, ad esempio in condizioni di leggera ombreggiatura o con cielo nuvoloso. La fotosintesi avviene con un'efficienza relativamente bassa: la quantità di energia luminosa che si traduce in massa è molto inferiore all'energia incidente totale sulla pianta. L'apertura stomatica è principalmente indotta dalla luce blu, tipica dell'alba.  Questa induce l'apertura degli stomi, che poi tendono a chiudersi per evitare un'eccessiva perdita d'acqua. La luce blu è percepita da un fotorecettore, una proteina sensibile a questa lunghezza d'onda, che innesca una catena di trasduzione del segnale. Sebbene non approfondiremo nel dettaglio la specifica catena di trasduzione per l'apertura delle cellule di guardia, è importante comprendere che la luce blu, e quindi l'alba, è lo stimolo principale per l'apertura degli stomi.  Questo stimolo viene tradotto in un'azione specifica: qual è il primo evento che determina l'apertura stomatica? E, una volta avvenuta la trasduzione del segnale, su quale effettore agisce? Quale proteina effettua l'azione di apertura? La risposta è la pompa protonica. La pompa protonica, situata sia sulla membrana plasmatica che sul vacuolo, consuma ATP per estrudere protoni. Non è sempre attiva, ma si attiva in presenza di luce blu. Un recettore specifico percepisce la luce blu e, tramite una catena di trasduzione del segnale, attiva la pompa protonica. L'attivazione della pompa protonica determina l'estrusione di protoni, con conseguente consumo di energia. Questo processo, innescato dalla luce blu, porta all'attivazione di un effettore e all'acidificazione dell'ambiente esterno. L'estrusione di protoni, a sua volta, attiva un altro componente a valle della catena di segnalazione. Dopo il trasporto secondario, l'ingresso di potassio attiva un canale voltaggio-dipendente per il potassio. Avviene una traduzione dei nuovi con altri nuovi nello stesso modo. In seguito all'attivazione della pompa protonica, si attiva una proteina carrier che trasporta potassio. Questo canale voltaggio-dipendente, estrudendo protoni, causa una variazione del pH della membrana plasmatica a seguito dell'espulsione di protoni da parte della pompa protonica. Spostando i protoni al di là della membrana plasmatica, grazie al consumo di ATP e alla pompa protonica, si modifica non solo la concentrazione dei protoni ma anche il potenziale di membrana. Questo cambiamento di potenziale fa sì che i trasportatori, definiti voltaggio-dipendenti, si aprano, permettendo l'ingresso del potassio. Oltre a questo fattore elettrochimico, c'è anche un fattore chimico, legato alla differenza di concentrazione. Il movimento di cariche, come i protoni, attraverso la membrana ha una natura elettrica, influenzando il pH. Il potassio, ione carico positivamente, entra nella cellula per gradiente elettrico e chimico. L'uscita di cariche positive favorisce l'ingresso di cariche positive, quindi il potassio entra più facilmente del cloro. Il sodio, invece, non entra a causa delle sue dimensioni. Successivamente, un trasportatore secondario, un simporto di protoni e zuccheri, trasporta gli zuccheri attivi. L'accumulo di ioni e zuccheri fa diminuire il potenziale negativo della cellula di guardia, richiamando acqua dalle cellule adiacenti. A livello di ordine 6, la luce blu attiva la catena di trasduzione del segnale, che a sua volta attiva la pompa protonica della membrana plasmatica. L'estrusione dei protoni genera un'iperpolarizzazione della membrana, che inattiva il canale del potassio. Contrariamente, l'inattivazione del canale del potassio permette l'ingresso degli ioni $K^{+}$, che partecipano alla formazione del complesso CD.S. Questo processo porta alla diminuzione dell'ossido di over 2. Il protone, entrando nel citosol, trascina con sé molecole osmoticamente attive, come gli zuccheri, aumentando la complessità della situazione. Questa complessità è dovuta all'elevato numero di molecole che, entrando nel citosol, non possono rimanervi a causa dell'incompatibilità con le sue attività. Si attiva quindi la pompa protonica del vacuolo, situata nel tonoplasto. Le molecole osmoticamente attive vengono immagazzinate temporaneamente nel vacuolo, contribuendo al potenziale idrico senza interferire con l'omeostasi cellulare.  L'omeostasi del citosol, quindi il pH e la concentrazione delle sostanze, viene mantenuta perché le molecole in eccesso vengono trasportate attraverso la membrana plasmatica e accumulate nel vacuolo. In questo modo, il potenziale idrico e le condizioni di osmolarità compatibili con la vita della cellula vengono mantenuti. L'apertura stomatica è il meccanismo che permette l'apertura degli stomi e garantisce il progredire della fotosintesi, la movimentazione dell'acqua e altre attività. Il movimento degli ioni, in particolare del potassio ($K^+$), avanti e indietro attraverso la membrana cellulare, mantiene uno squilibrio di concentrazione tra l'ambiente extracellulare e intracellulare. Questo squilibrio, simile a quello mantenuto per gli ioni cloro ($"){Cl}^-$) nella neurobiologia per la trasmissione degli stimoli nervosi, genera correnti elettriche utilizzate per diverse funzioni cellulari.  Il movimento del potassio è regolato e, in risposta a stimoli specifici,  influenza l'apertura e la chiusura degli stomi. Quando lo stoma è ben aperto, a livello di catene ossidriliche, blocca l'ingresso dell'acqua per via dell'umidità. Il tessuto conduttore adulto funzionante in una pianta, chiamato legno, è formato da singole cellule morte dette cellule xilematiche. Per svolgere la funzione di trasporto dell'acqua, è più funzionale che la cellula sia morta, priva di protoplasma al suo interno. Il tessuto più adatto per il trasporto della linfa grezza (acqua e sali minerali disciolti), dovendo resistere a forti pressioni, è il tessuto xilematico, costituito da cellule morte di cui rimane solo la parete. Questo tessuto, caratteristico delle piante terrestri, fa parte dei tessuti vascolari. Le cellule che lo compongono, impilate le une sulle altre, prendono il nome di trachee o tracheidi. La parete cellulare, componente fondamentale di questo tessuto, presenta caratteristiche di spessore e robustezza variabili a seconda del tessuto, ma la sua composizione molecolare rimane costante. La sua funzione principale è quella di conferire protezione e robustezza. Nel caso del tessuto xilematico, la rigidità delle pareti cellulari, composte principalmente da cellulosa, contribuisce al sostegno della pianta. Tutte le cellule vegetali hanno una parete cellulare, come mostrato nell'immagine B della slide iniziale. La forma di una cellula vegetale è determinata dalla sua parete cellulare. C'è un controllo molto preciso su come, dove e quando la parete cellulare si forma, determinando la morfologia della pianta e delle singole cellule. Ad esempio, il tessuto epidermico nella figura B ha una morfologia particolare. Perché le cellule epidermiche in quella figura sono quadrate? Una pestrellatura con un tessuto epidermico che tappa tutto, di forma anche quadrata, garantisce la chiusura. Questo fenomeno, con cellule allineate o di diverse dimensioni, è sotto controllo genetico. Un controllo che fa sì che la parete cellulare in quelle cellule non sia distribuita omogeneamente, ma presenti punti di diverso spessore fino ad agganciarsi con la cellula contigua. Indipendentemente da ciò, la parete cellulare, che siano gli irrobustimenti radiali delle cellule visuard, la parete della rima stomatica o dell'epidermide, ha sempre la stessa composizione. Le molecole che la compongono sono sempre le stesse, quasi tutti zuccheri. Il grosso della parete cellulare è formato da zuccheri. Da qui nasce l'interesse biotecnologico: se si riuscisse a liberare facilmente gli zuccheri incastrati nella parete cellulare, la biomassa, e ad ottenere energia rompendoli, forse potremmo abbandonare l'utilizzo dei biofuels e dei combustibili fossili. In questa struttura c'è un'enorme quantità di energia. Non si opera spesso con il carbone, i fosfati che arrivano da questi, quindi c'è tantissima energia. Il problema è che non si riesce a liberare facilmente e in modo efficiente e non inquinante i singoli zuccheri presenti nella parete cellulare, però ha un'applicazione enorme. Bene, questa è come viene classicamente disegnata una parete cellulare, da libro di testo. Queste sono due immagini, entrambe al microscopio elettronico a trasmissione, che mostrano una cellula circondata da parete nell'immagine in alto a destra. Focalizzando l'attenzione su questa cellula, circondata da una parete, possiamo individuare al suo interno il nucleo. Questo è il nucleolo. Tutta questa parte più estesa è il nucleo. Questi oggetti più grandi sono i cloroplasti. I mitocondri sono presenti in tutte le cellule vegetali, a differenza dei cloroplasti che possono anche non esserci. I mitocondri sono quegli organelli di dimensioni decisamente inferiori rispetto ad altri presenti nella cellula. Questi sono vacuoli. Il nome "vacuolo" deriva dalle prime osservazioni al microscopio che li descrivevano come sacchi vuoti. Sebbene nelle immagini appaiano vuoti, in realtà sono pieni. L'etimologia della parola, però, deriva da "vuoto".  Come dicevo prima, nelle cellule vegetali, il luogo di nascita di una cellula è anche il luogo in cui essa rimane, incastrata in una parete cellulare. Questo impedisce lo spostamento. Il movimento di una pianta, ad esempio verso la luce, avviene tramite la divisione cellulare: la crescita e l'allungamento in una direzione. Le cellule vegetali, a differenza di quelle animali, restano vicine per sempre, bloccate dalla parete cellulare. L'insieme delle pareti cellulari di un organismo vegetale si chiama apoplasto. L'immagine a sinistra mostra un esempio di parete cellulare. La parete cellulare si forma durante la mitosi, la divisione cellulare, definendo il confine tra due cellule figlie. Appena formata la cellula figlia, la parete cellulare è già presente, impedendo ulteriori spostamenti e definendo la sua posizione rispetto alle cellule adiacenti. La prima struttura che si forma nella parete cellulare, essendo la più esterna rispetto alla membrana plasmatica, è la lamella mediana. Questa viene secreta dalla cellula durante la divisione cellulare e, essendo sintetizzata dal protoplasma, spinge la lamella mediana verso l'esterno. La lamella mediana, presente in tutte le cellule vegetali, è lo strato più esterno della parete cellulare ed ha l'aspetto di un gel. Gli zuccheri sono i principali costituenti della lamella mediana e della parete primaria, la seconda parte della parete cellulare che si forma. La parete primaria, priva della consistenza gelatinosa della lamella mediana, è composta da zuccheri diversi o con una maggiore percentuale di componente fibrillare, che conferisce resistenza. Nei tessuti conduttori delle piante vascolari, si forma anche una parete secondaria, più robusta, che spinge verso l'esterno. La formazione della parete secondaria porta alla morte del protoplasta, lasciando un tubo vuoto con una parete spessa, utilizzata per costruire travi e traversine ferroviarie. Quindi, sedendosi su una panca di legno, si sta sedendo sulla parete cellulare. Gli anelli di accrescimento annuali in un frusto di albero sono dati dall'insieme delle pareti cellulari che si formano nel corso degli anni. Lo spessore degli anelli è determinato dalla presenza di una parete cellulare secondaria, caratteristica dei tessuti conduttori. Anche nelle cellule con parete secondaria, la formazione prevede una sequenza di deposizione: lamella mediana, parete primaria e infine parete secondaria. Con l'accrescimento e lo sviluppo della parete cellulare, la lamella mediana viene compressa fino a scomparire. I melanoplasti o altre tipologie cellulari presentano sempre i mitocondri.
\end{spacing}
\end{document}
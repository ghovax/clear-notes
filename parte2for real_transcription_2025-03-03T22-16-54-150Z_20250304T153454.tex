\documentclass[11pt, a4paper]{article}
\usepackage[utf8]{inputenc}
\usepackage[T1]{fontenc}
\usepackage{lmodern}
\usepackage{microtype}
\usepackage[margin=0.75in]{geometry}
\usepackage{parskip}
\usepackage{setspace}
\usepackage{amsmath}
\usepackage{amsfonts}
\usepackage{xcolor}
\usepackage[autostyle, english = american]{csquotes}
\MakeOuterQuote{"}
\setlength{\parindent}{1em}

\title{Transcription}
\author{ClearNotes}
\date{\today}

\begin{document}
\maketitle
\begin{spacing}{1.15}
Questo esercizio, presentato nella slide, può essere svolto applicando diverse condizioni. Si tratta di un esempio di calcolo del potenziale idrico ($\psi_w$), che, in cellule contigue, dipende da due parametri: il potenziale di pressione ($\psi_P$), determinato dalla parete cellulare, e il potenziale osmotico ($\psi_S$), determinato dalla presenza di soluti. Il potenziale osmotico si calcola con la formula $\psi_S = -RT\cdot C_S$, dove $R$ è la costante dei gas, $T$ la temperatura e $C_S$ la concentrazione dei soluti. Questa formula, inversa al potenziale isomorfico a causa del segno meno, descrive accuratamente lo smesi.
Come determinare la $
ewline$ $\footnote{Minor disfluencies were removed to improve the overall readability. Some numerical values were rewritten in standard decimal notation (e.g., zero uno -> 0.1).}
ewline$ di una soluzione? Bisogna conoscere il valore di $R$, costante dei gas, di $T$ che è la temperatura e avere un'idea della concentrazione dei soluti. In questa immagine viene rappresentato un contenitore vuoto con acqua pura. Il potenziale idrico di un contenitore con acqua pura, il valore di $\footnote{The phrase "il periodo piu basilico come riferimento" was difficult to interpret in a meaningful way within the context. It might be a specific term or instruction related to the course, but without further information, it's unclear. It has been omitted in the processed text.}
ewline$ è zero. È zero perché non ci sono soluti, non ci sono pareti, quindi il potenziale idrico dell'acqua è pari a zero. Se sciogliamo una bustina di zucchero, ovvero poniamo una concentrazione di saccarosio all'interno del contenitore.\footnote{Minor disfluencies were removed for clarity. "Aperture somatiche" was interpreted as "aperture stomatiche" (stomata) based on the context. Please verify if this interpretation is correct.}
Il saccarosio è lo zucchero che si usa normalmente nel caffè. Sciogliamo una quantità di saccarosio pari a $0.1$ molare, ottenendo una concentrazione finale di $0.1$ molare. Il potenziale idrico di questa soluzione con saccarosio aggiunto è dato da $\Psi_w = \Psi_s + \Psi_p = -RT \cdot C_s$. Dove $C_s$ è la concentrazione di saccarosio, pari a $0.1$ molare. A $20^"\circ C$, $R$ è una costante e il valore di $\Psi_s$ è $-0.244$ MPa. $\Psi_p$ non contribuisce al potenziale idrico perché non ci sono pareti. Quindi, il potenziale idrico della soluzione con saccarosio $0.1$ molare è $-0.244$ MPa a $20^"\circ C$. In questa soluzione, inseriamo una cellula. Ci sono due casi: inseriamo una cellula turgida, piena d'acqua, e una cellula flaccida.
Queste due cellule si comporteranno in modo differente anche se poste nella stessa soluzione. Nella cellula turgida, dove la turgidità restituisce una concentrazione di $0{,}244$, se la poniamo in questa soluzione, perde acqua verso l'esterno perché la sua concentrazione di soluti è minore e quindi diventa flaccida. Nel caso opposto, tipico di una cellula vegetale, la cellula assume acqua finché il suo potenziale di turgore $\psi_P$ non impedisce un ulteriore ingresso di acqua. Quindi, diventa turgida fino a quando il suo potenziale idrico non eguaglia quello della soluzione. Queste condizioni sono tipiche di un ambiente fisiologico. La figura in alto a sinistra mostra il comportamento di una pianta quando si identifica l'apnea fisiologica. Quando fermata, la pianta si affloscia e perde turgore perché perde acqua. Quindi, una pianta, in parte, vive grazie al contenuto di acqua.\footnote{The sentence "Cioè, un trasporto... le piante devono soffrire, insomma." was slightly difficult to interpret due to the fragmented nature and the conversational marker. It seems like the professor was about to say something about "trasporto" (transport) but then changed the subject to suffering of the plants.  The meaning might be related to the plant needing to experience some stress or hardship (siccità) to trigger the opening of the stoma. However, without further context, the exact intended meaning of "trasporto" in this fragmented sentence remains unclear.}
Se la pianta non viene manipolata eccessivamente, ma semplicemente annaffiata, è in grado di recuperare bene in 24-48 ore, tornando in una condizione di turgidità, con flussi verticali ben turgidi. Questo è possibile perché il potenziale idrico delle radici nel vaso è minore rispetto a quello del suolo annaffiato. Quindi, aggiungendo acqua al suolo, quest'ultimo avrà un potenziale idrico maggiore rispetto alle radici e la pianta assorbirà acqua dall'esterno verso l'interno.  L'ingresso di acqua dalle radici non è sufficiente, altrimenti solo l'apparato radicale diventerebbe turgido. L'acqua deve essere movimentata attraverso tessuti specifici, un adattamento evolutivo dalla colonizzazione delle terre emerse, assente nei muschi. Questi tessuti trasportano l'acqua all'interno della pianta, anche grazie alle aperture stomatiche che permettono la fuoriuscita di acqua, formando una sorta di catena di molecole d'acqua che attraversa l'intera pianta. Questa catena si forma solo se le cellule di guardia degli stomi sono aperte. L'apertura e chiusura stomatica è un meccanismo principale con cui la pianta controlla numerosi eventi fisiologici.\footnote{The original text contained some unclear parts and logical jumps. For example, the initial sentence "Il peso, ma la musica è stata grave" seems out of context and its meaning is unclear. It has been included in the output as it is. There's a jump from discussing the "blue light" to the state of the stomi. It has been interpreted as a cause-effect relationship, assuming the professor was implying that the blue light triggers the initial opening of the stomi. The phrase "passano" in the last sentence was ambiguous and has been interpreted in the broader context of the plant's survival, leading to the addition of "o appassiscono" to complete the possible outcomes.}
L'acqua risale nella pianta principalmente quando le aperture stomatiche sono aperte, permettendo la traspirazione. Il passaggio dell'acqua attraverso la pianta è fondamentale e richiede un grande apporto idrico. Se gli stomi sono chiusi, la pianta non traspira e ristagna. La traspirazione, ovvero il movimento dell'acqua dall'interno all'esterno attraverso gli stomi, è essenziale. Una delle forze trainanti principali che richiama l'acqua dal suolo alla pianta è proprio la perdita d'acqua tramite traspirazione. Questo meccanismo permette alla pianta di mantenere il suo potenziale idrico interno leggermente inferiore rispetto a quello del suolo, garantendo l'assorbimento. Le cellule di guardia, fondamentali per la regolazione del flusso d'acqua e l'ingresso di $\footnote{The spoken language was quite clear and didn't present particular issues in the transcription process. Some filler words were removed and sentences were slightly rephrased for better clarity, while ensuring the original meaning was preserved.}
ewline$ $\footnote{The original text contained several disfluencies and repetitions, which were removed to improve clarity. For instance, phrases like "cioè" and "ehm" were omitted.  The fragmented sentence "per parla- per rimane- il mondo vegetale" was reconstructed as "per rimanere nel mondo vegetale". The question "Siamo guardando la traduzione, ma che cosa succede?" was rephrased as "E, una volta avvenuta la traduzione del segnale, su quale effettore agisce?" to better reflect the intended meaning within the context of the lecture.}
ewline$ anidride carbonica ($\footnote{Il testo originale presenta alcune ambiguità. In particolare, non è chiaro cosa venga attivato dall'estrusione di protoni. La frase "Chi si attiva dopo?" rimane senza risposta. Inoltre, il riferimento all'alba e al "consumo di attività di protoni" è poco chiaro e non si capisce come si inserisca nel discorso sulla pompa protonica.}
ewline$ $\text{CO}_2$), permettono il processo fotosintetico. Queste cellule possono passare da uno stato chiuso a uno aperto modificando il loro turgore cellulare tramite la movimentazione interna dell'acqua. Questa regolazione avviene grazie alla presenza di una rima stomatica con una parete cellulare ispessita, una parete meno ispessita in una regione specifica e una disposizione radiale della parete cellulare. Tale struttura consente l'allargamento o la chiusura delle cellule di guardia in funzione del movimento dell'acqua, garantendo l'apertura o la chiusura stomatica e la trasmissione degli stimoli.
State facendo, come ho già detto, la traduzione del segnale stimolo e vale. Che cos'è che determina l'apertura dello stomaco secondo voi? Quello che vi è scritto nella slide? Sì. Lo stomaco non è sempre aperto, un poro scontato non è sempre aperto, si apre e si chiude. Si chiude in apertura alla siccità, diciamo così, ma non è sufficiente. Cioè, un trasporto... le piante devono soffrire, insomma. Perché sennò non soffrono. Qual è lo stimolo, secondo voi, più importante che innesta l'apertura dello stomaco? Sì? Il peso.
Il peso, ma la musica è stata grave. La luce. Lo stimolo principale, il primo stimolo che in condizioni fisiologiche determina il benessere di una pianta e l'apertura stomatica è la luce. Quindi la luce viene percepita, qui c'è scritto "blue light", perché effettivamente è la luce dell'alba. Gli stomi sono, nel 90%, 92% dei casi, quasi sempre poco aperti. Stanno quasi sempre poco aperti perché le condizioni ambientali per una pianta, se adattata bene, sono abbastanza estreme. Cioè, in estate, se lo stoma fosse sempre aperto, nelle ore centrali della giornata la pianta perderebbe un sacco di acqua. Quindi tendenzialmente nelle ore centrali della giornata gli stomi sono socchiusi. Perché da lì esce l'acqua, quindi o si continuano ad innaffiare queste piante oppure, se no, o le piante vengono attraversate da fiumi di acqua o appassiscono.
Quindi, tendenzialmente, le piante sono più fotosinteticamente attive di giorno, ma la loro efficienza, in termini di quantità di luce assorbita e quantità di carbonio organicato, è maggiore all'inizio della giornata rispetto al centro della giornata. Le forti radiazioni delle ore centrali estive rappresentano una condizione di stress per le piante. La luce fotosinteticamente attiva, oltre ad avere una specifica qualità, ha anche una determinata quantità. Nelle giornate estive di luglio, con temperature elevate e luce intensa, una pianta, non potendosi spostare, soffre molto a causa dell'eccesso di luce, che produce specie reattive dell'ossigeno dannose. La pianta si difende da questo eccesso. Le condizioni migliori per la fotosintesi sono quelle leggermente ombreggiate, con nuvole che attenuano la luce solare diretta. In caso di eccesso di luce, si attivano meccanismi di difesa nella pianta.\footnote{It's unclear what "CD.S." stands for. It could be an abbreviation specific to the lecture context. Also, the phrase "diminuzione dell'ossido di over 2" is unclear. It might refer to a specific chemical compound or process, but without further context, it's difficult to interpret with certainty. It could potentially refer to a decrease in a specific oxide with an oxidation state greater than +2.}
La fotosintesi avviene normalmente, ma la quantità di energia che si traduce in massa è molto inferiore rispetto alla quantità di energia che colpisce la pianta. La luce che induce l'apertura stomatica è la luce blu, tipica dell'alba. Quindi, gli stomi si aprono all'alba e si richiudono durante il giorno per evitare un'eccessiva perdita d'acqua. La luce blu è percepita da un fotorecettore, una proteina in grado di captarla, che avvia la catena di trasduzione del segnale. Le catene di trasduzione del segnale sono simili nel mondo vegetale a livello molecolare. Non approfondiremo una specifica catena per l'apertura delle cellule di guardia, ma è importante capire che la luce blu, e quindi l'alba, è lo stimolo che determina l'apertura degli stomi. Questo stimolo viene tradotto in un'azione specifica: qual è il primo evento che determina l'apertura stomatica? E, una volta avvenuta la traduzione del segnale, su quale effettore agisce? Qual è la proteina che determina l'apertura? La pompa protonica.
La pompa protonica, situata sia sulla membrana plasmatica che sul vacuolo, consuma ATP per estrudere protoni. Non è sempre attiva, ma si attiva in presenza di luce blu. Un recettore specifico percepisce la luce blu e, tramite una catena di trasduzione del segnale, attiva la pompa protonica. L'attivazione della pompa protonica determina l'estrusione di protoni, che a loro volta attivano un processo non specificato nel testo. Il testo menziona inoltre l'alba e il consumo di attività di protoni, ma la relazione tra questi elementi e la pompa protonica non è chiara.\footnote{Il testo originale presentava numerose disfluenze, ripetizioni e passaggi poco chiari. In particolare, la parte finale relativa alla domanda sulla diminuzione del muscolo e l'intervento dello stilo è stata difficile da interpretare e potrebbe non riflettere fedelmente il significato originale. Inoltre, la frase "...sono i chilioni che si spostano..." è stata interpretata come un riferimento agli ioni, ma potrebbe essere una trascrizione errata. Infine, non è chiaro il collegamento tra il movimento del potassio e la domanda finale relativa al muscolo e allo stilo.}
Dopo il trasporto secondario, l'ingresso di potassio attiva un canale voltaggio-dipendente per il potassio. Avviene una traduzione dei nuovi con altri nuovi nello stesso modo. In seguito all'attivazione della pompa protonica, si attiva una proteina carrier che trasporta potassio. Questo canale voltaggio-dipendente, estrudendo protoni, causa un cambiamento del pH della membrana plasmatica a seguito dell'espulsione dei protoni da parte della pompa protonica.\footnote{The initial part of the transcription "Quindi, a un certo punto, quando lo stomo e bello aperto, a livello di catene ossidriche, semplicemente blocca l'ingresso dell'acqua. E un'umidita? No, e un'umidita. Si, si, si, si, si, si, si, si, si, si, si, si, si, si, si, si, si, si, si, si, si, si, si, si, si, si, si, si, si, si, si, si, si, si, si, si, si, si, si, si, si, si, si, si, si, si, si, si, si, si, si, si, si, si, si, si, si, si, si, si, si, si, si, si, si, si, si, si, si, si, si, si, si, si, si, si, si, si, si, si, si, si, si, si, si, si, si, si, si, si, si, si, si, si, si, si, si, si, si, si, si, si, si, si, si, si, si, si, si, si, si, si, si, si, si, si, si, si, si, si, si, si, si, si, si, si, si, si, si, si, si, si, si, si, si, si, si, si, si, si, si, si, si, si, si, si, si, si, si, si, si, si, si, si, si, si, si, si, si, si, si, si, si, si, si, si, si, si, si, si, si, si, si, si, si, si, si, si, si, si, si, si, si, si, si, si, si, si, si, si, si, si, si, si, si, si, si, si, si, si, si, si, si, si, si, si, si, si, si, si, si, si, si, si, si, si, si, si, si, si, si, si, si, si, si, si, si, si, si, si, si, si, si, si, si, si, si, si, si, si, si, si, si, si, si, si, si, si, si, si, si, si, si, si, si, si, si, si, si, si, si, si, si, si, si, si, si, si, si, si, si, si, si, si, si, si, si, si, si, si, si, si, si, si, si, si, si, si, si, si, si, si, si, si, si, si, si, si, si, si, si, si, si, si, si, si, si, si, si, si, si, si, si, si, si, si, si, si, si, si, si, si, si, si, si, si, si, si, si, si, si, si, si, si, si, si, si, si, si, si, si, si, si, si, si, si, si, si, si, si, si, si, si, si, si, si, si, si, si, si, si, si, si, si, si, si, si, si, si, si, si, si, si, si, si, si, si, si, si, si, si, si, si, si, si, si, si, si, si, si, si, si, si, si, si, si, si, si, si, si, si, si, si, si, si, si, si, si, si, si," contains many repetitions and disfluencies. It was simplified to convey the core message clearly.  The long lists of "si" were interpreted as affirmations. The phrase "Vedi che le fa della pianta, va bene. Cosa state citando? Che cos'e che guardate? Stai guardando l'endocardi, cioe state guardando il tessuto conduttore della pianta, perche il tessuto conduttore adulto funzionante in una pianta" seems to be addressed to students and was omitted as it doesn't add information to the core explanation. Finally, the question at the end was rephrased into a statement for better clarity.}
Di sicuro fa cambiare il pH, abbiamo già visto, poi sono protoni, è facile. C'è un altro fattore che entra in gioco: elettrochimico. La differenza di concentrazione è chimica. Il fatto che quello che io sto spostando di qua e di là ha una carica, è elettrico. Quindi, spostando i protoni al di là della membrana plasmatica, grazie al consumo di ATP, grazie alla pompa protonica, cambio la concentrazione dei protoni, ma cambio anche il potenziale di membrana. E quindi quelle proteine che trasportano il potassio si aprono perché sentono un cambiamento elettrico. Ed è per questo che sono chiamati trasportatori elettrovoltaggio-dipendenti. Quindi si apre il canale e permette l'ingresso del potassio. Entra il potassio.
Perché il potassio entra e non uno ione carico positivamente? Per una reazione al gradiente elettrico. Perché chimicamente, ma anche elettricamente, stiamo buttando fuori cariche positive, quindi chi più facilmente manderà all'interno, chi ritornerà più facilmente all'interno delle cariche positive. Quindi sì, entra il potassio, non il cloro, ma il potassio. E tendenzialmente non entra il sodio, questo è un po' un mestiere che deve fare senza il sodio, non entra perché è grosso. Quindi entra il potassio, poi lì dopo c'è un trasportatore secondario, che sarà un'intermembrana simporto, protoni e zuccheri. Gli zuccheri sono specificamente attivi, li abbiamo visti nella slide precedente, per quanto meno è contrattata la concentrazione dei soluti. Quindi gli zuccheri che sono specificamente attivi. Tutto questo insieme di ioni, gli zuccheri, fa sì che il potenziale negativo della cellula di guardia diminuisca e cominci a richiamare acqua dalle cellule adiacenti.\footnote{The initial question "Ah, che dovevo dire? Si, vi siete domandati?" was omitted since it's a conversational marker without relevant information. Some repetitions like "ma veramente molto preciso" and "Come, dove, quando la parete deve essere formata" were removed to improve clarity. The sentence "Quel tessuto epidermico che vedete nella figura B ha una morfologia veramente strana, ma veramente strana" was simplified to "Il tessuto epidermico nella figura B ha una morfologia particolare."}
A livello di ordine 6, la luce blu attiva la catena di trasduzione del segnale. Questa attiva la pompa protonica della membrana cosmatica, che estrude protoni. Si genera così un'iperpolarizzazione della membrana, che inattiva il canale del potassio. Il canale del potassio, contraddipendentemente, fa entrare gli ioni $K^{+}$, che partecipano alla formazione del CD.S. Quindi si ha una diminuzione dell'ossido di over 2.\footnote{The terms "pestrellatura", "rima stomatica", "biofuels", "phosphofuels" were kept in their original form as they represent specific technical terminology that might not have a direct equivalent. "Phosphofuels" was interpreted as a mispronunciation of "fossil fuels" and thus adapted in the processed text.}
Il protone, insieme ad altre molecole osmoticamente attive come gli zuccheri, entra nel citosol. Questa situazione, già complessa di per sé, aumenta in complessità data la quantità di sostanze coinvolte. Queste sostanze non rimangono nel citosol, in quanto ciò interferirebbe con le sue attività. Di conseguenza, si attiva la pompa protonica del vacuolo, situata nel tonoplasto. Le molecole osmoticamente attive vengono quindi temporaneamente accumulate nel vacuolo, contribuendo al potenziale idrico senza alterare l'omeostasi cellulare. In questo modo, il citosol mantiene il suo pH e la concentrazione delle sue componenti. Il potenziale idrico e le molecole osmoticamente attive vengono trasportate all'esterno, attraverso la membrana plasmatica e i citoplasmi, fino al vacuolo.  Così facendo, si mantengono stabili sia il potenziale idrico che le condizioni di osmolarità necessarie alla vita della cellula.
Questo è il meccanismo di apertura stomatica, il sistema che permette l'apertura dello stoma e garantisce il progredire della fotosintesi, la movimentazione dell'acqua e altre attività. Il movimento degli ioni, in particolare del potassio ($K^+$), avanti e indietro tra l'ambiente intracellulare ed extracellulare, genera uno squilibrio di concentrazione. Questo squilibrio, simile a quello mantenuto nei sistemi neurobiologici per la trasmissione degli impulsi nervosi (ioni $Na^+$ e $K^+$), viene utilizzato per generare correnti elettriche.  Questo squilibrio ionico è mantenuto in quasi tutti i sistemi biologici e non è mai in equilibrio. Il potassio, in particolare, si muove avanti e indietro attraverso la membrana cellulare. Per quanto riguarda la domanda sulla diminuzione del muscolo e l'aumento... quando lo stoma è aperto, lo stilo non interviene finché il prodotto non raggiunge le pareti.\footnote{The sentence "Questa e una cellula che e circondata da parete. Questa e la parete." was slightly redundant and has been streamlined for clarity in the processed text.  The question "Qual e il nucleo? Quale?" was interpreted as a prompt to identify the nucleus within the described cell structure.}
Quindi, quando lo stoma è ben aperto, a livello di catene ossidriliche, blocca l'ingresso dell'acqua. È un'umidità? Sì. Il tessuto conduttore adulto funzionante in una pianta, chiamato legno, è formato da singole cellule morte dette cellule xilematiche. Per svolgere la funzione di trasporto dell'acqua, è più comodo che la cellula sia priva di protoplasma.
Stiamo parlando della linfa grezza, non di quella per gli zuccheri. È più comodo avere un tubo che resista alle forti pressioni, come durante la traspirazione dell'acqua. Il tessuto migliore per la connessione della sola linfa grezza (acqua e sali minerali disciolti) è formato da cellule morte, di cui resta solo la parete. Questo tessuto, detto xilematico, fa parte dei tessuti vascolari ed è una caratteristica delle piante terrestri. Le cellule impilate le une sulle altre prendono il nome di trachee o tracheidi. L'argomento dei prossimi 10-20 minuti sarà di nuovo la parete cellulare. Essa presenta caratteristiche differenti a seconda del tessuto, non a livello molecolare, ma di spessore, robustezza e presenza o assenza del protoplasma all'interno. Le molecole che la compongono sono sempre le stesse. La parete cellulare è una struttura esterna alle cellule che conferisce protezione e robustezza. Le trachee del tessuto xilematico conferiscono rigidità alla pianta perché sono pareti di cellulosa rigide, aiutandola a mantenersi dritta e integra.\footnote{The fragment "perche hanno il" was probably an incomplete sentence and its meaning is unclear. It has been omitted in the processed text. The repetitions and speech disfluencies like "eh", "no-no", "questi questi" have been removed to improve clarity.}
Tutte le cellule vegetali hanno una parete cellulare. Come fa una cellula ad assumere forme diverse? Perché assume una determinata forma? La parete cellulare è responsabile della morfologia della pianta e delle singole cellule, controllando precisamente dove e quando la parete deve essere formata.  Ad esempio, il tessuto epidermico nella figura B ha una morfologia particolare. Perché le cellule epidermiche sono quadrate?\footnote{The repetitions and unclear transitions such as "Qui si vede meglio", "Anche qua si vede meglio", "Anche qui si vede meglio", "Che vaco e vuoto",  "Qui e tutto vuoto, e bello pieno, eeeh, pero" were removed to create a more concise and readable paragraph.  The core meaning — that vacuoles appeared empty in initial observations hence the name — is preserved.}
No, non sarebbe più facile una pestrellatura di questo tipo? Cioè, una pestrellatura così mi garantisce un tessuto epidermico che tappa tutto. Può essere quadrata. Questo fenomeno, che sia quadrato con cellule allineate o con cellule di quelle dimensioni, è sotto il controllo genetico, non è casuale. È un controllo genetico ben preciso che fa sì che la parete cellulare in quelle cellule non sia distribuita in modo omogeneo, ma presenti dei punti in cui è più spessa, meno spessa, finché non si aggancia con la cellula contigua. Indipendentemente da tutto, però, la parete cellulare, che siano questi irrobustimenti radiali delle cellule, che sia la parete cellulare della rima stomatica, che sia la parete cellulare dell'epidermide, ha sempre la stessa composizione. Cioè, le molecole che compongono la parete cellulare sono sempre le stesse, non cambiano. E sono, più precisamente, quasi tutti zuccheri. Il grosso della parete cellulare è formato da zuccheri. Ben capite che è da qui che nasce l'interesse biotecnologico. Se tutti quegli zuccheri che sono incastrati nella parete cellulare, che sono la biomassa, si riuscisse a liberarli facilmente, se si riuscisse ad ottenere energia facilmente rompendo quegli zuccheri, forse potremmo anche abbandonare l'utilizzo dei biofuels, dei combustibili fossili. Perché lì, proprio in questa struttura, c'è una quantità di energia enorme.
Non si opera spesso con il carbone, i fosfofori che arrivano da questi. Quindi c'è tantissima energia. Il problema è che non si riesce a liberare facilmente e in modo efficiente e non inquinante i singoli zuccheri presenti nella parete cellulare. Però ha un'applicazione enorme. Bene, questa è come viene classicamente disegnata una parete cellulare, da libro di testo. Queste sono due immagini, entrambe al microscopio elettronico a trasmissione, che mostrano una cellula circondata da parete (immagine in alto a destra).\footnote{Il testo originale presentava alcune ripetizioni e disfluenze che sono state eliminate per rendere il testo più chiaro e leggibile. Ad esempio, la frase "Cioè, appena nasce una nuova cellula figlia, non c'è già la parete cellulare, ok? Non viene formata dopo, subito." è stata semplificata in "quando nasce una nuova cellula figlia, la parete non è ancora presente ma viene subito sintetizzata". Inoltre, la frase "Perche la parete e tutta sintetizzata dalla cellula vegetale, quindi e il protoplasma che sintetizza la propria parete." è stata riformulata come  "la lamella mediana [...] è prodotta dal protoplasma della cellula stessa" per maggiore chiarezza e fluidità.}
All'interno della cellula, focalizzando l'attenzione su questa cellula circondata da una parete, si osserva la presenza di un nucleo. Il nucleo è identificato come questo elemento specifico all'interno della cellula.\footnote{The sentence "Dove non c'e piu questa roba gelatinosa hanno disegnato i miei amici con delle caramelle che sono non sono altro che, di nuovo, zuccheri, sempre zuccheri sono, ma diversi" was simplified as it was difficult to interpret literally due to its grammatical inconsistencies. It was interpreted as referring to the different sugar composition of the primary wall compared to the middle lamella.  The phrase "con una percentuale, si dice, della componente fibrillare, la componente che da la mistezza, molto maggiore" was cleaned up for clarity to highlight the higher percentage of fibrillar components contributing to the rigidity of the primary wall.}
Questo è il nucleolo. Quindi tutta questa parte è il nucleo, quello grosso, è tutto il nucleo. Questi cosa sono? Questi sono... Bravissimi, l'anno scorso non ci avevano preso. Bravi, non so chi l'ha detto, bravi. Questi oggetti giganti sono i cloroplasti.
Cosa sono? Non ho sentito, ripeti. I mitocondri. Ricordatevi che le cellule vegetali hanno sempre i mitocondri. Possono non aver i cloroplasti perché hanno... Quindi, quelli più piccoli, che sono decisamente più piccoli, sono i mitocondri. Questi.
Questi sono vacuoli. Il nome vacuolo deriva dal fatto che le prime osservazioni indicavano un sacco vuoto. Nelle immagini il vacuolo si presenta principalmente come vuoto, sebbene in realtà sia pieno.
Bene. Questa cellula, se vedete, ed è questo quello che vi dicevo prima, qui secondo me è più evidente, forse incominciate a farvene una ragione, nelle cellule vegetali, come vi ho detto, dove nascono restano perché sono sempre incastrate in una parete, sempre, quindi non si possono spostare. Una cellula vegetale, una pianta si sposta, cioè si sposta in direzione della luce, è dotata di un movimento, ma è un movimento che avviene attraverso la divisione cellulare. Quindi io mi accresco, continuo a dividermi e mi allungo in quella direzione. Ma quella cellula che è nata lì, dove vicino a sé, voi tre siete tre cellule, se siete nati lì, voi tre sarete vicini per sempre. Cioè non vi potete spostare perché siete bloccati da una parete cellulare che non vi permette di spostarvi e viceversa. Siete incastonati all'interno di una parete cellulare e abbiamo detto che tutta la parete cellulare dell'intero organismo prende il nome di apoplasto. Bene. La parete cellulare ingrandita invece, quell'immagine a sinistra, qui si vede una bella paretona cellulare che è tutta questa struttura qua.
La lamella mediana è la prima struttura che si forma nella parete cellulare durante la divisione cellulare (mitosi). Essa divide due cellule figlie e, essendo la prima formata, è la più esterna rispetto alla membrana plasmatica. Le pareti cellulari si formano durante la mitosi: quando nasce una nuova cellula figlia, la parete non è ancora presente ma viene subito sintetizzata, dopodiché la cellula non si sposta più. La lamella mediana, essendo la struttura più esterna della parete cellulare presente in tutte le cellule vegetali, è prodotta dal protoplasma della cellula stessa e viene rappresentata come una sorta di gel.
Gli zuccheri sono i componenti principali della lamella mediana e della parete primaria. La parete primaria, che si forma successivamente alla lamella mediana, è caratterizzata da una struttura meno gelatinosa e da una maggiore componente fibrillare, anch'essa composta da zuccheri ma con una diversa composizione percentuale. Nei tessuti conduttori delle piante vascolari, si forma anche una parete secondaria, più robusta, che spinge la parete primaria verso l'esterno. La formazione della parete secondaria porta alla morte del protoplasta, lasciando un tubo vuoto con una parete spessa. Questo materiale resistente è utilizzato per costruire oggetti come travi e traversine ferroviarie. Quindi, sedendosi su una panca di legno, ci si siede sulla parete cellulare.
Quegli elementi vanno ad insieme di pareti cellulari. Quando si contano gli anni in un frusto, si contano le pareti cellulari che si sono formate nel susseguirsi degli anni. E quando diventano così spesse, sono sostanzialmente date dalla presenza di una parete secondaria, caratteristica dei tessuti conduttori. Indipendentemente dalla presenza della parete secondaria, la cellula che ne è dotata ha avuto un momento in cui aveva la lamella mediana, la parete primaria e poi la parete secondaria. Più si sviluppa la parete, più si accresce, più scompare la lamella mediana, perché a un certo punto viene schiacciata e scompare. Gli zuccheri saranno trattati la prossima settimana.
I melanoplasti o altre cose, ma hanno sempre i mitocondri.
\end{spacing}
\end{document}